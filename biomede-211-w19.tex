\documentclass[11pt]{book}
\usepackage{amssymb,amsmath}
\usepackage{minitoc}
\usepackage{hyperref}
\usepackage{graphicx}
\usepackage{xcolor}
\usepackage{pdfpages}


\title{Notes on BIOMEDE 211, or: \\ Circuits, Systems, \& Signals \\ in Biomedical Engineering}
\author{Barry Belmont}

\begin{document}
\frontmatter
\maketitle
\dominitoc
\tableofcontents


\newpage 
\section{How can I print off and use this document?}
Frankly, in just about any way that’s useful to you. I am going to try something here, where I will try to make more or less the entirety of the notes associated with the Winter 2019 semester of BIOMEDE 211, ``Circuits, Systems, and Signals in Biomedical Engineering'', to you, dear reader.\\
\\
Please don’t plagiarize this. If you were raised right, you ought to know what that is. If you’d like my judgment on any sort of action, my opinions can be laid bare.\\
\\
The first assignment I am giving you (worth 4\% of your grade and which must be completed by the end of the semester) is to figure out where this document is located online, joining the Github, and making at least one contribution to this repository. Failure to contribute to this living document by the end of the semester for those in this class will result in a loss of up to 4\% of one's total grade outright.

\newpage
\section{How can I contribute to GitHub?}
Follow these general steps to propose a change to this online document:
\begin{enumerate}
\item Create a GitHub account
\subitem This should be rather self-explanatory. Use your e-mail account and verify it to be able to edit. You should proceed with the following steps while logged onto your account.
\item Find Dr. Belmont's GitHub page and go to the biomede-211-w19 repository (``repo''). Then click on the biomede-211-w19.tex file.
\item Edit the file
\subitem You will find a small pencil icon on the right side of the page. Click on this to create your own branch (``forking''), and edit the file as you wish.

\includegraphics[width=\textwidth]{figures/GitHub_Tutorial_1.png}

\item Propose file change
\subitem After making your changes, you should scroll to the bottom of the page, find the message box that says, 'Propose file change', and fill it out. The first line should say what you have updated and can be explained in the description.

\includegraphics[width=\textwidth]{figures/GitHub_Tutorial_2.png}

\subitem For those of you that are trying to figure out how to add your board pictures to the Github, the code to add images is as follows. To use the following code, you need to make sure the images are uploaded to the "figures" folder in the Github. Also, you will need to replace ``INSERT\_FILE\_NAME\_HERE'' with the name of the image file and ``FILE\_EXTENSION'' with the extension of the time (e.g. png or jpeg). You will need to copy and paste this code for each image you want to include. 

\begin{verbatim}
\includegraphics[width=textwidth]{figures/INSERT_FILE_NAME_HERE.FILE_EXTENSION}
\end{verbatim}

\item Create pull request
\subitem After finishing your file, you will be brought to a page that displays what you have modified on the original document. Press the green `Create pull request' button to let Dr. Belmont know that you want to create a change. Once he has approved via his own GitHub account, your changes should now be in the updated master branch!

\includegraphics[width=\textwidth]{figures/GitHub_Tutorial_3.png}
\includegraphics[width=\textwidth]{figures/GitHub_Tutorial_4.png}
\includegraphics[width=\textwidth]{figures/GitHub_Tutorial_5.png}

\item Uploading Images
\subitem To upload images to the Figures folder, you will need to locate your "fork" of the "biomede-211-w19" repo and commit changes there. To get there, click on your profile in the upper right corner and click on "Your Repositories". Then click on " biomede-211-w19" in the center of the screen. 

\includegraphics[width=\textwidth]{figures/GitHub_Tutorial_6.png}

\subitem Then, you'll want to click on the Figures folder and then click on Upload files in the top right. Upload your image files, scroll down, and "commit" your changes. 

\includegraphics[width=\textwidth]{figures/GitHub_Tutorial_7.png}

\subitem Then, click on "Pull Request" at the top and create a new pull request. Submit that pull request the same way that you submitted the other one, and you're good to go!

\includegraphics[width=\textwidth]{figures/GitHub_Tutorial_8.png}

\end{enumerate}

\newpage



\section{Who comprises this class and how can they be reached?}
\subsection{The Captain at the helm}
Barry Belmont \\ Wednesdays 11:00 a.m. | 1:00 p.m., 2130 LBME \\ belmont@umich.edu \\
\subsection{The A-Team}
Annabelle St. Pierre \\ Wednesdays 5:00 p.m. | 6:30 p.m., UGLI basement \\ astpierr@umich.edu \\
\\
Alice Tracey \\ Wednesdays 4:00 p.m. | 5:00 p.m., UGLI basement \\ atracey@umich.edu \\

\subsection{You, yourselves}
In this class, we will be learning a lot from each other. You are encouraged to learn from one another. You are encouraged to talk to one another. You are are encouraged to share ideas and at times data. You are not encouraged and are hereby expressly forbidden to submit the work of another as your own. If you get help from others, you will put their name on it somewhere. Too much of this and you are committing plagiarism, not enough and you are committing fraud. Please be honest and let's all learn together.
\begin{enumerate}
	\item Kristian A.
	\item Matt A.
	\item Rayna B.
	\item Ellianna B.
	\item Megan B.
	\item Ashley B.
	\item Dawn C.
	\item Weihong C.
	\item Caila C.
	\item Jenna D.
	\item Sandra D.
	\item James G.
	\item Natalie H.
	\item Yazmin H.
	\item Felicia H.
	\item Isabel H.
	\item Danica J.
	\item Surabhi J.
	\item David K.
	\item Erica K.
	\item Eunjeong L.
	\item Annie L.
	\item Ian M.
	\item Madelynn M.
	\item Devak N.
	\item Likitha N.
	\item Bree P.
	\item Neelay P.
	\item Matei P.
	\item Shaunak P.
	\item Francesca Q.
	\item Raahul R.
	\item Abigail R.
	\item Alexander R.
	\item Andrew R.
	\item Elizabeth R.
	\item Kiana S.
	\item Ryan S.
	\item Sydney S.
	\item Hao S.
	\item Andrew S.
	\item Elijah S.
	\item Cara S.
	\item Aparna S.
	\item Alec S.
	\item Madison W.
	\item Jordyn W.
\end{enumerate}


\section{What is this class about?}
From the course catalog, ``Students learn circuits and linear systems concepts necessary for analysis and design of biomedical systems. Theory is motivated by examples from biomedical engineering. Topics covered include electrical circuit fundamentals, operational amplifiers, frequency response, electrical transients, impulse response, transfer functions and convolution, all motivated by circuit and biomedical examples. Elements of continuous time-domain and frequency-domain analytical techniques are developed.''

From your instructor's heart, "You’re going to learn the fundamental basis of electricity and its specific applicability to biomedical engineering, broadly. With this knowledge, you will have a command over a vaster swath of this world than most as you’ll know something important about how it works: namely, how human beings push electrons around the world to do some quite interesting things and how those electrons push back. This has relevance to a great many disparate situations: from the swelling of potential as your heart dances in your chest to nearly every source of light you’ve ever seen sans the sun (and even then...). All this to say, what we do here matters. And we’ll aim to do it well."

\section{When and where does this class meet?}
Tuesdays and Thursdays, 12:30 -- 2:30 p.m., 1006 DOW

\section{What is required for this class?}
\subsection{Writing materials}
A good pen and some sheets of paper ought to do.

\subsection{Access to the internet periodically}
A good bulk of our class’s administration will be done through Canvas. Check it regularly.

\subsection{A textbook of some kind}
Getting more than one perspective on a topic helps most people learn better. As such, in addition to the textbook which you are currently reading, I recommend consulting another electric circuits text. We have, in the past, used \textit{Fundamentals of Electric Circuits}, 6th edition, by Charles Alexander and Matthew Sadiku, which is also used by our EECS brothers and sisters. Digital or hardcopy is fine. As are older editions or equivalents.

\section{What determines the grade?}
\subsection{The abject material of the the thing}
Grades are determined essentially through three distinct types of assessment. 
\begin{itemize}
	\item The first and foremost of these are out-of-class based assignments (henceforth, ``homework''), individually worth 8 percent, and collectively worth 48 percent of the overall grade.
	\item The second and secondmost of these are three in-class based assignments (henceforth, ``glorious quizzes/exams'), which taken together worth 44 percent of the overall grade. 
	\item The final type is the evaluation of a certain in-class/out-of-class \textit{je’nais se quoi} quality of the participatory student. ``Participation'' is something I take seriously in my classes and in this class will consist of (1) at least one in-class demonstration of your solutions to problems posed during the lecture period and (2) contribution to this specific document through the GitHub. Each of these forms of participation are worth 4 percent of the grade.
\end{itemize}

\subsection{The largely arbitrary, but nevertheless significant scale by which the grade will be measured by the instructors}
As this is my second time teaching this course, some calibration (on a per assignment and/or class- wide basis) may be required. That said, I plan on grading this class just as you might expect any other engineering class to do so against the following scale: \\
\\
\begin{center}
	A+ $\geq$ 97; A $\geq$ 94; A- $\geq$ 90; \\
	B+ $\geq$ 87; B $\geq$ 84; B- $\geq$ 80; \\
	C+ $\geq$ 77; C $\geq$ 74; C- $\geq$ 70; \\
	D $\geq$ 60; F $\geq$ 50
\end{center}

\subsection{The policy regarding missed chances, do overs, and dishonesty}
\begin{itemize}
	\item \textbf{If you will be absent} for some key assessment (a homework, an exam, participation), let the instructor know ahead of time and we can work something out.
	\item \textbf{If you feel the assessment of your work was wrong, misguided, or unfair}, let the instructor know immediately and specifically (within 48 hours of the assessment being returned) and we will work something out. 
	\item \textbf{If you insist at any point on swindling yourself out of an honest education}, please reconsider your decisions to do so (perhaps by discussing the matter with your instructor), otherwise this whole engineering thing for you is not going to work out. You are encouraged to read the Honor Code of the institution to which you are responsible.
\end{itemize}

\section{What are the objectives of this class?}
\begin{enumerate}
	\item To generate physical understanding of fundamental circuit and systems concepts.
	\item To relate classroom material to real-world applications including selected biomedical systems.
	\item To teach students basic circuit and linear systems including transient, frequency response, impulse response, and transfer functions.
	\item To introduce mathematical concepts necessary to accomplish the above including convolution, Laplace transforms, and Fourier transforms.
\end{enumerate}

\section{What are the outcomes I can expect of myself from this class?}
\begin{enumerate}
	\item Learn basic circuit and systems techniques necessary for understanding biomedical instrumentation systems, bioelectrical systems, and medical imaging systems.
	\item Develop an insight into systems analysis techniques motivated by development of electrical circuits concepts.
	\item Develop an insight into operational amplifier analysis and design techniques as motivated by a series of practical biomedical amplifier designs.
	\item Understand mathematical tools necessary for linear systems and circuit analysis/design.
\end{enumerate}



\mainmatter
\setcounter{page}{1}



\part{Circuits}



\chapter{I. Potential, current, energy, conservation}
01/10/2019
\minitoc



\section{What is electricity?}

\begin{enumerate}
	\item A form of energy resulting from the existence of charged particles 
	\item The physical phenomena arising from the existence, presence, and motion of charged particles
	\item Rather ill-defined in common vernacular – we will generally avoid its use
\end{enumerate}



\section{Charge}
 
\begin{enumerate}
	\item Charge is the property of matter that causes it to experience a force when placed in an electromagnetic field; measured in coulombs (C)
	\item 	Charges are found in nature in discrete, integral multiples of electronic charge: e  = -1.602 x $10^{-19}$ C (the charge of one electron)
	\item \textbf{How many electrons are needed to form one coulomb?} (What is the weight of all those electrons?) 
	\item One byte is eight bits. Bits are essentially a single electron stored in a transistor. \textbf{If we were to take all the electrons from one terabyte of well distributed information (equal number of ones and zeros), how many coulombs would we have?}
\end{enumerate}


\section{Current}
\begin{enumerate}
	\item The time rate of change of charge – charges (charged particles) in motion; measured in amperes; defined mathematically as
	\begin{equation}
	\label{i=dqdt}
		i := dq/dt
	\end{equation}
	where $i$ is current, $q$ is charge, and $t$ is time
	
	\item Conversely, the total charge transferred over time can be expressed as 
	\begin{equation}
	\label{q=it}
		Q := \int_{t_0}^{t}i dt
	\end{equation}

	\item 1 ampere is equal to 1 coulomb/second
	\item Direct current, ``DC'', is current that remains constant with time
	\item Alternating current, ``AC'', is current that varies sinusoidally with time
\end{enumerate}


\subsection{The directionality of current}
Ultimately, the direction in which we say "current" flows is largely arbitrary. As arbitrary as choosing one type of charge and calling it ``positive'' and another ``negative''. The reason it doesn't matter is that the only consequence of having chosen a ``wrong direction'' for the current in a given analysis is that we have to switch the sign of the value. Thus, 3 amps in one direction is \textit{the exact same thing }as -3 in the opposite direction.
\begin{enumerate}
	\item Thanks to Benjamin Franklin we say that current is 
	\subitem i.	\textbf{Positive in the direction in which positively charged particles flow} and 
	\subitem ii.	\textbf{Negative in the direction in which negatively charged particles}
	\subitem iii.	We also now know that current results primarily from the movement of negatively charged particles (electrons) and therefore our convention is “wrong” in one sense, though convenient and entrenched enough that we’re not liable to change it in our life time (besides, the math comes out the same, and the actual flow of electrons will only matter to us in a few special circumstances, diodes)
\end{enumerate}


\subsection{The at times deadly serious nature of current}
Much of the point of learning this material here is its eventual application by our hands or by the hands of those we work with. Before we put any of this stuff in our hands, we should probably know what is and is not safe.
\begin{enumerate}
	\item 1 mA, you will feel
	\item 10 mA, you will really feel
	\item 100 mA, you will likely die
	\item 1000 mA, you will definitely die
\end{enumerate}


\subsection{The ``speed'' of current}
A possible misconception is that the electrons inside a wire travels at the speed of light. The speed of current is actually relatively slow. If one were to imagine an electron starting at the wire next to a light switch in an average classroom, it would take a very long period of time for it to travel to the light itself. The light's immediate reaction to a switch is due to a ''hose effect''; the electrons inside the wire push other electrons in the direction opposite to the [conventional] current. This cascade of electrons is what happens close to the speed of light, not the electron movement itself.

\begin{enumerate}
	\item T\textbf{he \textit{signal} of electrical current (that is electromagnetic radiation) travels anywhere between about 50-99\% the speed of light} (dependent on a number of conditions) depending upon the material through which it travels (based on a “dielectric” behavior known as permittivity)
	\item \textbf{The drift velocity of electrons} within a copper wire is ~25 $\mu$m/s, so how does anything ever turn on?
	\item \textbf{The hose effect} - The electrons at the light switch will almost certainly never pass through a light bulb, but they will move around and bump into their neighbors which bump into their neighbors which bump into their neighbors, etc., until it causes the electrons nearest the light to pass through. This is how water at a spigot is able to push water at the end of a hose.
	\item \textbf{How current and drift veolcity related?} - Mathematically, it's represented as follow:
	\begin{equation}
		I = \frac{Q}{t} = \frac{neAd}{d/v_d}
	\end{equation}
	Imagine there is a cylindrical wire with length d and cross-sectional area A. Suppose there are n electrons per unit volume and each with a charge of e. Then in the whole cylinderical wire, the total charge is derived by multuplying the total volume Ad, the number of electron per unit volume n, and the elementary charge e. This is the numerator part of the story. On the denominator is the time component. Assume all electrons are moving in one direction and with drift veolcity. It would take t amount of time for the last electron to move a distance of d from one end to the other and exit the cylinderical wire. In fact, this is how we, in a microscopic sense, perceive the concept of current, which is defined by the total charge that passes through a single reference cross-section over a given amount of time. 
\end{enumerate}



\section{Potential (difference)}
\begin{enumerate}
	\item The amount of work needed to move a unit of (positive) charge from a reference point to another point [without producing an acceleration]).
	\item Potential is measured in ``volts'' and is often called ``voltage''. In this class we will endeavor to avoid such a term as it can be very confusing to talk about potential as if there were such a \textit{thing} as voltage.
	\item Defined as 
	\begin{equation}
		v:= \frac{dw}{dq}
	\end{equation}
	\item Potential describes the \textit{potential} to do something. Increasing the potential is akin to increasing the height of a cliff. The height does not \textit{do} anything other than increase what can be done on the drop. If potential is the cliff's height, charge would be pebbles you'd drop off the side, and current describes how fast those pebble fall.
	\item In this class, and for the vast vast majority of electrical engineering work, we care about the \textit{difference} in potential. One element held at 100 billion volts and another held at 100 billion + 1 volts has a \textit{potential difference} of 1 V, which is less than a single AA battery.
	\item Voltage can also be thought of as how badly the current "wants" to flow, while current is the actual flow of charges per second. Since charge flows but not the voltage, voltage can exist without current - a single charge induces a voltage. On the other hand, current can't exist without voltage since having a current means that charges are flowing, and if charges are flowing there is a potential difference across the charges. 
	\item Some typical voltages to be aware of
	\subitem \textbf{Consumer level batteries} (AA, AAA): 1.5 V (DC); 9 V (DC) 
	\subitem \textbf{Car batteries}: 12 V (DC)
	\subitem \textbf{The ``mains''} (levels provided by power companies to consumers): 110-120 V (AC) and 220-240 V (AC) in America
	\subitem \textbf{Power transmission lines}: 110-1200 kV (AC), transformers are used to step up and down the potential before used by consumers
\end{enumerate}



\section{Power}
\begin{enumerate}
	\item The time rate of expending or absorbing energy.
	\item Quantifies the rate of energy transfer.
	\item Mathematically:
	\begin{equation}
		p = \frac{dw}{dt} = \frac{dw}{dq}\cdot\frac{dq}{dt} = v\cdot i
	\end{equation}
	\item Measured in watts: 1 W = 1 $\frac{\text{J}}{\text{s}}$ = 1 $\frac{\text{N}\cdot \text{m}}{\text{s}}$ = 1 $\frac{\text{kg}\cdot \text{m}^2}{\text{s}^3}$ = 1 V $\cdot$ 1 A
	\item \textbf{Passive sign convention}: If current enters through the positive terminal of an element, $p = +vi$; if current enters through the negative terminal of an element, $p = -vi$.
\end{enumerate}



\section{Energy}
\begin{enumerate}
	\item The capacity to do work.
	\item Measured in joules.
	\item $E = \int\frac{dw}{dt}dt \rightarrow$ power x time
	\item J = $\frac{\text{kg}\cdot\text{m}^2}{\text{s}^2}$ = N $\cdot$ m = Pa $\cdot$ m$^3$ = W $\cdot$ s = C $\cdot$ V
\end{enumerate}



\section{Conservation}
Here, as elsewhere, things will be conserved. In electrical circuits there are two laws of conservation that will matter most for us:
\begin{enumerate}
	\item \textbf{The Conservation of Mass.} The conservation of mass means that no mass can be added to or removed from a circuit without being accounted for. Put differently, in a closed system (the type we will concern ourselves with here) no mass is added or removed.
	\subitem In electrical circuits, the mass we care the most about are the charges whipping around. Thus, for us, \textit{the amount of charge within a circuit must remain constant.}
	\item \textbf{The Conservation of Energy.} The conservation of energy means that no energy can be added to or removed from a circuit without being accounted for. Put differently, in a closed system (the type we will concern ourselves with here) no energy is added or removed.
	\subitem In electrical circuits, the energy we care the most about is the potential provided by sources and depleted by other elements in the circuits. Thus, for us, \textit{the sum of potentials within a circuit must equal zero.} 
\end{enumerate}

In evaluating circuits, the main focus of the first third of this class, it will be the application of these two conservative laws that will enable us to ``solve'' them. That is, by understanding (1) how energy is generated and used and (2) how charges move around in closed loops (``circuits'') we will be able to predict the behavior of the myriad electrical systems which may cross our paths.


 
\newpage



\section{Worksheet}

\subsection{A constant charge through a cross-section}
How much charge passes through a cross-section of a conductor in 60 seconds if a DC current value is measured at 0.1 mA?
\textbf{Solution}


\subsection{An arbitrary charge through a cross-section}
Determine the total charge entering a terminal between $t = 0$ seconds and $t = 10$ seconds if the current (in amps) passing through is
\begin{equation}
	i(t) = \frac{1}{\sqrt{5t+2}}. 
\end{equation} 
\textbf{Solution}


\subsection{A ``tera''ble puzzle}
Approximately how much current is necessary to transmit one terabyte of information in an hour?
\textbf{Solution}


\subsection{A pacemrker's power requirements}
A cardiac pacemaker will provide approximately 5,000 J of energy over 5 years. Determine the capacity of a 5 V lithium battery necessary to drive this pacing such that only 40\% of its energy is spent over that time.
\textbf{Solution}


\subsection{A neuron's excitation energy}
A colleague of yours has been in their lab ginning up new neurons. You, as their resident electrical expert, are tasked with determining the energy consumed by the cell. If the current and voltage variations are found to be functions of time ($t \geq 0$)
\begin{eqnarray}
	i(t) = 3t \\
	v(t) = 10 e^{6t}
\end{eqnarray}
determine the energy consumed between 0 and 2 ms.
\textbf{Solution}


\subsection{A thump to the chest}
(a) A typical defibrillator delivers 200-1000 V in less than 10 ms. How much current is needed to deliver 120, 240, and 360 Joules?
\\
(b) A human heart ways about 300 grams. From approximately how high of a cliff would one have to drop a heart such that the impact was equivalent to the energy delivered to someone's chest from a defibrillator?
\textbf{Solution}



\chapter{An introduction: II. Circuit elements}
\minitoc
01/15/2019
\section{Active v. passive}
\begin{enumerate}
	\item Active elements are capable of generating energy while passive components cannot
	\item \textbf{Active}: generators, batteries, operational amplifiers, ``sources''
	\item \textbf{Passive}: resistors, capacitors, inductors, i.e., most circuit elements
\end{enumerate}

\section{Ohm's Law and what it means}

Ohm's Law is concerned with the relationship between voltage, or potential difference, and current across a conductor.  The potential difference across a conductor is proportional to the current flowing thorugh the conductor with the proportionality constant being denoted as R, or resistance.  This can be expressed as:

	\begin{equation}
	\label{V=iR}
		V := iR
	\end{equation}
	
This essentially states that the drop in potential across the conductor, or resistor, is equivalent to the current flowing through the conductor and its resistance.  When considering impedance, the equation can be modified to state:

	\begin{equation}
	\label{V=iZ}
		V := iZ
	\end{equation}

\section{Sources}
\begin{enumerate}
	\item \textbf{An ideal independent source} is an active element that provides a specified value of potential or current, regardless of other circuit elements.
	\subitem Batteries and power supplies may be approximated as ideal potential sources.
	\item \textbf{An ideal dependent (or controlled) source} is an active element in which the source quantity is controlled by another quantity (such as potential, current, temperature, measured resistance, etc.).
	\item \textbf{An ideal potential source} will produce any current required to ensure that the terminal voltage stated is satisfied.
	\item \textbf{An ideal current source} will produce any voltage required to ensure that the terminal current as stated is satisfied
	\item Symbols
	\subitem Voltage-controlled voltage source, VCVS
	\subitem Current-controlled voltage source, CCVS
	\subitem Voltage-controlled current source, VCCS
	\subitem Current-controlled current source, CCCS 
\end{enumerate}

\section{Resistors}
\textbf{Resistors} are electrical (circuit) elements that resist the flow of electric charge (current); passive two-terminal components that implement a defined/``constant'' resistance; meant to reduce current flow and change potential

\subsection{Resistance, $R$}
\begin{enumerate}
	\item \textbf{Resistance} is the physical property describing an element's ability to resist current and is most often represented by $R$
	\item Resistance is measured in ``ohms'', $\Omega$, which is equivalent to 1 V/A
	\item Resistance is one half of a broader physical phenomenon known as \textbf{``impedance''} - the property describing an element's ability to \textit{impede} current. Impedance is typically represented by $Z$, which we'll explore more thorough in a bit.
\end{enumerate}

\subsection{Resistivity, $\rho$}
\begin{enumerate}
	\item The resistance of an element (such as a resistor) depends on three things: 
	\subitem \textbf{Resistivity}, $\rho$, of the material comprising the element, which is the \textit{material}'s ability to resist the flow of charges; measured in ohm-meters
	\subitem \textbf{Length}, $l$, of the element; measured in meters
	\subitem \textbf{Area}, $A$, of the cross-section of the element; measured in m$^2$
	\subitem Such that $R = \rho\frac{l}{A}$
	\subsubitem What units are we left with?
	\subsubitem What are the effects of length and area?
	\item \textbf{Materials with low resistivity} are generally called (and treated as) ``conductors'' as they are able to more effectively \textit{conduct} the motion of electrical charges than materials with high resistivity
	\item \textbf{Materials with very high resistivity} are generally used as ``insulators'' as they prevent the flow of current through them and thus \textit{insulate} the current within prescribed bounds, such as with a copper wire with plastic wrapped around it.
\end{enumerate}

Here is a link to a video that further explains the concepts of resistivity and resistance: \url{<https://www.youtube.com/watch?v=4rsswT_Rv1M>}.

\subsection{Conductance}
\begin{enumerate}
	\item The inverse of resistance is conductance, $G$, which describes the ability of an element to conduct current
	\item Measured in Siemens
	\item Allows us express Ohm's law slightly differently, $i = Gv$, which says that the current generated through an element by a potential is directly proportional to some constant, namely conductance.
	\item The material specific property \textbf{conductivity}, $\sigma$ is measured in S/m
\end{enumerate}

\section{Capacitors}
\begin{enumerate}
	\item Passive two-terminal components that store energy in an electric field; introduces capacitance to a circuit.
	\item Can be thought of as two conductive plates sandwiching a ``dielectric'' material. Essentially it is two ``conductors'' separated by a ``non-conductive region''.
	\item When a capacitor is attached across a source, an electric field develops across the dielectric causing a net positive charge to collect on one conductor and a net negative charge to collect on the other.
	\item We can define the capacitance of an element mathematically as 
	\begin{equation}
		C = Q/V
	\end{equation}
	where $C$ is capacitance in farads, $Q$ is positive or negative charge on each conductor, and $V$ is the potential between them
	\item We can also represent capacitance by the voltage-based rate of charge accumulation: $C = dQ/dV$.
\end{enumerate}

\subsection{Its time varying behavior}
Unlike resistors, capacitors have a \textit{time-varying} element to that. That is, since $C = Q/V$, $V = Q/C$.

If we then recall Equation \ref{q=it}, we can write the time-dependent potential relationship of a capacitor
\begin{equation}
	V(t) = \frac{Q(t)}{C} = \frac{1}{C}\int_{t_0}^{t} i(\tau) d\tau + V(t_0)
\end{equation}

We can also recall\footnote{We could also take the derivative of the equation preceding this one and do a little rearrangement. As it turns out, these physical relationships are rather codified and thus can be gotten out by any number of means.} Equation \ref{i=dqdt}, and represent the time-dependent current relationship as
\begin{equation}
	I(t) = \frac{dQ(t)}{dt} = C\frac{dV(t)}{dt}
\end{equation}

\subsection{Charge accumulation}
\begin{enumerate}
	\item While charges accumulate on a capacitor, no current flows \textit{through} the capacitor.
	\item \textbf{Well, then why use them? After awhile won't the current just stop?} Yes, indeed it will -- in a DC circuit!
	\item The capacitor will become ``charged'' over time, eventually reaching the same potential as that established across it, e.g., by a source. Since potential only ever travels down potential gradients, if the capacitor and the source (say, a battery) are at the same potential, no current will flow.
	\item Thus, a fully charged capacitor will act as an ``open'' circuit, while an uncharged capacitor will act as a ``short'' circuit.
\end{enumerate}

\subsection{A simple example}
If we consider Ohm's law for a simple RC circuit (one in which a source, a resistor, and a capacitor are in series), we can describe the system by

\begin{align}
	V_0 &= v_{R}(t) + v_{C}(t) \\
	V_0 &= i(t)R + \frac{1}{C}\int_{t_0}^t i(\tau)d\tau
\end{align}
Taking the derivative of both sides:
\begin{align}
	0 &= R\frac{di(t)}{dt} + \frac{1}{C}i(t) \\
	0 &= RC\frac{di(t)}{dt} + i(t) \\
	i(t) &= \frac{V_0}{R}\cdot e^{-t/RC} \\
	v(t) &= V_0\left(1 - e^{-t/RC}\right) \\
	Q(t) &= C\cdot V_0\left(1 - e^{-t/RC}\right)
\end{align}


\section{Inductors}

\begin{enumerate}
	\item Passive two-terminal components that store energy in a magnetic field
	\item Can be thought of as an insulated wire wound into a coil around a core (which may either be filled with a material or left open to the environment)
	\item Behavior can be modeled as $L = \frac{\Phi}{I}$, where $L$ is the inductance, $\Phi$ is the magnetic flux generated by a current, $I$.
	\item By Faraday's law of induction, voltage induced by a change in magnetic flux through a circuit is
	\begin{equation}
		v = \frac{d\Phi}{dt}
	\end{equation}
	which we can rewrite as
	\begin{equation}
		v = \frac{d}{dt}(Li) = L\frac{di}{dt}
	\end{equation}
	\item In this class, at this level, and for most biomedical applications you're liable to experience in your tenure, you will not work extensively with inductors. However, you should be able to recall at least this much at a moment's notice to be able to ascertain a system's behavior.
\end{enumerate}




\section{Impedance}

\begin{enumerate}
	\item The measure of opposition a circuit element presents to a current when a potential is applied. (It is measured in ohms.)
	\item It is ``complex'' in two sense of the term. First, the actual phenomenon itself comprises complex numbers; that is, there is both a ``real'' and an ``imaginary'' component.
	\subitem \textbf{The real component} is known as resistance, $R$
	\subitem \textbf{The imaginary component} is known as reactance, $X$
	
	Impedance can be represented as a combination of either
	\subitem \textbf{Resistance and reactance}: $\mathbf{Z} = R + \jmath X$, where $\mathbf{Z}$ is impedance, $R$ is resistance, and $X$ is reactance, or
	\subitem \textbf{Magnitude and phase}: $\mathbf{Z} = |Z|e^{\jmath \theta}$, where $|Z|$ is the magnitude of the impedance vector, $\mathbf{Z}$, and $\theta$ is the phase of said vector (i.e., the delay between current and potential). Phase, $\theta$ is equivalent to $\tan^{-1}(X/R)$
	\item Impedance is also complex in the sense that it is complicated. The impedance of an object is a factor of many parameters including permittivity, geometry, quantum states, thermal stability, etc. Let us not view this sort of complexity as an impediment to our understanding of impedance.
	\item The inverse of impedance is \textbf{admittance}, $Y$, and comprises a real component, \textbf{conductance}, $G$, (which is the inverse of resistance) and an imaginary component, \textbf{susceptance}, $B$ (which is the inverse of reactance). (It is measured in Siemens.)
	\subitem $\mathbf{Y} = G + \jmath B$
\end{enumerate}

\subsection{A quick note on ``imaginary'' numbers}
The term ``imaginary'' is an unfortunate name for an excellent mathematical tool. All the imaginary operator -- in this class represented by $\jmath = \sqrt{-1}$ -- is is a type of number ``orthogonal'' to our ``real'' numbers. Imaginary numbers are no less ``real'' than real numbers. Unfortunately, they aren't necessarily the most intuitive to our little mammalian brains and thus we must be trained to work with them. However, as we will see in this class, they can be quite useful.

\section{Equivalent impedance}
\begin{enumerate}
	\item It will often be more convenient to think about the impedance which a component burdens a system with (or the conductance which it affords) rather than its resistance. \textbf{Therefore, we need to begin to think in terms of equivalent impedances as we start to evaluate circuits.}
	\item Recall Ohm's law 
	\subitem \textbf{Resistors}, $v = iR \qquad \qquad \rightarrow Z_{eq,R} = R$
	\subitem \textbf{Capacitors}, $v = \frac{1}{C}\int i dt \quad \rightarrow Z_{eq,C} = \frac{1}{\jmath \omega C}$
	\subitem \textbf{Inductors}, $v = L\frac{di}{dt} \quad \qquad \rightarrow\jmath \omega L$
	
	I want to plant a flag here for you to notice the relationship between the $\jmath \omega$ terms from the capacitor and inductor and the corresponding derivative and integral forms of current in the Ohm's law representation. This will become very important once we get into the Laplace and Fourier transforms.
	\item We must also recognize that few will be the circuits comprising but a single element. As such, we should know how to find the equivalent impedance of many elements.
\end{enumerate}

\subsection{Impedances in general}
\textbf{Series}
\begin{equation}
	\label{zeq,ser}
	Z_{eq,series} = Z_1 + Z_2 + Z_3 + ...
\end{equation}
\\
\textbf{Parallel}
\begin{equation}
	\label{zeq,par}
	\frac{1}{Z_{eq,parallel}} = \frac{1}{Z_1} + \frac{1}{Z_2} + \frac{1}{Z_3} + ...
\end{equation}
\\
\textbf{A special case to remember}. When dealing with only two elements:
\begin{equation}
	\label{zeq,two}
	Z_{eq} = \frac{Z_1\cdot Z_2}{Z_1 + Z_2}
\end{equation}

\subsection{Resistors}

\textbf{Series}
\begin{equation}
	\label{req,ser}
	R_{eq,series} = R_1 + R_2 + R_3 + ...
\end{equation}
\\
\textbf{Parallel}
\begin{equation}
	\label{req,par}
	\frac{1}{R_{eq,parallel}} = \frac{1}{R_1} + \frac{1}{R_2} + \frac{1}{R_3} + ...
\end{equation}


\subsection{Capacitors}
\textbf{Series}
\begin{equation}
	\frac{1}{C_{eq,series}} = \frac{1}{C_1} + \frac{1}{C_2} + \frac{1}{C_3} + ...
\end{equation}
\\
\textbf{Parallel}
\begin{equation}
	C_{eq,parallel} = C_1 + C_2 + C_3 + ...
\end{equation}

\subsection{Delta-Wye ($\Delta$-$Y$) transformations}
\textbf{Going from Delta to Wye}
\begin{align}
	Z_1 = \frac{Z_b Z_c}{Z_a+Z_b+Z_c} \\
	Z_2 = \frac{Z_a Z_c}{Z_a+Z_b+Z_c} \\
	Z_3 = \frac{Z_a Z_b}{Z_a+Z_b+Z_c}
\end{align}
\\
\textbf{Going from Wye to Delta}
\begin{align}
	Z_a = \frac{Z_1Z_2 + Z_2Z_3 + Z_1Z_3}{Z_1} \\
	Z_b = \frac{Z_1Z_2 + Z_2Z_3 + Z_1Z_3}{Z_2} \\
	Z_c = \frac{Z_1Z_2 + Z_2Z_3 + Z_1Z_3}{Z_3} \\
\end{align}

\subsection{A few examples}
\textbf{Example 1}
Find the equivalent resistance, if a resistor $R_1 = 10 \text{ k}\Omega$ is connected in parallel to $R_2 = 3.3 \text{ k}\Omega$.

\textit{Solution}. $R_{eq} = \frac{R_1 \cdot R_2}{R_1 + R_2} = \frac{(10)(3.3)}{10 + 3.3} = 2.48 \text{ k}\Omega$
\\
\\
\textbf{Example 2}
Find the equivalent resistance of three parallel-connected resistors of equal value. If $R = R_1 = R_2 = R_3 = 10 \text{ k}\Omega$, what's $R_{eq}$?

\textit{Solution}.
Recall, Equation \ref{req,par}
\begin{equation}
	\frac{1}{R_{eq}} = \frac{1}{R} + \frac{1}{R} + \frac{1}{R} \rightarrow 3R_{eq} = R \rightarrow R_{eq} = \frac{R}{3} \rightarrow R_{eq} = \frac{10k}{3} = 3.33 k\Omega
\end{equation}
\\
\\
\textbf{Example 3}
Four resistors are connected in parallel. $R_1 = 10 \text{ k}\Omega$, $R_2 = 1 \text{ k}\Omega$, $R_3 = 5 \text{ k}\Omega$, and $R_4 = 3 \text{ k}\Omega$. Calculate their equivalent resistance.
\begin{align}
	\frac{1}{R_{eq}} &= \frac{1}{R_1} + \frac{1}{R_2} + \frac{1}{R_3} + \frac{1}{R_4} \\
	\frac{1}{R_{eq}} &= \frac{1}{10k} + \frac{1}{1k} + \frac{1}{5k} + \frac{1}{3k} \\
	&= 612.3 \text{ }\Omega
\end{align}

\section{Grounds}
\begin{enumerate}
	\item A reference point in an electrical circuit from which potentials are measured
	\item A common return path within a circuit
\end{enumerate}


\section{Conductors}
\begin{enumerate}
	\item Allow from the transmission of electrical energy
	\item Serve to connect circuit elements
	\item Also known as wires and traces
	\item Within circuit schematics we must be mindful of ``junctions'' and ``jumps'' in conductors
\end{enumerate}


\section{Operational amplifiers ("Op-amps")}
\begin{enumerate}
	\item Active components that deliver the amplified difference between its inverting and non-inverting terminals
	\item Will be discussed at length in the next class and along with resistors, capacitors, and sources, will be among the primary circuit components we work with
	\item Allow us to model mathematical functions; any mathematical function that can be represented by a differential equation can be replicated with an op-amp
\end{enumerate}


\section{Diodes}
Two-terminal circuit elements that allow current to flow only in one direction

\section{Switches}
Make/break/change circuit paths (thereby diverting current or removing potential)

\begin{enumerate}
	\item Single pole, single throw, SPST
	\item Single pole, double throw, SPDT
	\item Double pole, single throw, DPST
	\item Double pole, double throw, DPDT
\end{enumerate}


\section{Transistors}


\section{Transformers}
\begin{enumerate}
	\item Transfer electrical energy between circuits using induction
	\item Allows for the effective transmission of power and the stepping up/down of potential
	\item Crucial for the transmission, distribution, and utilization of AC
\end{enumerate}

\newpage
\section{Worksheet}
\subsection{Problem 1, expressing power in ohms}
Utilizing Ohm's law, express units of power to include ohms.

\textbf{Solution}


\subsection{Problem 2, a couple toaster based problems}
A toaster draws 2 A at 120 V. What is its resistance?

\textbf{Solution}
\\
\\
How much current is drawn by a toaster with a resistance of 10 $\Omega$ at 110 V?

\textbf{Solution}

\subsection{Problem 3, currently conducting power}
In the circuit shown, calculate the current, $i$, the conductance, $G$, and the power, $p$.

\textbf{Solution}


\subsection{Problem 4, conductance of a sodium channel}
Conductance ($G$/$A$) of a sodium channel of a cell membrane at a specific time is 10 mS/cm$^{2}$. If the channel length as 100 nm, what is its conductivity?

\textbf{Solution}


\subsection{Problem 5, resistance of a simple tissue}
Determine the resistance of a homogenous and isotropic tissue with a cross-sectional area which can be described by the functions $y = 8 - x^2$ from $x = -2$ cm to $x = +2$ cm, a length of 10 cm (parallel to the z-axis), and a resistivity of 80 $\Omega$m.

\textbf{Solution}




\chapter{An introduction: III. Operational amplifiers}
01/17/2019 
\minitoc

\section{Some details}
\includegraphics[width=0.5\textwidth]{figures/op-amp1}
\\
\includegraphics[width=\textwidth]{figures/op-amp2}

\begin{enumerate}
	\item Behaves like a voltage-controlled voltage source
	\item They can amplify, sum, subtract, multiply, differentiate, integrate
	\item They are active circuit elements
	\item Though they have somewhat more complicated internal workings, we typically represent them in electrical circuits as a triangle with three (sometimes five) very important terminals:
	\subitem An inverting input (— sign, typically represented up top for convenience, but it need not be)
	\subitem A non-inverting input (+ sign, typically on bottom)
	\subitem An output
\end{enumerate}

\section{Some rules}
There are \textbf{three important features of ideal operational amplifiers} that we must understand thoroughly. These are things worth stamping in your brain.

\begin{enumerate}
	\item \textbf{Infinite open-loop gain.} The ``A'' of the gain is infinitely large such that any difference in voltages $V_1$ and $V_2$ causes an enormously large output voltage. As much as is being supplied. (The real value of gain in most operational amplifiers is between $10^{5}$ and $10^8$.)
	\item \textbf{Infinite input impedance.} Current cannot travel between the inverting and non-inverting terminals. (Really, the impedance is between $10^5$ and $10^13$ ohms and is often signal dependent.)
	\item \textbf{Zero output impedance.} There is no loss transmitting a voltage difference to the output. (Really is about 10-100 ohms and is chip dependent.)
\end{enumerate}

\section{Some conveniences}
\begin{enumerate}
	\item With infinite input impedance, no current can flow into or out of the terminals and hence $i_1$ and $i_2$ are equal to 0.
	\item Since no current flows across the terminals, the terminals are at equal potential. Hence ``$v_1 = v_2$''.
\end{enumerate}

Some extra facts with operational amplifiers are that they can be combined with a capacitors to create different filters. Adding a capacitor in series with the input resistor creates a "high-pass filter" amplifier, where it passes signals with frequency higher than a specific cutoff frequency and attenutates signals lower than the cutoff. Adding a capacitor in parallel with the feedback resistor creates a "low-pass filter" amplifier, where it passes signals with frequency lower than the cutoff and attenuates signals higher than it. The corner frequency for the cutoff may be caulcated with f=1/(2piRC). Having both capacitors would create a band-pass filter which attenuates signals lower than the lower cutoff and higher than the upper cutoff frequencies.

\newpage
\section{Some examples}

\subsection{Inverting amplifier}
We will apply the conservation of mass at this point to solve our equations.  This is among the simplest and most effective ways to add gain to a circuit. So much so that you will use it again and again and again in life and especially in labs

\includegraphics[width=0.5\textwidth]{figures/op-amp3}

We apply KCL at the node for v1
\begin{enumerate}
	\item $I1 – i2 – i3 = 0$
	\item $I3 = 0$
	\item $I1 – i2 = 0$
	\item $I1 = i2$
	\item $I1 = (vi – v1)/R1$
	\item $I2 = (v1 - Vo)/Rf$
	\item $(vi – v1)/R1 = (v1 - Vo)/Rf$
	\item $V1 = V2 = 0$
	\item $Vi/R1 = -Vo/Rf$
	\item $Vo = -Rf/R1 * Vi$
	\item $R2/R1$ is our gain, gain factor.
\end{enumerate}




\subsection{Non-inverting amplifier}
\includegraphics[width=0.5\textwidth]{figures/op-amp4}

Again, the name might imply what it does. It will amplify our input signal without inverting it.

We can again perform Nodal analysis.
\begin{enumerate}
	\item $i_1 – i_2 – i_3 = 0$
	\item $i_3 = 0$, since no current enters the non-inverting input
	\item $i_1 = i2$
	\item $(v_g – v_2)/R_1 = (v_2 – v_o)/R_f$
	\item $–v_2/R1 = (v_i – v_o)/R_f$
	\item $v_o = (1 + R_f/R_1) * V_i$
\end{enumerate}



\subsection{Voltage follower}
\includegraphics[width=0.5\textwidth]{figures/op-amp5.png}

What if we didn’t have any resistors? $\rightarrow$ Vi = v2 = v1 = Vo $\rightarrow Vi = Vo$


\subsection{Summing amplifier}
\subsection{Differential amplifier (as homework)}



\chapter{Circuit analysis: I. Nodal analysis}
01/22/2019
\minitoc
\newpage
\section{Nodes and branches}
\begin{enumerate}
	\item \textbf{A branch} is any two-terminal element. (examples: Resistor, Capacitor, Wire, etc.) 
	\subitem \textit{\textbf{A branch} is any two-terminal element. What are some two-terminal elements we’ve learned?}
	\item \textbf{A node} (junction) is a point of connection between two or more branches. 
	\subitem \textit{\textbf{A node} is a point of connection between two or more branches. Often indicated by a dot. What else have we called a node? A junction.}
	\item A loop is \textbf{independent} if at least one branch is not part of any other independent loop. 
	\subitem \textit{\textbf{A loop} is any closed path within a circuit. A closed path formed by starting at a node, passing through a set of nodes, returning to the starting node without passing through any node more than once.}
\end{enumerate}

\subsection{The Seven Bridges of Konigsberg}
The Königsberg bridge problem asks if the seven bridges of the city of Königsberg (left figure; Kraitchik 1942), formerly in Germany but now known as Kaliningrad and part of Russia, over the river Preger can all be traversed in a single trip without doubling back, with the additional requirement that the trip ends in the same place it began.

\begin{center}
	\includegraphics{figures/04.konisberg.png}
\end{center}



\subsection{Independence}
A loop is independent if it contains at least one branch which is not a part of any other independent loop
\begin{enumerate}
	\item Each of the loops in the circuit at right are independent 
	\item Independent loops lend themselves to sets of equations to be solved!
\end{enumerate}

\subsection{Fundamental theorem of network topology}
Fundamental theorem of network topology states that the number of branches, $b$, must equal the sum of the independent loops, $l$ and nodes, $m$ minus one, that is
\begin{equation}
	b = l + n - 1
\end{equation}

With this fundamental theorem we can also “redefine”/”refine” our definition of series and parallel.
\begin{itemize}
	\item \textbf{Series} | two or more elements share a single node and thereby carry the same current
	\item \textbf{Parallel} | connected to the same two nodes and thereby have the same voltage across them (potential difference)
\end{itemize}

\begin{center}
	\includegraphics{figures/04.example1.png}
\end{center}

 
\newpage
\section{Kirchhoff's Laws}
I am not personally a fan named laws of nature, especially ones which are mere recapitulations of already perfectly good laws. Thus do we introduce Kirchhoff’s laws, known as 
\begin{itemize}
	\item Kirchhoff’s current law (``KCL'') 
	\item Kirchhoff’s voltage law (``KVL'')  
\end{itemize}
	
These are, as far as I’m concerned, mere restatements of \textit{the conservation of mass} and \textit{the conservation of energy}, respectively.


\subsection{Kirchhoff's Current Law}
\textbf{Kirchhoff's current law} states that any node (junction) in an electrical circuit, the sum of currents flowing into that node is equal to the sum of currents flowing out of that node.

\begin{itemize}
	\item Put differently, the algebraic sum of currents entering a node (or any closed boundary) is zero
	\item For those mathematically inclined among us, that is: $\sum_{x=1}^n i_x = 0$
	\item Another way this often gets stated is by saying that the algebraic sum of charges within a system cannot change and is thus sometimes referred to as ``the conservation of charge''.
	\item Well since all of our charge carriers are merely particles (electrons, protons, ions, etc.), this is just another layer over the top of the underlying law which is that mass cannot be created or destroyed.
	\item However, Kirchhoff's formulation of this law (conserving charge, mass) is useful in circuits as it gives us a great tool, being able to say that current going in is equal to current going out
	\item For the figure $i_1 + (-i_2) + i_3 + i_4 + (-i_5) = 0 \rightarrow i_1 +i_3 + i_4 = i_2 + i_5$
	\item KCL forms the basis of a technique we’ll spend the next couple of lectures on known as “nodal analysis” because we evaluate the current going into and coming out of nodes
\end{itemize}

\includegraphics{figures/04.node.png}
\includegraphics{figures/04.single-loop-kvl.png}

\subsection{Kirchhoff's Voltage Law}
Kirchhoff's voltage law states that the sum of electrical potential differences (voltage) around any closed network is zero
\begin{itemize}
	\item That is, for any closed path (loop), the sum of voltages is zero.
	\item Mathematically: $\sum_{m=1}^{M} v_m = 0$, where M is the number of voltage drops (caused by circuit elements) in the loop and vm is the $m$th voltage drop
	\item The sum of voltage rises = the sum of voltage drops
	\subitem $v_1 + (-v_2) + (-v_3) + v_4 + (-v_5) = 0$
	\subitem $v_1 + v_4 = v_2 +v_3 + v_5$
	
\end{itemize}

\subsection{A few examples}
\begin{enumerate}
	\item Simple 1 Vs – 1 R circuit; Vs = 10, R = 1 kohm, I = 0.01 A / 10 mA
	\item Simple 1 Vs – 2 R in series; Vs = 10, R = 1 kohm, I = 0.005 A / 5 mA
	\subitem \textit{What does KCL tell us? (Current is the same through  resistors). }
	\item Simple 10 mA source, 2 R (1 kohm) in parallel; What is the voltage?
	\subitem $i_1 + (-i_2) + (-i_3) = 0 ] \rightarrow i_1 = i_2 + i_3 \rightarrow i_1 = V/R_1 + V/R_2$ 
	\subitem $0.01 = V/1000 + V/1000 \rightarrow 0.01 = 2V/1000 \rightarrow 0.010/2*1000 = 5$ V
	\subitem \textit{Nodes are at the same voltage!}
	\item 10 mA up, 5 mA down, 2 R (1 kohm), Rl (500 ohm), all in parallel. What is the current through load Rl? [have someone come to the board and solve]
	\subitem When current sources are in parallel they add together 
	\item For the circuit shown below, use KCL to find the remaining branch currents
\end{enumerate}

\begin{center}
	\includegraphics{figures/04.example2.png}
\\
\includegraphics{figures/04.example3.png}

\end{center}

For Figure 2.13, you know to use KCL because the circuit contains two current sources. Ideal current sources will deliver whatever potential is necessary to obtain the desired current so we don't know how potential behaves through it.
If we look at node x and arbitrarily set $i_1$ as entering the node from the left, $i_2$ as exiting the node down the branch containing the resistor, and $i_3$ as entering the node from the right, then:
\subitem $i_1 - i_2 + i_3 = 0$
\subitem $2i_0 - i_0 + 5 = 0$
\subitem $i_0 = - 5 A $
The negative sign just means that the arbitrary direction we chose for our analysis is opposite of the actual direction of current flow

For Figure 2.77, start at a node with only one unknown current such as the upper right node.
\subitem $2 - i_4 - 4 = 0$
\subitem $i_4 = -2 A$
Repeat the process for remaining nodes
\subitem $7 + i_4 - i_3 = 0$
\subitem $i_3 = 7 + (-2) = 5 A$

\subitem $-i_2 - 3 - 7 = 0$
\subitem $i_2 = -10 A$

\subitem $i_1 + i_2 - 2 = 0$
\subitem $i_1 = 2 - (-10) = 12 A$

We could check our work by performing nodal analysis at the last node. However, it is not a necessary step as we have already found the desired unknowns.



\newpage
\section{Nodal analysis}
Nodal analysis | a general circuit analysis technique in which we try to determine the potential difference between nodes by applying KCL and KVL (in my experience, usually focusing a bit more on KCL)

Your textbook offers the following description of the technique which seems pretty good to me:
\begin{enumerate}
	\item Select a node as a reference. Assign voltages ($v_1, v_2, ..., v_{n-1}$) for the remaining n-1 nodes, all of which will be referenced with respect to the reference node.
	\item Apply KCL to each of the n-1 nonreference nodes. Use Ohm’s law to express the branch currents in terms of node voltages.
	\item Solve the resulting simultaneous equations (system of equations) for each unknown node voltage.
	\item It's as easy as that! But, well, actually, it can get a little hairy once you start to apply it in earnest.
\end{enumerate}


\subsection{The procedure}

\begin{center}
	\includegraphics{figures/04.typical-nodal-circuit.png}
\end{center}

\begin{enumerate}
	\item Begin by putting a reference, usually a ``ground''
	\subitem This ground can be one of two sorts: (1) \textbf{Earth ground} in which ultimately the whole earth is used as the reference point; or \textbf{Chassis ground} in which the case of the device in which the circuit is in will act as a reference as it will presumably be sufficiently large as to serve fine [this is also partly the reason why you can ``feel'' a MacBook charge up – its charger does not utilize a traditional ``earth ground''
	\subitem Either will suffice for our purposes here
	\item Next we label all the nodes.
	\subitem How many branches, nodes, and loops?
	\subitem 5 branches, 3 nodes, 3 loops. Satisfies our network condition. 
	\subitem You can give them any label you want, but I find working your way up from the ground in a clockwise manner and numbering them sequentially is a good habit to get into.
	\subitem Keep in mind that we typically set our reference node to have a voltage of 0. We can actually set it to be anything we’d like, but the math is often easier if we just make it 0.
	\item Then we apply KCL to each nonreference node in the circuit.
	\subitem At node 1, $I_A - i_1 - i_2 - I_B \rightarrow I_A = I_B + i_1 + i_2$
	\subitem At node 2, $I_B + i_2 - i_3 \rightarrow I_B + i2 = i3$
	\subitem Once we’ve got that, now it’s a matter of applying Ohm’s law. Thought typically written as $V = iR$, it is perhaps more helpful to write its full extension here and note that $(V_a – V_b) = iR
$	\subsubitem $I = (V_a - V_b)/R$
	\subitem Thus we can state
	\subsubitem $I1 = (v_1 - v_0) /R_1 \rightarrow I_1 = G_1 (v_1 - v_0)$
	\subsubitem $I2 = (v_1 - v_2) /R_2 \rightarrow I_2 = G_2(v_1 - v_2)$

\end{enumerate}





\section{Solving simultaneous equations}
\subsection{Cramer's Rule}
When given a system of linear equations, Cramer's Rule allows us to solve directly for the specific variable whose value we are looking for.

To use Cramer's Rule, the system of equations must satisfy two conditions:
\begin{enumerate}
	\item There must be the same number of equations as variables (the coefficient matrix must be a square).
	\item The determinant of the coefficient must be non-zero.
\end{enumerate}

Use the following steps to apply Cramer's Rule:
\begin{enumerate}
	\item Write the coefficient matrix of the system (call this matrix A). Make sure this is a square matrix; otherwise Cramer's Rule is not applicable.
	\item Compute the determinant of matrix A. Make sure that this value is non-zero; otherwise Cramer's Rule is not applicable here.
	\item Suppose the first variable of the system is x. Write the matrix Ax by placing the column of numbers to the right of the equals sign as the first column of Ax and using the non-x coefficients of matrix A as the remaining columns.
	\item The value of x is the determinant of Ax divided by the determinant of A.
	\item Repeat steps 3 and 4 to solve for any other variable as needed.
\end{enumerate}

Example implementing Cramer's Rule:
We will use the following system of equations to demonstrate how to use Cramer's Rule to solve for the value of x:

\begin{center}
x + y + z = 4;  -x + 2y = 1;  -y + z = 1
\end{center}

\begin{enumerate}
	\item Write the coefficient matrix A and check that it is a square matrix:
		\begin{center}
		\[
  		A=
  		\left[ {\begin{array}{ccc}
   		1 & 1 & 1\\
   		-1 & 2 & 0\\
   		0 & -1 & 1\\
  		\end{array} } \right]
		\]
		\end{center}
	\item Solve for the determinant of A. ($|A| = 4$). The determinant does not equal zero.
	\item Write the matrix Ax:
		\begin{center}
		\[
  		Ax=
  		\left[ {\begin{array}{ccc}
   		4 & 1 & 1\\
   		1 & 2 & 0\\
   		1 & -1 & 1\\
  		\end{array} } \right]
		\]
		\end{center}
	\item Solve for the determinant of Ax. ($|Ax| = 4$). 
	\item Solve for the value of x:
		\begin{center}
		x = $\frac{|Ax|}{|A|}$
		x = 1
		\end{center}
\end{enumerate}

Reference: Explanation and example inspired by "Solving System of Linear Equations: (lesson 4 of 5), Cramers Rule", MathPortal: https://www.mathportal.org/algebra/solving-system-of-linear-equations/cramers-rule.php. [Accessed 23 January, 2019].


\newpage

\section{Worksheet}
\subsection{Problem 1, KCL at a few nodes}
Use KCL to write equations at each node.
\begin{center}
	\includegraphics[width=0.5\textwidth]{figures/04.problem1.png}
\end{center}


\textbf{Solution}


\subsection{Problem 2, matrix notation}
Write the matrix form of the equations written above. 


\textbf{Solution}


\subsection{Problem 3, Cramer's rule}
Using Cramer’s rule on the matrix equations above, what are the results?


\textbf{Solution}


\subsection{Problem 4, Straight to the matrix}
Write the node-voltage equations to the circuit at right in the matrix form.
\begin{center}
	\includegraphics[width=0.5\textwidth]{figures/04.problem4.png}
\end{center}


\textbf{Solution}









\chapter{Circuit analysis: II. Mesh analysis}
01/24/2019
\minitoc
\newpage
\section{Clarifications on the homework}

\section{Mesh analysis}
\begin{itemize}
	\item A mesh is a loop which does not contain any other loops within it.
	\item ABEF and BCDE are meshes. ABCDEF is not.
	\item We apply KVL to find the mesh currents within a given circuit (this can be particularly helpful / convenient when dealing with parallel circuits).
\end{itemize}
\begin{center}
	\includegraphics{figures/05.mesh1.png}
\end{center}

\section{Steps of mesh analysis}
\begin{enumerate}
	\item \textbf{For each of your $n$ meshes assign a mesh current ($i_1$, $i_2$, ..., $i_n$).} You may draw them in any direction you want, but generally a nice clockwise direction will help with consistency.
	\item \textbf{Apply KVL to each of the n meshes.} Use Ohm’s law to express the voltages in terms of the mesh currents
	\item \textbf{Solve the resulting n simultaneous equations to get the mesh currents.}
\end{enumerate}

\subsection{Important condition for use}
Unlike nodal analysis, mesh analysis only works for ``planar” networks'' | that is one which can be written in two-dimensions without any branches.\footnote{Nonplanar networks can still be solved using nodal analysis.}

\section{An example}
\includegraphics[width=\textwidth]{figures/05.mesh2.png}

For the example seen above, we might at this point in our lives be tempted to solve the circuit with branch currents (the current that actually flows into and out of every branch. That is certainly a valid way by which to look at the problem. But now we're going to try another. We're going to try to understand ``mesh currents''.

Those branch currents can be formed into a pair of coupled equations
\begin{eqnarray}
	V_A - R_1 i_1 - R_3 i_3 = 0 \\
	V_B - R_2 i_2 - R_3 i_3 = 0 
\end{eqnarray}

We can also apply KCL at the node between the three resistors and recognize a third bounding equation: $i_1 + i_2 - i_3 = 0$. We can then solve these three equations through any number of means (substitution, elimination, etc.).

But mesh analysis allows us to model the situation more simply and often more efficiently.

We begin by defining two arbitrary mesh currents, $i_1$ and $i_2$. Then we apply KVL as we do a ``walk'' around the loop defined by the mesh current.
\begin{eqnarray}
	V_A - R_1i_1 - R_3(i_1 - i_2) = 0 \\
	-R_3(-i_1 +i_2) - R_2i_2 - V_B = 0
\end{eqnarray}
This given us two equations with two unknowns (the arbitrary mesh currents, $i_1$ and $i_2$, we defined).

\section{Another example}
\begin{center}
	\includegraphics{figures/05.mesh3.png}
\end{center}

For the circuit above, let's try to find the branch currents by performing mesh analysis (i.e., by. first finding the mesh currents).

\textbf{Solution.}

Obtain mesh current using KVL. For mesh 1:
\begin{equation}
	-15 + 5 i_1 + 10(i_1 - i_2) + 10 = 0
\end{equation}
which simplifies to 
\begin{equation}
	3i_1 - 2i_2 = 1
	\label{mesh1}
\end{equation}

For  mesh 2:
\begin{equation}
	6i_2 + 4 i_2 + 10(i_2 - i_1) - 10 = 0
\end{equation}
simplifying to
\begin{equation}
	i_1 = 2i_2 - 1
	\label{mesh2}
\end{equation}

\subsection{Solving it one way}
Perhaps the technique we'd most likely reach for if we knew nothing else about a situation is substitution. In this case, let's substitute equation \ref{mesh2} into equation \ref{mesh1}:
\begin{equation}
	6i_2 - 3 - 2i_2 = 1 \rightarrow i_2 = 1 \text{ A}
\end{equation}
And plugging that result back in
\begin{equation}
	i_1 = 2i_2 -1 = 2 - 1 = 1 \text{ A}
\end{equation}

And thus $I_3$ being equal to the difference in current between $i_1$ and $i_2$ is equal t zero.


\subsection{Solving it another}
Another way of doing this is by recognizing some of the techniques afforded to us by linear algebra. If we squint at the equations with a little bit of good brain wrinkling, we can see that 
\begin{equation}
	\begin{bmatrix}
		3 & -2 \\ -1 & 2
	\end{bmatrix}
	\begin{bmatrix}
		i_1 \\ i_2
	\end{bmatrix}
	=
	\begin{bmatrix}
		1 \\ 1
	\end{bmatrix}
\end{equation}
is just as valid of way of putting it as any other.

In this form we can use a few techniques to solve for the unknowns. One technique to be aware of given its deployability in those times in which you might need to write out your work is \textit{Cramer's rule}.

Cramer's rule says that 
\begin{eqnarray}
	i_1 = \frac{\Delta_1}{\Delta} \\
	i_2 = \frac{\Delta_2}{\Delta}
\end{eqnarray}
where
\begin{eqnarray}
	\Delta = 
	\begin{vmatrix}
		3 & -2 \\ -1 & 2
	\end{vmatrix}
	= (3)(2) - (-1)(-2) = 4 \\
	\Delta_1 = 
	\begin{vmatrix}
		1 & -2 \\ 1 & 2
	\end{vmatrix}
	= (1)(2) - (1)(-2) = 4 \\
	\Delta_2 = 
	\begin{vmatrix}
		3 & 1 \\ -1 & 1
	\end{vmatrix}
	= (3)(1) - (-1)(1) = 4 \\
\end{eqnarray}
Thus
\begin{eqnarray}
	i_1 = \frac{4}{4} = 1 \\
	i_2 = \frac{4}{4} = 1
\end{eqnarray}
which agrees with the results we obtained before.



\section{Yet another example}
\begin{center}
	\includegraphics{figures/05.mesh5.png}
\end{center}
Let's begin by defining the mesh currents (flowing clockwise around each mesh)

\begin{center}
	\textbf{Mesh 1}
\end{center}
\begin{equation}
	20(i_1 - i_3) + 10(i_1 - i_2) - 70 = 0
\end{equation}

\begin{center}
	\textbf{Mesh 2}
\end{center}
\begin{equation}
	10(i_2 - i1) + 12(i_2 - i_3) + 42 = 0
\end{equation}

\begin{center}
	\textbf{Mesh 3}
\end{center}
\begin{equation}
	20(i_3 - i_1) + 14i_3 + 12(i_3 - i_2 = 0
\end{equation}

Putting the equations into standard form:
\begin{eqnarray}
	30i_1 - 10i_2 - 20i_3 = 70 \\
	-10 i_2 +22 i_2 - 12 i_3 = -42 \\
	-20 i_1 - 12i_2 +46 i_3 = 0
\end{eqnarray}

In matrix notation that becomes:
\begin{equation}
	\begin{bmatrix}
		30 & -10 & -20 \\ -10 & 22 & -12 \\ -20 & -12 & 46
	\end{bmatrix}
	\begin{bmatrix}
		i_1 \\ i_2 \\ i_3
	\end{bmatrix}
	=
	\begin{bmatrix}
		70 \\ -42 \\ 0
	\end{bmatrix}
\end{equation}
While this can be solved any number of ways, in MATLAB one can simply write \texttt{I = R/V}, which should tell you something about what is ``going on'' mathematically in these operations.


\section{Writing mesh equations directly in matrix form}
Let's take a look at yet one more example.
\begin{center}
	\includegraphics{figures/05.mesh6.png}
\end{center}

We might, by now, begin to appreciate that one of our matrices is simply the resistances (impedances) of a mesh bundled up in one matrix, the currents bundled up in another, and the potentials bundled up in another, shown below.

\begin{equation}
	\begin{bmatrix}
		r_{11} & r_{12} & r_{13} \\
		r_{21} & r_{22} & r_{23} \\
		r_{31} & r_{32} & r_{33} \\
	\end{bmatrix}
	\begin{bmatrix}
		i_1 \\ i_2 \\ i_3
	\end{bmatrix}
	=
	\begin{bmatrix}
		v_1 \\ v_2 \\ v_3
	\end{bmatrix}
\end{equation}

To properly assemble these matrices there's just a few simple rules.


First, let's take a loop around each mesh and determine the equivalent impedance (of that mesh!)
\begin{itemize}
	\item \textbf{Mesh 1.} $R_2 + R_5 + R_6$
	\item \textbf{Mesh 2.} $R_2 + R_1 + R_3$
	\item \textbf{Mesh 3.} $R_3 + R_6 + R_5$
\end{itemize}

These will comprise values along the diagonal of the resistance (impedance) matrix ($r_{11}$. $r_{22}$, and $r_{33}$).

Next, let's look for elements that are shared among meshes
\begin{itemize}
	\item \textbf{Mesh 1 \& 2} share $R_2$, so $r_{21}$ and $r_{12}$ both become $-R_2$
	\item \textbf{Mesh 2 \& 3} share $R_3$, so $r_{23}$ and $r_{32}$ both become $-R_3$
	\item \textbf{Mesh 3 \& 1} share $R_5$, so $r_{13}$ and $r_{31}$ both become $-R_5$
\end{itemize}

Thus, our entire resistance matrix becomes

\begin{equation}
	\begin{bmatrix}
		r_{11} & r_{12} & r_{13} \\
		r_{21} & r_{22} & r_{23} \\
		r_{31} & r_{32} & r_{33} \\
	\end{bmatrix}
	=
	\begin{bmatrix}
		(R_2 + R_5 + R_6) & -R_2 & -R_5 \\
		-R_2 & (R_2 + R_1 + R_3) & -R_3 \\
		-R_5 & -R_3 & (R_3 + R_6 + R_5) \\
	\end{bmatrix}
\end{equation}

The matrix it buts up against is easy enough to construct, it is simply the mesh current of each mesh

\begin{equation}
	\begin{bmatrix}
		i_1 \\ i_2 \\ i_3
	\end{bmatrix}
	=
	\begin{bmatrix}
		i_1 \\ i_2 \\ i_3
	\end{bmatrix}
\end{equation}
	
And the potential matrix is the sum of sources in the direction of the mesh current

\begin{equation}
		\begin{bmatrix}
		v_1 \\ v_2 \\ v_3
	\end{bmatrix}
	=
	\begin{bmatrix}
		-v_A + v_B \\ v_A \\ -v_B
	\end{bmatrix}
\end{equation}

Giving rise to a final form which can be solved efficiently by any computation engine we put before us.
\begin{equation}
	\begin{bmatrix}
		(R_2 + R_5 + R_6) & -R_2 & -R_5 \\
		-R_2 & (R_2 + R_1 + R_3) & -R_3 \\
		-R_5 & -R_3 & (R_3 + R_6 + R_5) \\
	\end{bmatrix}
	\begin{bmatrix}
		i_1 \\ i_2 \\ i_3
	\end{bmatrix}
	=
	\begin{bmatrix}
		-v_A + v_B \\ v_A \\ -v_B
	\end{bmatrix}
\end{equation}



\chapter{Circuit analysis: III. Supernodes and supermeshes}
01/29/2019 – Lecture 6. 
\minitoc
\newpage
\section{A review of nodal and mesh analysis}
\subsection{Nodal analysis}
\begin{enumerate}
	\item Select a node as a reference. Assign voltages ($v_1$, $v_2$, ..., $v_{n-1}$) for the remaining n-1 nodes, all of which will be referenced with respect to the reference node.
	\item Apply KCL to each of the n-1 nonreference nodes. Use Ohm’s law to express the branch currents in terms of node voltages.
	\item Solve the resulting simultaneous equations (system of equations) for each unknown node voltage.
	\item (It’s as easy as that! But, well, actually, it can get a little hairy once you start to apply it in earnest.)
\end{enumerate}

\newpage
\section{Nodal analysis with an independent current source}
Let's begin by analyzing the following circuit.
\begin{center}
	\includegraphics{figures/06.example1.png}
\end{center}
\begin{enumerate}
	\item Begin by putting a reference, usually a ``ground''.
	\item Next we label all the nodes.
	\subitem How many branches, nodes, and loops? (5 branches, 3 nodes, 3 loops. Satisfies our network condition.)
	\subitem You can give them any label you want, but I find working your way up from the ground in a clockwise manner and numbering them sequentially is a good habit to get into.
	\subitem Keep in mind that we typically set our reference node to have a voltage of 0. We can actually set it to be anything we’d like, but the math is often easier if we just make it 0. 
	\item Then we apply KCL to each nonreference node in the circuit.
	\subitem At node 1, $I_A - i_1 - i_2 - I_B \rightarrow I_A = I_B + i_1 + i_2$
	\subitem At node 2, $I_B + i_2 - i_3 - I_B + i_2 = i_3$
	\subitem Once we’ve got that, now it’s a matter of applying Ohm’s law. Thought typically written as $V = iR$, it is perhaps more helpful to write its full extension here and note that $(V_a – V_b ) = iR$
	\item $I = (V_a - V_b ) /R$, Thus we can state
	\begin{itemize} 
		\item $I_1 = (v_1 - v_0 ) /R1 \rightarrow I_1 = G_1(v_1 - v_0 )$
		\item $I_2 = (v_1 - v_2 ) /R2 \rightarrow I_2 = G+2(v_1 - v_2 )$
		\item $I_3 = (v_2 - v_0 ) /R3 \rightarrow I_3 = G_3(v_2 - v_0 )$
	\end{itemize}
	\item Now we can substitute these relationships into our previous equations
\end{enumerate}
\newpage

\section{A brief review of Cramer's rule}
\includegraphics[width =\textwidth]{figures/06.cramers.png}
\newpage

\section{Nodal analysis with voltage sources, \textbf{Supernodes}}
We often pick the reference node at one end of the source (and then we have one less unknown node voltage to solve.

Typical examples can be typically solved. Write current equations at nodes 1 and two.

\includegraphics{figures/06.example2.png}

\includegraphics{figures/06.example3.png}
If we try to write a current equation at node 1, we must include a term for current through the 10V source. We could assign an unknown, or...

Or we can create a ``supernode'' which we do by drawing a dashed line around several nodes, including the elements between them, and apply KCL more broadly

\begin{itemize}
	\item Recall that KCL says that the net current flowing through any closed surface must be equal to zero. Thus, we apply KCL to the supernode.
\end{itemize}
\begin{equation}
	\frac{v_1}{R_2} + \frac{R_1}{v_1 - (-15)} + \frac{v_2}{R_4} + \frac{v_2 - (-15)}{R_3} = 0
\end{equation}

Now you might be tempeted to write another current equation for the other super node, however, you’d find that the equation is equivalent to the one we just found. If we tried to solve by substitution, all the terms would drop out and we’d see that the matrix was singular (the determinant is 0, it won't tell you anything new. 

Instead we apply KVL, noting that $-v_1 - 10 + v_2 = 0$

Next, we find an expression for the controlling variable ix in terms of the node voltages. Notice that since in this case ix is the current flowing away from node 3 through R3, we can say $i_x = (v_3 – v_2) / R_3$.


\newpage

\section{Nodal analysis with controlled sources}
We approach it the same way, but we're mindful of the dependence
\begin{center}
	\includegraphics{figures/06.example4.png}
\end{center}


Write KCL equations at each node, including the current of the controlled source, just as if it were an ordinary current source

\begin{center}
	\includegraphics{figures/06.eqn1.png}
\end{center}

Next, we find an expression for the controlling variable ix in terms of the node voltages. Notice that since in this case ix is the current flowing away from node 3 through R3, we can say $i_x = (v_3 - v_2) / R_3$.
\begin{center}
	\includegraphics{figures/06.eqn2.png}
\end{center}

So long as only three of those variable are unknown (say the voltages), then it can be solved by whatever method you’d like.

\section{Mesh analysis with current sources}
\section{Mesh analysis with controlled sources, \textbf{Supermeshes}}



\chapter{Circuit analysis: IV. Circuit theorems}
02/05/2019 – Lecture 7. 
\minitoc
\newpage
\section{Circuit theorems}
Circuit theorems are ways of quickly describing, summarizing, analyzing circuit, based on some mathematical tricks, two of which we’ll review today: Thevenin and Norton equivalent circuits. 
\begin{itemize}
	\item \textbf{Thevenin} | taking a complex circuit seen between two terminals and representing it as a resistor in series with a voltage source
	\item \textbf{Norton} | taking a complex bit of circuitry between two terminals and representing it as a current source in parallel with a current source
\end{itemize}

\begin{enumerate}
	\item These only apply to ``linear circuits'' – that is, circuits made of (ideal) resistors, capacitors, inductors, op-amps, filters
	\item Non-linear elements include things like diodes, transistors, and some of the digital logic stuff we’ll get into later in the semester.
	\item It might pay us dividends for us to consider what we mean by linearity at this point in the semester as it will become very relevant in our up-coming analyses
\end{enumerate}

\section{Linearity}
Comprises two separate yet equally important properties

\begin{enumerate}
	\item Homogeneity | If the input (the excitation) is multiplied by a constant, the output (the response) is multiplied by the same constant
	\subitem $v = iR \rightarrow kv = kiR$
	\item Additivity | The response to a sum of excitations is equal to the responses to each individual excitation
	\subitem  $v_1 = i_1R; v_2 = i_2R; \rightarrow v = (i_1 + i_2)R = i_1R_1 +i_2R = v_1 + v_2$
\end{enumerate}

Much of what we will do in this class is linear and much in life, given sufficient approximating, can be considered linear. Such relationships are useful, if not always strictly true (recall that even wires have resistivities and resistors have frequency ranges)

Note that since power is $i^2R = v^2/R$, the relationship is a quadratic and thus our necessary assumptions of linearity are not applicable.


\section{Superposition}
\begin{itemize}
	\item Until now, we’ve been considering everything in a circuit always ``on'', but one of the things the linearity of the system allows us to do is to selectively turn on and off sources while we solve, so long as we sum them up in the end.
	\item This principle | \textbf{superposition} | states that the voltage (or current) through an element (in a linear circuit) is the algebraic sum of the voltages across (or current through) that element due to each independent source acting alone. 
	\item To apply the superposition principle
	\begin{enumerate}
		\item Turn off all independent source but one and find the output (voltage or current) due to that source [using any of the techniques you’ve previously learned]
		\item Do this for each independent source
		\item Find the total contribution [for any given element, or indeed for all of them] by adding all the contributions due to the independent sources
		\item (Leave in all dependent sources since they are controlled by circuit variables)
	\end{enumerate}
	\item To “turn off” a source:
	\subitem Voltage sources get replaced by a 0V source / a short circuit | \textbf{Turn voltage sources into wires}
	\subitem Current sources get replaced by a 0A source / an open circuit | \textbf{Cut current sources out}
\end{itemize}

\includegraphics[width=\textwidth]{figures/07.01.png} \\
\includegraphics[width=0.5\textwidth]{figures/07.02.png}


\section{Source transformation}
\textbf{Source transformation} is the process of replacing a voltage source $v_s$ in series with a resistor $R$ by a current source $i_s$ in parallel with a resistor $R$ or vice versa.
\includegraphics{figures/07.03.png}

\begin{itemize}
	\item We can prove these two are equivalent between a-b by short circuiting the two-terminals.
	\item If the sources are turned off, the resistance at terminal a-b are the same (R)
	\item If the terminals are shorted, current flowing from a to be is $i_{ab}$ = $Vs/R$ and $i_{ab}$ = $i_s$. Thus $Vs/R = i_s$
	\item You can also replace dependent sources this way, provided you’re careful, but we’ll skip that analysis here and I won’t encourage its use unless you personally feel comfortable with the material
\end{itemize}


\section{Thevenin equivalents}
\begin{itemize}
	\item A linear two-terminal circuit can be replaced by an equivalent circuit consisting of a voltage source, $V_{th}$, in series with a resistor, $R_{th}$
	\subitem $V_{th}$ is equal to open-circuit voltage of original network, $V_{th} = V_{oc}$
	\item If we were to short the connection across terminals a and b, we can see (as we did previously), that $i_{sc} = V_{th}/R_{th}$
	\subitem Thus the short-circuit current is equal to the current for the original circuit and the Thevenin equivalent.
	\item These two facts allow us to say something rather interesting, namely that the \textbf{Thevenin resistance} of a circuit is equal to the ratio of its open-circuit voltage and its short-circuit current. $R_{th} = V_{oc}/I_{sc}$
	\item Thus, \textbf{to determine the Thevenin equivalent circuit}, we start by analyzing the original network for its open-circuit voltage and its short circuit current. [A more robust derivation of this is shown in your book in chapter 4.7]
	\item \textbf{To find the thevenin equivalent resistance}:
	\subitem \textit{If there are \textbf{no} dependent sources}, turn off all independent sources in the network. $R_{th}$ is the input resistance to the network between the two terminals of interest
	\subitem \textit{If \textbf{there are} dependent sources}, still turn off all independent sources and apply superposition.
	\subsubitem Apply a voltage at the terminals and determine the resulting current. Then $R_{th}$ is the ratio of applied voltage and elicited current $v_o/i_o$
	\subsubitem Or apply a current and determine the resulting voltage. Use which ever you feel comfortable with.
\end{itemize}
\newpage

\subsection{An example}
Create thevenin equivalent circuit between a and b [$R_L$ is a potentiometer and thus its value can change. We want to simplify our understanding of how all the complex stuff before it will act.]

\includegraphics{figures/07.04.png}

\begin{enumerate}
	\item Start by turning off the independent sources (32 V and 2 A), replacing them with a short- and an open-circuit respectively
	\item This will yield our resistance as a $4 || 12 + 1 = (4*12)/16 + 1 = 4$ ohms
	\item Next, identify what our $V_{Th}$ would be
	\item To find $V_{Th}$ we can apply mesh analysis
	\begin{eqnarray}
		32 – 4i_1 - 12i_1 + 12i_2 = 0 \\ 
		i_2 = 2 \\ 
		i_1 = 0.5 A \\ 
	\end{eqnarray} 
	then plug it in for the voltage across the 12 ohm 
	\begin{equation}
		V_{th} = 12(i_1-i_1) = 30
	\end{equation}
	\item Or could use nodal analysis
	\begin{equation}
		(32 – V_{th})/4 + 2 = V_{Th}/12 \rightarrow V_{Th} = 30 
	\end{equation} 
\end{enumerate}

\newpage

\section{Norton equivalents}
\begin{itemize}
	\item A linear two-terminal circuit can be replaced by an equivalent circuit comprising a current source, $I_N$ in parallel with a resistor $R_N$
	\subitem $I_N$ is equal to the short-circuit current through the terminals
	\item To find the Norton equivalent resistance, we start the same way we did for $R_Th$ | that is, short our voltage sources, open our current sources and find the resistance.
	\subitem Well then this should suggest to us that $R_N$ is equal to $R_Th$, and indeed they are equal.
	\item The Norton current, $I_N$, is equal to the short-circuit current. 
	\subitem So we merely find the current that would travel through the short 
\end{itemize}

\subsection{An example}
\includegraphics{figures/07.05.png}
\begin{enumerate}
	\item We begin by zeroing all of our independent sources 
	\subitem $R_N = 5 || (8 +4 +8 [+0]) = 5||20 = (5*20)/(5+20) = 4$ ohms
	\item We can apply mesh analysis to find $I_N$
	\subitem We can ignore the 5-ohm resistor as its in parallel with a short and current will always seek the path of least resistance.
	\subsubitem Recall our bioimpedance example where with increasing frequency caused the current to flow down a different branch, because with greater frequency the capacitor went from acting like an open circuit (at low frequencies) to a short circuit (at high frequencies), thus bypassing the other branch altogether.
	\begin{eqnarray}
		I_1 = 2 \\
		I_2 = 12 – 4i_2 +4i_1 – 8i_2 – 8 i_2 \rightarrow 4(2) – 20(i_2) = -12 \\ 
		I_2 = 1 = i_{sc} = I_N
	\end{eqnarray}
	\item Alternatively, we could have found the Thevinin voltage [meshing it up]
	\begin{eqnarray}
		i_3 = 2 \\
		12 -4(i_4) + 4(i_1) – 8(i_4) -5(i_4) – 8(i_4) \rightarrow 4(2) – 25(i_4) = -12 \\
		I_4 = 0.8 A
	\end{eqnarray}
	\subitem Since we know 0.8 A goes through the 5 ohm resistor, and since we know that that 5 ohm resistor experiences the same voltage drop as the open circuit, we simple multiple 5*0.8 to give us, 4 V
	\item Well since we know $R_N = R_{Th}$ is 4 and $V_{Th}$ is 4: $I_N = V_{Th}/R_{Th} = 1 A$
	\item Thus Norton and Thevenin circuits are two sides of the same “source transformation” coin
\end{enumerate}

\newpage

\section{Equivalents with dependents}
When we’ve got a dependent source, we can’t just remove the sources and combine impedances. Instead, we’ve got to analyze the circuit to find the open-circuit voltage and the short-circuit current (and indeed such an approach will work in all cases, it’s just a little bit more work on our parts)

\includegraphics[width=0.9\textwidth]{figures/07.06.png}

\begin{itemize}
	\item I’m personally a fan of finding the Thevenin open voltage because it just clicks with my brain a little better, but if you like the Norton short-circuit current, feel to take that approach
	\item So I apply nodal analysis
	\begin{eqnarray}
		I_x + 2i_x – V_{oc}/10 = 0 \rightarrow 3i_x = V_{oc}/10 \\
		\text{Next we write an expression for our controlling variable, ix} \\
		10 – V_{oc} = i_x(5) \rightarrow ix = (10 – V_{oc})/5 \\
		\text{We can substitute this into our previous equation} \\
		3*(10-V_{oc})/5 = Voc/10 \rightarrow \mathbf{V_{oc} = V_{Th} = 8.57 }
	\end{eqnarray}
	\item Now we can consider the short-circuit conditions.
	\begin{eqnarray}
		I_x + 2i_x i_{sc} = 0 \rightarrow 3i_x = i_{sc} \\
		I_x = 10/5 = 2 \rightarrow \mathbf{I_{sc} = 3i_x = I_N = 6} 
	\end{eqnarray}
	\item From this, we can take the ratio of the open circuit voltage and the short circuit current and find the equivalent resistance of the network via Ohm's law: 8.57 V / 6 A = 1.43 ohms
\end{itemize}

\section{A step-by-step procedure}
\begin{enumerate}
	\item Perform two of these:
	\subitem Determine the open-circuit voltage $V_t = v_{oc}$
	\subitem Determine the short-circuit current $I_n = i_{sc}$
	\subitem Zero the independent sources and find the Thevenin resistance $R_t$ looking back into the terminals. Do not zero dependent sources
	\item Use the equations $V_t = R_tIn$ to compute the remaining value
	\item The Thevenin equivalent consists of a voltage source $V_t$ in series with $R_t$
	\item The Norton equivalent consists of a current source $I_n$ in parallel with $R_t$
\end{enumerate}












\part{Systems}



\chapter{The Laplace Transform: I. What it is and why it is important}
02/14/2019 – Lecture 9. 
\minitoc
\newpage
\section{Euler's formula, Euler's identity }
You must all know, from this moment on, that Euler's formula is thus
\begin{equation}
	e^{\jmath x} = \cos x + \jmath \sin x
\end{equation}
where $e$ is Euler's number, $\jmath$ is the imaginary unit, and $\cos$ and $\sin$ are trigonometric functions cosine and sine respectively. This is a very unique relationship in the history of mathematics because it combines so much of our tools in mathematics. It has an exponential, it has a variable, it's got all of trigonometry, and it's even got this orthogonal reality of the ``imaginary'' domain. The formula describes how a vector on the complex manifests as a function of the variable $x$ alone. The implications of this will be appreciated with a time and effort on our parts.

\begin{center}
	\includegraphics[width=0.5\textwidth]{figures/09.01.png}
\end{center}
There is a special case of this formula known as Euler's identity and regardless of the number of times I prove it to myself, I just can't believe it all works out like this. If $x = \pi$ then
\begin{equation}
	e^{\jmath \pi} = -1
\end{equation}
which has just about every important mathematical thing in it: exponentials, imaginary numbers, constants of the universe, negative numbers, the single unit. In fact, add the one back to the other side and you'll see a zero appear from the depths. All this to say, recognizing the interrelatedness of exponential functions to just about everything else in mathematics and engineering will help us here and elsewhere immensely.


\section{The Laplace transform}
Mathematically, the Laplace transform is simple to describe. First, get yourself a curly L, looks like
this
\begin{equation}
	\mathcal{L}
\end{equation}
Next to that $\mathcal{L}$, put some curly brackets, like so

\begin{equation}
	\mathcal{L}\{\}
\end{equation}
Within those brackets put some function, say one of time, $f(t)$
\begin{equation}
	\mathcal{L}\{f(t)\} = \text{The Laplace transform of } f(t)
\end{equation}

We may also, sometimes, take a function's name, say $f(t)$, capitalize it, $F(t)$, and call that the Laplace transform. I am not personally the biggest fan of this notation as it can become quite distracting and confounding when combined with other transforms you might want to one day use, such as Fourier transforms, Hilbert transforms, and so on. 

I prefer to designate my Laplace transforms with a squiggly sign over a capitalized version of the function name, like so

\begin{equation}
	\mathcal{L}\{f(t)\} = \tilde{F}(s)
\end{equation}
That squiggly line, and many of those that follow, help explain what the Laplace transform is kind of doing to the reality around you. It's seeing relationships outside of time, cast in a plane of infinite exponential sine waves, with bottomless zeros and infinities you can draw a small circle around. The Laplace transform, should be, I believe, a transformative way of viewing the world around you. The power of the Laplace transform (and its ilk) is to reveal useful realities to us and our machines. Turns out there's more to life than just the time ahead of us. And we can usefully describe it. 

Some of those astute among you will notice I tried to slip something by you in the above definition. (Those of you glancing over the mathematics, get yourself back in the habit of $reading$ mathematics, out loud if need be. Make sure you understand what I'm saying before agreeing to it.) What I tried to sneak into this definition is a new variable $s$ that somehow our function, $f(t)$, became in terms of when transformed. We will get into the more mystical nature of the $s$-plane and its variable $s$, but for now, let's just call it some dummy variable. To you, at this point, it is just some stranger you've met in the street. You don't know them at all. First thing you've got to do, is say, Hello.

Before any of that, though, we must get to the definition of the definition of the Laplace transform. It is what we do to a function $f(t)$ to yield another equation $\tilde{F}(s)$. In that sense, the Laplace transform is defined as

\begin{equation}
	\mathcal{L}\{f(t)\} = \int_0^{\infty}e^{-st}f(t)dt
\end{equation}
which is to say it is the integral over all of time of the product of our function $f(t)$ with an exponential function raised to a power of time, $t$, scaled negatively with some dummy variable, $s$. Essentially, if I were to multiple my function, $f(t)$ at every point in time with a probing function $e^{-st}$ and summed it all up, what would I get? And more importantly to all of this, what would it tell me?

Another way of thinking about what the Laplace transform is ``doing'' is to think about what the mathematics itself is proposing to do. It's taking some function, $f(t)$, and wrapping it along a real and an imaginary exponential, producing both sines and cosines.

\section{The Laplace transform of 1}
Let me start off with the simplest number I can think of, sophist that I am. What is the Laplace transform of the number 1. If I can't solve this integral for the number 1, what the heck can I do? So let's convince ourselves that we can solve at least that much.

\begin{equation}
	\mathcal{L}\{f(t)\} = \mathcal{L}\{(1)\} = \int_0^{\infty}e^{-st}(1)dt
\end{equation}

Some, at this point, might be concerned with that $\infty$ up there in this integral. It creates what is sometimes called an improper integral, that is, one which has either (or both) limits of infinity or an integrand that approaches infinity at one or more points in the range of integration\footnote{This latter fact forms much of the basis of our understanding of control theory.}. To get around this improperness, we create for ourselves yet another dummy variable, let's call this one $A$. Thus we can also write the Laplace transform as

\begin{equation}
	\mathcal{L}\{(1)\} = \int_0^{\infty}e^{-st}(1)dt = \lim_{A\rightarrow \infty}\int_0^{A}e^{-st}(1)dt 
\end{equation}
which is the integral from $0$ to $A$ -- which we can do -- evaluated, ultimately, as $A$ goes infinitely. We take the antiderivative, bounded by our limits, to be
\begin{align}
	\lim_{A\rightarrow \infty}\int_0^{A}e^{-st}(1)dt &= \lim_{A\rightarrow \infty}\left[-\frac{1}{s}e^{-st}\right]_0^A \\
	&= \lim_{A\rightarrow \infty}\left[\left(-\frac{1}{s}e^{-sA}\right)-\left(-\frac{1}{s}\right)\right] \\
	&= \lim_{A\rightarrow \infty}\left[-\frac{1}{s}e^{-sA}+\frac{1}{s}\right]
\end{align}
Let us think about what our limit does. In this case the only factor with an $A$ involved is $-\frac{1}{s}e^{-sA}$. Here, if $A$ becomes huge, it knocks against that $s$, getting even bigger, and knocks against that negative sign to become very very negative. Having a very very negative negative number in our exponential function causes it to go screaming down to 0. In this case, the case where the function $f(t) = 1$, upper infinite limit has obliterated the term associated with it, whereas the lower limit has preserved the term associated with it. This leaves us with the final result:

\begin{equation}
	\mathcal{L}\{f(t)\} = \frac{1}{s}
\end{equation}

\section{The $s$-plane}
Where some see time, others see cycles, decay. The $s$-plane allows us to describe the physical world around us and much of the activity that occurs therein as a bunch of sines and exponentials, which is a quite clever way of squinting at a world that's always got us looking across time. Instead of charting a function or describing some behavior (perhaps of some system) as an amplitude changing over time, let us consider what such a function might look like if on one axis we had a variable describing how fast it decays over time and on the other axis we described how hard the system rattles at each frequency. In plotting such a function, one could see how a system will react under a myriad of conditions which you literally do not have the time to do if you confine yourself to just this slice of reality.

And so we must turn orthogonally to the only world we've ever known and stare into the $s$-domain.

We can start by taking a peak at our variable, $s$. For our first use of the Laplace transform, we didn't even need to know what $s$ was to work with it and to get an answer (of some sort). Indeed I took to calling it a dummy variable since the mechanical evaluation of the mathematics does not require it. It is important to make sure you can always do the math if you are ever called upon to do so. For much of what follows clear and clever shorthand will help us to view signals in a multitude of ways, including our bundling up of terms within $s$. As with much in life and in this class, $s$ comprises a real component, $\sigma$ and an imaginary component, $\omega$:
\begin{equation}
	s = \sigma + \jmath \omega 
\end{equation}
By themselves these two parameters are underwhelming. When raised as the exponent to an exponential and they start to describe a whole other world. Whether it has been stressed to you in your differential equations class or not, every single differential equation in the world has as at least one of its solutions the combination of exponential functions with and sine (or cosine) waves. The reasons for this range from the philosophic (we are trapped between repetition and degradation) to the mechanical (the derivatives of exponentials and sines and just scaled exponentials and sines). Let us take a look at the form $s$ takes itself for the Laplace transform:
\begin{equation}
	e^{-st} = e^{-(\sigma + \jmath \omega)t} 
\end{equation}
If were were to expand it out, we could perhaps convince ourselves that two separate yet equally important components emerge. One describes a function which dampens a response with time and one which oscillates back and forth with time.
\begin{equation}
	e^{-(\sigma + \jmath \omega)t}= \underbrace{e^{-\sigma t}}_{\text{damping}} + \underbrace{e^{-\jmath \omega t}}_{\text{sinusoids}}
\end{equation} 

The first of these behaviors we can probably prove to ourselves by inspection. If $s >0$, and let us say that it is, then as $t$ gets really large, the exponential function will die down to zero, under the influence of that negative infinity. Therefore, we are right to call $e^{-\sigma t}$ a damping term, since it will eventually smooth everything away to nothingness.

The second of these behaviors requires a brief detour back to our Eulerian days\footnote{One should get use to thinking in both Eulerian and Laplacian terms.}.

A benefit to this whole situation is that it reduces the mathematical complexity of the reality of the situation (represented by the mathematical operation of convolution, which we will get to in a couple weeks) to algebra. 


\section{The linearity of the Laplace transform}
The Laplace transform has at least one property that we must address here, its linearity. As the transform satisfies the two conditions of linearity previously established, homogeneity (a response is proportional to an excitation) and additivity (a response is the sum of excitations), we can use it with near reckless abandon for linear systems, i.e., those we will concern ourselves with here.

The linearity of the transform means that the transform of a sum is that same as the sum of individual transforms. Moreover, scaling constants are not affected by the transform and remain constant throughout. Mathematically this may all be stated as:
\begin{equation}
	\mathcal{L}\{a_1f_1(t) + a_2f_2(t)\} = a_1\tilde{F}_1(s) + a_2\tilde{F}_2(s)
\end{equation}
This result is convenient for more involved analyses and is one we must keep in mind.

\section{The Laplace transform of $e^{at}$}
To apply the Laplace transform, we take our $f(t)$, multiple it by our probing function, $e^{-st}$, and integrate over all time (for the moment, let us stipulate $s > a$):
\begin{align}
	\tilde{F}(s) = \mathcal{L}\{f(t)\} &= \int_0^{\infty}e^{-st}e^{at} dt \\
	&= \int_0^{\infty}e^{(a-s)t} dt \\
	&= \int_0^{\infty}e^{-(s-a)t} dt \\
	&= \left[-\frac{e^{-(s-a)t}}{s-a}\right]_0^{\infty} \\
	&= \left(-\frac{e^{-(s-a)\infty}}{s-a}\right) - \left(-\frac{e^0}{s-a}\right) \\
	&= 0 + \frac{1}{s-a} \\
	\mathcal{L}\{e^{at}\} &= \frac{1}{s-a}
\end{align}
From this we can perhaps see a pattern or two. Our Laplace transform took the exponent of the exponential and dropped it down in the denominator with $s$ as a subtractive component. Thus, one might just as well commit to memory that that Laplace transform of $e^{4t}$ is $\frac{1}{s - 4}$ as look it up in a table. From this simple derivation we now know that $e^{-0.25t}$ transforms to $\frac{1}{4s + 1}$ and could even convince ourselves that $3e^{-2t}$ becomes $\frac{3}{s+2}$. In fact, hopefully we can recognize that our previous derivation of the Laplace transform of the number 1 is just the case where $a = 0$ when $f(t) = e^{at}$.

You might wonder, what do I care about having a new fraction? What good does that do me? Already it has told you something you likely didn't know: as the values of $a$ and $s$ approach equality, the fraction blows up to infinity. We have what we call a ``pole'' at $s = a$.

\section{The Laplace transform of $dx/dt$}
Generally one does not spend a lot of time working through the mathematical conjuring of taking the Laplace of a derivative. The reason for this is because its final form is easy to learn, memorize, and deploy readily. 

Let us begin with a simple, perhaps much too simple, example.

\begin{equation}
	\frac{dx}{dt} - ax = 0
\end{equation}
Leveraging what we've learned of the linearity of our technique, let's find the Laplace transforms each portion of that equation. 

\begin{align}
	\text{Laplace transform 1 } &= \int_{0}^{\infty} e^{-st}\frac{dx}{dt} dt \\
	\text{Laplace transform 2 } &= \int_{0}^{\infty} e^{-st} x dt
\end{align}
The latter equation, ``Laplace transform 2'' is just another way of saying $\mathcal{L}\{x(t)\}$, so we can feel comfortable simply labeling that as $\tilde{X}(s)$.

The former equation requires us to integrate by parts. I will be assumed here that some calculus teacher somewhere once taught you the importance of integration by parts. It will be further assumed here that you know how to do it or could figure out how to do it if called upon to do so. As such, I will not belabor its implementation here and instead present my formulation of the logic of the situation for you to follow:
\begin{align}
	\mathcal{L}\left\{\frac{dx}{dt}\right\} &= \int_{0}^{\infty} e^{-st}\frac{dx}{dt} dt \\ 
	&= -\int_{0}^{\infty} x(t)\left(-se^{-st}dt\right) + \left[xe^{-st}\right]_0^{\infty} \\
	&= s\int_{0}^{\infty} x(t)\left(e^{-st}dt\right) + \left[\left(xe^{-\infty}\right) - \left(xe^{0}\right) \right]\\
	&= s\tilde{X}(s) - x(0)
\end{align}
What this end result suggests to us, is that the Laplace of the derivative of some function $x(t)$ is the Laplace of the function itself, $\tilde{X}(s)$, multiplied by $s$ to which an initial condition, $x(0)$ is subtracted.\footnote{In fact, we can expand this to $n$-th order derivatives. It will be of the form:
\begin{align}
	\mathcal{L}\left\{\frac{dx}{dt}\right\} &= s\tilde{X} - x(0)\\
	\mathcal{L}\left\{\frac{d^2x}{dt^2}\right\} &= s^2\tilde{X} - sx(0) - \frac{dx(0)}{dt} \\
	\mathcal{L}\left\{\frac{d^2x}{dt^2}\right\} &= s^2\tilde{X} - sx(0) - x^{\prime}(0) \\
	\mathcal{L}\left\{\frac{d^3x}{dt^3}\right\} &= s^3\tilde{X} - s^2x(0) - sx^{\prime}(0) - x^{\prime\prime}(0) \\
	\mathcal{L}\left\{\frac{d^4x}{dt^4}\right\} &= s^4\tilde{X} - s^3x(0) - s^2x^{\prime}(0) - sx^{\prime\prime}(0) - x^{\prime\prime\prime}(0)
\end{align}}

Combining these two results and rewriting:

\begin{align}
	\mathcal{L}\left\{\frac{dx}{dt} - ax = 0\right\} \rightarrow \mathcal{L}\left\{\frac{dx}{dt}\right\} - \mathcal{L}\left\{ax\right\} &= \mathcal{L}\left\{0\right\} \\
	s\tilde{X}(s) - x(0) - a\tilde{X}(s) &= 0 \\
	\tilde{X}(s)\left(s - a\right) &= x(0) \\
	\tilde{X}(s) &= \frac{x(0)}{s - a}
\end{align}
Without too much squinting, we can probably see that such a form is quite close to what we found for our exponential function's transformation, $\frac{1}{s - a}$. In fact, as will become clearer and more comfortable to you over time, is that indeed we can begin to quickly map a solution to a differential equation by first transforming it over the $s$-domain. In this particular case, we can hazard a guess that a solution to the differential equation $\frac{dx}{dt} -ax = 0$ is the function $x(t) = x(0)e^{at}$, which indeed from inspection appears to work.

It you didn't catch it there, let me point it out here: you have just learned a very neat mathematical trick for an engineer. This might be the quickest way to solve a differential equation that you will ever know of.

We've taken a differential equation and reduced it to simple algebra.

\section{The Laplace transform of $\sin$}
There are many ways of going about proving this to yourself. This is the one I prefer. It forces us to begin, as we should, with Euler's formula.

\begin{eqnarray}
	\mathcal L\left\{ {\sin at}\right\} &=& \mathcal L\left\{ {\frac {e^{iat}-e^{-iat} } {2i} }\right\} \\
	&=& \frac 1 {2i} \left( {\mathcal L\left\{ { e^{iat} }  \right\} - \mathcal L \left\{ { e^{-iat} } \right\} } \right) \\
	&=& \frac 1 {2i} \left( {\frac 1 {s-ia}-\frac 1 {s+ia} } \right) \\
	&=& \frac 1 {2i} \left( {\frac {s+ia-s+ia} {s^2+a^2} } \right) \\
	&=& \frac 1 {2i} \left({\frac{2ia}{s^2+a^2} }\right) \\
	&=& \frac a {s^2+a^2}
\end{eqnarray}

\section{The Laplace transform of $\int_{0}^{t} x(u) du$}

We have shown that taking the derivative of a function in the time domain is analogous to multiplying its Laplace transform by s in the frequency domain. Using integration by parts, we can similarly show that taking the integral of a function is analogous to dividing its Laplace transform by s. (Source: https://www.youtube.com/watch?v=IYOzTt-gB8A)

%\begin{align}
%	\mathcal L \left\{\int_{0}^{t} x(u) du}\right\} &= \int_{0}^{\infty} e^{-st}\int_{0}^{t} x(u) du dt \\ 
%	&= \left[\int_{0}^{t} x(u) du \frac{e^{-st}}{-s}\right]_{0}^{\infty} - \int_{0}^{\infty} X(t)\left(\frac{e^{-st}}{s}dt\right)\\
%	&= \left[\int_{0}^{t} x(u) du \frac{e^{-s\infty}}{-s} - \int_{0}^{0} x(u) du \frac{e^{-s0}}{-s} \right] + \frac{1}{s} \int_{0}{\infty}\\
%	&= \frac{\tilde{X}(s)}{s} + x(0)
%\end{align}



\section{The Laplace transform in RLC circuits}
\subsection{Resistors}
To find the voltage drop across a resistor we apply Ohm's law, $v = iR$. To go from the time-domain to the Laplace-domain for is notationally simple:
\begin{align}
	v(t) & \rightarrow \tilde{V}(s) \\
	i(t) & \rightarrow \tilde{I}(s) \\
	v(t) = i(t)R & \rightarrow \tilde{V}(s) = \tilde{I}(s)R
\end{align}
That is, the voltage drop in the $s$-domain is equal to the current in the $s$-domain, scaled to the resistance, $R$.


\subsection{Inductors}
\begin{align}
	v(t) & \rightarrow \tilde{V}(s) \\
	i(t) & \rightarrow \tilde{I}(s) \\
	v(t) &= L\frac{di(t)}{dt} 
\end{align}
Recall from a previous section that the Laplace of a derivative, $\mathcal{L}\{\frac{dx}{dt}\}$ is equal to the product of the Laplace of the original function and $s$ (with the initial conditions subtracted): $s\tilde{X}(s) - x(0)$. We can apply this same logic to solve the above problem, yielding the inductor's behavior:
\begin{align}
	v(t) &= L\frac{di(t)}{dt} \\
	\tilde{V}(s) &= L(s\tilde{I}(s) - i(0)) \\
	\tilde{V}(s) &= sL\tilde{I}(s) - Li(0)
\end{align}

We can even redraw the circuit to see it as we would from the $s$-domain. If I've got a circuit, that starts with a horizontal wire with a current $\tilde{I}(s)$ going through it, I've currently got a potential of $\tilde{V}(s)$ at the rightmost end of that wire. Connecting that wire vertically, we can walk across our equation and see what it's telling us. Our first element is going to affect the current by $sL\tilde{I}(s)$. That's our inductor. We continue our KVL-inspired walk and see that we have a negative drop (a rise) in potential equal to $Li(0)$, the current at our initial condition scaled to our inductor's size. This is a potential source. From this, we have a new vantage point to survey circuits from. Where before we thought of voltage sources as blackboxes, there's a curious response that gets induced in them.

If we were to solve for $\tilde{I}(s)$ by rearranging the equation
\begin{equation}
	\tilde{I}(s) = \frac{\tilde{V}(s)}{sL} + \frac{i(0)}{s}
\end{equation}
and we can redraw this as a parallel circuit. Or essentially, we are losing $\tilde{I}(s)$ through a node with two branches (see the two components on the right hand side). So for our rightmost branch we have a current source, scaled to $1/s$, and then we have our inductor, with an inductance of $sL$.

This technique is extensible to other electronic components. And is in fact another way of transforming our circuits, should the need or desire arise, more specifically source transformation with voltage sources in series with an inductor and current source in parallel with an inductor.

\subsection{Capacitors}

Imagine you've got current going into a capacitor and it's forming a potential on either side of it. we can describe this as 
\begin{equation}
	i = C\frac{dv}{dt}
\end{equation}
We can take the Laplace of both sides
\begin{align}
	\mathcal{L}\{i\} &= \mathcal{L}\{C\frac{dv}{dt}\} \\ 
	\tilde{I}(s) &= C(s\tilde{V}(s) - v(0)) \\
	\tilde{I}(s) &= Cs\tilde{V}(s) - Cv(0)
\end{align}
which is to say we have a current, $\tilde{I}(s)$, being divvied up at a node between two branches. The rightmost branch is a source, $v(0)$, scaled by a capacitance, $C$. (And because it's negative, it turns out this branch is actually entering the node just like $\tilde{I}(s)$). The other branch is a capacitor, $1/C$, multiplied by $s$. 

Just as before with the inductor we can also imagine transforming this source. The simplest way I can think to prove that to ourselves is to rearrange our equation
\begin{equation}
	\tilde{V}(s) = \frac{1}{sC}\tilde{I}(s) + \frac{v(0)}{s}
\end{equation}
From this we can see that our potential drop across a series connected components, $\tilde{V}(s)$ is first dropped by a capacitor $1/sC$, and then dropped by a source, $v(0)$ scaled to $s$.

\begin{equation}
	v(t) = \frac{1}{C}\int_{t_1}^{t_2}
\end{equation}

\subsection{RLC circuits}
Combining these results and recognizing the linearity of each, in a series RLC circuit, we can merely add these all up to yield:
\begin{equation}
	Ri(t) + L\frac{di}{dt} + \frac{1}{C}\int_{-\infty}^{\tau = t}i(\tau)d\tau = v(t)
\end{equation}

\subsubsection{An example}
What is the Laplace transform of a series RLC circuit with $L$ = 1, $R$ = 7, and $C$ = 10.






\chapter{The Laplace Transform: II. How to use it}
02/19/2019
\minitoc
\newpage

\section{A brief review}
Returning for a moment to our series RLC circuit from last time, once converted with the Laplace transform, the system is described via a second-order polynomial. 
\begin{align}
	v_L &= L\frac{di_L(t)}{dt} \\
	v_R &= i_R(t)R \\
	v_C & = \frac{1}{C}\int_0^t i(\tau)d\tau + V_c(0)
\end{align}
When all elements are placed in series, a simple KVL walk will suggest to us that
\begin{equation}
	L\frac{di_L(t)}{dt} + i_R(t)R + \frac{1}{C}\int_0^t i(\tau)d\tau + V_c(0) = 0
\end{equation}
Transforming via Laplace\footnote{It may do you some good to remember that in the Laplace domain, differentiation is achieved by multiplying by $s$ and integration is accomplished by dividing by $s$.}:
\begin{eqnarray}
	L(s\tilde{I} - I_0) + R\tilde{I} + \frac{\tilde{I}}{sC} + \frac{V_0}{s} = 0 \\
	s^2\tilde{I} - sI_0 + s\frac{R}{L}\tilde{I} + \frac{\tilde{I}}{C} + V_0 = 0 
\end{eqnarray}
Rearranging and rewriting
\begin{equation}
	\tilde{I} = \frac{sI_0 - \frac{V_0}{L}}{s^2 + \frac{R}{L}s + \frac{1}{LC}}
\end{equation}
If we give to our RLC circuit the values of $L$ = 1, $R$ = 7, and $C$ = 10 and initial conditions $V_0$ =10 and $I_0$ = 1 we can see, without too much effort that the equation becomes

\begin{equation}
	\tilde{I} = \frac{s - \frac{10}{1}}{s^2 + \frac{7}{1}s + \frac{1}{1 \cdot 10}}
\end{equation}
\begin{equation}
	\tilde{I} = \frac{s - 10}{(s+2)(s+5)}
\end{equation}
In this form, we now see laid bare before us two important places in the $s$-domain: poles and zeros.


\section{Two important places, zeros and poles}
It may help us to recall that a point in the $s$-domain characterizes a system's response as a  combination of decay, growth, and oscillations. Thus, since $s$ is a coefficient of a exponential function, any value thereof will tell us whether the function 
\begin{itemize}
	\item will growth with respect to time (s $>$ 0);
	\item will decay with respect to time (s $<$ 0); 
	\item will oscillate over time (s has an imaginary component); or
	\item will remain constant (s = 0).
\end{itemize}

A system can be characterized by its combination of these behaviors by finding the roots of $s$ for the system as described in the $s$-domain. These can either be a ``zero'' or a ``pole''.

In our systems, as the value of $s$ approaches
\begin{itemize}
	\item a \textbf{zero}, the numerator of the transfer function approaches 0 and thus so too does the whole system and
	\item when approaching a \textbf{pole}, the denominator of the transfer function approaches 0, causing the system to approach the infinite.
\end{itemize}

Recalling our RLC circuit:
\begin{equation}
	\tilde{I} = \frac{s - 10}{(s+2)(s+5)}
\end{equation}

For this system, we have three special points.
\begin{enumerate}
	\item $s \rightarrow 10$ causes the numerator to approach 0 (\textbf{and is thus a zero}) and so too will the system;
	\item $s \rightarrow -2$ causes the denominator to approach 0 (\textbf{and is thus a pole}), causing the system response to be quite large; and
	\item $s \rightarrow -5$ also causes the denominator to approach 0 (\textbf{and is thus a pole}), causing the system response to be quite large.
\end{enumerate}
 


\newpage
\section{The general form of just about all systems}
As it turns out, most systems on earth and else can be expressed by a generalized second-order equation 
\begin{equation}
	s^2 + 2\zeta\omega_n s  + \omega_n^2 = f(t)
\end{equation} 
where $\zeta$ is the damping coefficient and $\omega_n$ is the natural frequency of the system, and $f(t)$ is some ``forcing'' function.

In our case, $\zeta = \frac{R}{2}\sqrt{\frac{C}{L}}$\footnote{It may be more intuitive to think of this terms instead as $\zeta = \alpha/\omega_n$, where $\alpha = R/2L$ represents an attenuation coefficient (of units nepers/second).} and $\omega_n = \sqrt{\frac{1}{LC}}$



\begin{itemize}
	\item \textbf{The natural frequency} is the frequency at which a system tends to oscillate in the absence of any driving or damping forces. It's just what wants to happen.
	\item \textbf{The damping coefficient}, or the damping ratio, is a dimensionless measure of decay. It describes how oscillations in a system die down after a disturbance.
\end{itemize}

For the moment, we will consider systems for which $s < 0$. These systems are guaranteed to be ``stable'' in the sense that they will not grow exponentially large. Rather, given the fullness of time, we are assured that each of these types of systems will eventually decay away to a stable point.

\subsection{Undamped systems}
When $\zeta = 0$, a system comprises a single harmonic oscillator, as is the case of a mass suspended by a perfect spring. This is what we would expect with no ``real'' components ($e^{\jmath \omega_n t}$) 
\begin{itemize}
	\item Two imaginary poles at $\pm \jmath \omega_n$
	\item Undamped sinusoid with frequency equal to the imaginary part of the pole. 
	\item Solution: $f(t) = A\cos (\omega_n t - \phi)$
\end{itemize}

\subsection{Underdamped systems}
When $0< \zeta < 1$, a system  will initially oscillate in response to a disturbance before dying down to zero. Can be described by the function $e^{\jmath\omega_n\sqrt{1-\zeta^2}t}$ 

\begin{itemize}
	\item Two complex poles at $-\sigma \pm \jmath \omega_d$ 
	\item Damped sinuosoid with exponential envelope. Time constant is equal to the reciprocal of the pole's real part. 
	\item Solution: $f(t) = Ae^{-\sigma t}\cos (\omega t - \phi)$
\end{itemize} 

\subsection{Critically damped systems}
When $\zeta = 1$, a system  decays away as fast as possible with no oscillations or ``overshoots''. As you might expect, this is a desirable outcome for many many cases of engineering design (e.g., the rate at which a door opens or closes) 
\begin{itemize}
	\item Two real poles at $-\sigma$. 
	\item Response: the sum of two scaled exponentials of the same time constant, one scaled to time. 
	\item Solution: $f(t) = k_1e^{-\sigma t} + k_2te^{-\sigma t}$
\end{itemize}

\subsection{Overdamped systems}
When $\zeta > 1$, a system in does not oscillate and decays  at a slower rate than the critical case. 
\begin{itemize}
	\item Two real poles at $-\sigma_1$ and $-\sigma_2$. 
	\item Response: sum of two exponentials with different time constant
	\item Solution: $f(t) = k_1e^{-\sigma_1 t} + k_2 e^{-\sigma_2 t}$
\end{itemize}


\newpage
\section{``Inverting'' the Laplace transform}
\begin{equation}
	{\mathcal  {L}}\{f\}(s)={\mathcal  {L}}\{f(t)\}(s)=\tilde{F}(s)
\end{equation}
\textbf{Mellin's inverse formula} An integral formula for the inverse Laplace transform, called the Mellin's inverse formula, the Bromwich integral, or the Fourier–Mellin integral, is given by the line integral:
\begin{equation}
	{\displaystyle f(t)={\mathcal {L}}^{-1}\{\tilde{F}(s)\}={\frac {1}{2\pi i}}\lim _{T\to \infty }\int _{\gamma -iT}^{\gamma +iT}e^{st}\tilde{F}(s)\,ds,}
\end{equation}
where the integration is done along the vertical line Re(s) = $\gamma$ in the complex plane such that $\gamma$ is greater than the real part of all singularities of F(s) and F(s) is bounded on the line, for example if contour path is in the region of convergence. If all singularities are in the left half-plane, or F(s) is an entire function , then $\gamma$ can be set to zero and the above inverse integral formula becomes identical to the inverse Fourier transform.

In practice, computing the complex integral can be done by using the Cauchy residue theorem. Or simply by recognizing common forms.

\subsection{A convenient way of doing it}
To find solutions to the equations left behind after we have transformed them via Laplace, I personally prefer to use the ``cover-up method''.

Returning to our function describing current throughout the circuit
\begin{equation}
	\tilde{I} = \frac{s - 10}{(s+2)(s+5)}
\end{equation}

\newpage
\section{Worksheet}
\subsection{Problem 1}
Find the Laplace transform of the following:
\begin{itemize}
	\item $f(t) = 9$
	\item $f(t) = \delta(t)$ the Dirac-delta function
	\item $f(t) = e^{-3t/2}$
	\item $f(t) = \sin\omega t$
\end{itemize}

\textbf{Solution}


\subsection{Problem 2}
What is the differential equation describing an inductor, a resistor, and a capacitor in series?

\textbf{Solution}


\subsection{Problem 3}
What is the differential equation describing an inductor, a resistor, and a capacitor in series? What is a solution to that  differential equation if L = 1 H, R = 1 $\Omega$, and C = 100 mF? Take the Laplace transform of the differential equation. What are its poles and zeros?

\textbf{Solution}

\newpage

\subsection{Problem 4}
Draw an s-plane. Label the axes. Plot the poles and zeros you found. What can you say of the behavior of the system? If the poles had an imaginary component to them (that is, if they were pushed along the vertical axis), how would that affect our system?

\subsection{Problem 5}
Draw an s-plane. Label the axes. Plot these poles -- (-2,0) and (-5,0) -- and this zero -- (0,0). Plot what the signal would look over time at each of these poles and zeros. 


\chapter{Circuits as ODEs: I. First-order}
02/21/2019 – Lecture 12. 
\section{Source-free RC circuits}
\subsection{One resistor, one capacitor}
This circuit is comprised of one resistor and one capacitor in series with no source.
To find the function for the voltage $V(t)$ of this source-free circuit, we start by doing KCL.
Using the node in between the branch containing the resistor and the branch containing the the capacitor, we get that the current coming into the node from the resistor is $(Vg-Va)/R$, in which Va is just the voltage at the node and Vg is the ground voltage, which would be zero.
The equation for the current going out of the node through the capacitor is $C(dV/dt)$, which can be found from the equation $Q=CV$.
From these, we get the equation $(0-Va)/R - C(dVa/dt) = 0$, which simplifies to $dVa/dt + Va/RC = 0$.
We now take the LaPlace transform which gives us: $s\tilde{V} - V(0) + (1/RC)V{tilde} = 0$
Simplifying gives us: $\tilde{V} = V(0)/(s+1/RC)$. The pole is s = $-1/RC$.
Taking the inverse LaPlace to get it back into the t-domain gives us: $V(t) = V(0)e^(-t/RC)$
As seen in this equation and with the negative pole, the solution to the circuit decays, making it stable.
This is what we'd expect for a circuit, for it must be stable otherwise we would not see it in real life. 


\subsection{Two or more resistors and/or capacitors}
\section{Source-free ``active'' circuits}
\section{First-order systems with sources}
\section{Several singular functions}
\subsection{Unit step function, $u(t-t_0) = 1, t>t_0$}
\subsubsection{The Laplace transform of the unit step function}
The unit step function essentially has two components to it. It will initially start out at 0, but then at a specific time that you choose ($t_0$ term) the function will become one. Therefore, we can take the Laplace of 1 and we will essentially get the Laplace transform of a unit step function to be $1/s$.
\subsection{Unit impulse function, $\delta(t) = du(t)/dt$}
\subsubsection{Its ``sifting'' abilities}
The unit impulse function is really good for ``sifting'' purposes. Since the impulse function is essentially a very sharp peak at a certain $t_0$, you can imagine that if we convolve a function with a unit impulse function at a certain time, we will simply get the point of that function at that certain time. Therefore, we can sort of imagine a function as a file and in order to get to a certain document in that file, we can utilize the unit impulse function to ``sift'' through it. 
\subsubsection{The Laplace transform of the unit impulse function}
The Laplace transform of a unit impulse function is just 1.
\subsection{Unit ramp function, $r(t) = \int u(t)dt $}
\subsubsection{The Laplace transform of the unit impulse function}
We know that the ramp function is 0 before a certain time $t_0$ and after that it is ``$t$'' after a certain $t_0$. Therefore, the Laplace transform of a ramp function is simply the Laplace transform of t, which is $1/(s^2)$.



\chapter{Circuits as ODEs: II. Second-order}
02/26/2019 – Lecture 13. 
\section{A series RLC circuit}



\chapter{System response: I. Convolution}
02/28/2019
\minitoc
\newpage
\section{An introduction to thinking in systems}
Completely characterizing the input v. output properties of a system by exhaustive measurement of all parameters involved is usually not possible. As

\section{Pulse and impulse}
We have previously noted that the unit impulse function 

\begin{equation}
	\delta(t) = \begin{cases} 
      \infty, & t = 0\\
      0, & t \neq 0\\
   \end{cases}
\end{equation}
But another useful way of conceiving of the impulse signal is as a limiting case of the ``pulse signal'', $\delta_{\Delta}(t)$:
\begin{equation}
	\delta_{\Delta}(t) = \begin{cases} 
      \frac{1}{\Delta}, & \text{if } 0< t < \Delta \\
      0, & \text{otherwise }\\
   \end{cases}
\end{equation}
It can be seen without much effort that the impulse signal is equal to the pulse signal when the pulse gets infinitely short.

It may also be helpful to consider integration at this point as a limiting case of summation:
\begin{equation}
	\int_{t = -\infty}^{\infty}x(t)dt = \lim_{\Delta\rightarrow 0}\sum_{k=-\infty}^{\infty}x(k\Delta)\Delta
\end{equation}
This even allows us to perhaps more easily reckon that the unit step signal can be obtained by integrating the unit impulse
\begin{equation}
	u(t) = \int_{-\infty}^{t}\delta(\tau)d\tau
\end{equation}

Well, as it turns out, any signal can be expressed as a sum of scaled and shift unit impulses. Indeed, thanks to our digital world, this is indeed how it mostly works out there in the real world, wherein discrete samples from an original signal may be approximated more or less exactly as pulse signals scaled to the amplitude of the sample. Mathematically,
\begin{equation}
	x(t) \approx \sum_{k=-\infty}^{\infty}x(k\Delta)\delta_{\Delta}(t-k\Delta)\Delta
\end{equation} 

Graphically,
\includegraphics[width=\textwidth]{figures/13.01.png}

As we let $\Delta\rightarrow 0$, our approximation of $x(t)$ hones in on reality with the summation approaching an integral and the pulse approach the unit impulse:
\begin{equation}
	x(t) = \int_{-\infty}^{\infty}x(\tau)\delta(t-\tau)d\tau
\end{equation}

Thus, we can represent any signal as an infinite sum of shifted and scaled unit impulse. This is indeed that best way we human beings have thus far figured out how to store information. A train of 1s and 0s.

\section{Convolution}
Recall, the two principles of linearity inherent in our studies here: \textbf{homogeneity} which enables the scaling of linear systems and \textbf{additivity} which allows us to sum up two or more linear systems. From these we arrive at the emergent phenomenon of \textbf{superposition} which states that $T(ax_1+bx_2)= aT(x_1) + bT(x_2)$.

There is also a facet of these systems known as \textbf{shift-invariance} in which we say that so long as we don't (cause long-term) damage (to) a system by inputting some signal, then whether we input that signal at a time $t$ or a time $t + \tau$ some while later will not significantly alter the response of the system. Such systems are said to be shift-invariant, and we will consider their behaviors here.\footnote{A shift-invariant system is the discrete equivalent of the time-invariant systems we have heretofore concerned ourselves with.}

To characterize a linear, shift-invariant, one need only measure how a system responds to a unit impulse. This response is known as the \textbf{impulse response function} and in principle it allows us to predict how a system will respond to any other possible stimulus. 

Graphically, this may be easy to understand. We can imagine that an impulse given to some system cause it to rapidly rise, then decay (exponentially) away. If we do this for ever point in time for a system of interest, we could fully characterize the total response of the system.

\begin{center}\includegraphics[width=\textwidth]{figures/13.03.png}\end{center}

Mathematically, we can call upon our two principles of linearity to guide our way. Let us define some output, $y(t)$, of a system, characterized by transfer function\footnote{It has helped some students to think of a ``transfer'' function merely as a ``transform'' function such that given an input, $x(t)$, there exists an output, $y(t)$, which is mapped from $x(t)$ via the transform, $T$.}, $T$, and an input signal, $x(t)$
\begin{align}
	y(t) = T[x(t)] &= T\left(\int_{-\infty}^{\infty}x(\tau)\delta(t-\tau)d\tau\right) \\
	&= T\left(\lim_{\Delta\rightarrow 0}\sum_{k=-\infty}^{\infty}x(k\Delta)\delta_{\Delta}(t-k\Delta)\Delta\right)
\end{align}
From additivity,
\begin{align}
	y(t) &= \lim_{\Delta\rightarrow 0}\sum_{k=-\infty}^{\infty}T\left(x(k\Delta)\delta_{\Delta}(t-k\Delta)\Delta\right)
\end{align}
Taking the limit,
\begin{align}
	y(t) = \int_{-\infty}^{\infty}T(x(\tau)\delta(t-\tau)d\tau)
\end{align}
Homogeneity suggests:
\begin{align}
	y(t) = \int_{-\infty}^{\infty}x(\tau)T(\delta(t-\tau))d\tau
\end{align}
If we then define for ourselves the response of $T$ to the unshifted unit response $h(t) = T[\delta(t)]$, then via shift-invariance we are left with
\begin{align}
	y(t) = \int_{-\infty}^{\infty}x(\tau)h(t-\tau)d\tau
\end{align}

This last result is worth sitting down and thinking about. Its consequence is that for any shift-invariant system, $T$, once we known its impulse response, we know how it will respond to every other system, since every other signal is merely a combination of scaled and shifted impulses.

This way of combining signals is so widespread and useful that it has its own mathematical shorthand, ``$*$'', known as convolution. For any two signals, $x$ and $y$, an output, $z(t)$, may be found by convolving $x$ with $y$
\begin{equation}
	z(t) = x*y = \int_{-\infty}^{\infty}x(\tau)y(t-\tau)d\tau
\end{equation} 
One way of thinking about this operation is it is as if you took two signals, left one where it was, put there at one side of eternity and found the piece-wise multiplication and sum of two signals as you moved the other signal to the other side of eternity.

\subsection{Convolution as a series of weighted sums}

\includegraphics[width=\textwidth]{figures/13.02.png}

The output of convolution may be thought of as a series of weight sums. More specifically, a linear, shift-invariant system will output a response that is a weighted sum of the inputs. That \textit{weighting function} which helps maps the inputs to the output is very closely related to the impulse response of the system. Indeed, the impulse response and weighting function are time-reversed copies if each other (as shown in the top part of the graph).

\subsection{A few properties of convolution}
\begin{itemize}
	\item Commutative: $x*y = y*x$
	\item Associative: $(x*y)*z = x*(y*z)$
	\item Distributive: $(x*z)+(y*z)=(x+y)*z$
\end{itemize}

\subsection{The Laplace transform of a convolution}
To editorialize a bit, perhaps the single most important thing to get out of this chapter is that convolution in the time-domain is the same as multiplication in the $s$-domain. That is
\begin{equation}
	\mathcal{L}(f*g) = \tilde{F}(s)\cdot\tilde{G}(s) = \tilde{G}(s)\cdot\tilde{F}(s) = \mathcal{L}(g*f)
\end{equation}
Let us prove this to ourselves.\footnote{0$^-$ and $t^+$ will be avoided in this derivation for convenience, but the interested reader is encouraged to redo this proof with their inclusion. It should come out the same}


\section{A few examples}

\subsection{Convolution of an exponential and a $\sin$ function}

\includegraphics[width=\textwidth]{figures/13.04.png}


\chapter{System response: II. Stability}
03/12/2019
\minitoc
\newpage

\section{Bound inputs, bound outputs}
Systems may be thought of as stable or unstable. That is, they either approach some known, finite value, or they approach some infinite value.

One of the more convenient ways to think about stability, is to concern ourselves with systems whose outputs are based on its inputs. With such an approach, we may say of a stable system that the output should be \textit{bounded} for each \textit{bounded} input for each and every time point. Such systems are \textbf{bounded-input, bounded-output stable}, or exhibit \textbf{BIBO stability}.

What ``bounded'' means here is that we are confining the values being input (and thus output) to some \textit{range} of values. Another way of thinking about it is that from time of $-\infty$ to $+\infty$, the amplitude of the input is never infinite. For a BIBO stable system, the amplitude of the output will never be infinite during all that time. For a BIBO unstable system, that is not true for at least one point in time.

Those really looking for formality here may use the following definition of boundedness: a signal is bounded if there exists a finite value $B > 0$ such that the magnitude of the signal never exceeds, $\|y(t)|\leq B, \forall t\in \mathbb {R} $.

\subsection{A few standard bounded signals}
\begin{itemize}
	\item \textbf{DC signals}, e.g., $y(t) = 2$ | consider that at all points of $t$, the amplitude will equal a constant
	\item \textbf{Sines and cosines}, e.g., $y(t) = \sin(t)$ | for all points $t$, the amplitude is bounded In other words, to -1 and +1
	\item \textbf{The unit step function}\footnote{But what about the unit impulse function? Doesn't this violate our ``never infinite'' rule? Only kind of! First, it will help to realize that the impulse function, $\delta(t)$, is an \textit{idealization} of a signal that is \begin{enumerate}
		\item very large near $t = 0$,
		\item very small away from $t = 0$, and
		\item has an integral of 1.
	\end{enumerate} That is, it is a signal whose width is $\epsilon$ and whose amplitude is $1/\epsilon$, where $\epsilon$ is very small. In fact, the impulse function is formally defined by the property that $\int_{-\infty}^{\infty}f(t)\delta(t)dt = f(0)$. Hence, $\delta(t)$ is not \textit{really} defined for any \textit{t}, only its behavior within an integral. For a simple demonstration of this conundrum, consider Problem 14.7.3.5.}, e.g., $y(t) = u(t)$ | for all points $t$, the amplitude is either 0 or 1.
\end{itemize}

\newpage

\subsection{A few example bounded signals}
In the examples which follow, we can an imagine an input signal, $x(t)$, going into a system (``being convolved with''), such that it produces an output, $y(t)$. Somewhat graphically, $x(t) \rightarrow \text{system} \rightarrow y(t)$.
\begin{enumerate}
	\item $y(t) = t\cdot x(t)$ | Let's choose a simple input, $x(t) = u(t)$, such that $y(t) = t\cdot u(t) = r(t)$. \textit{Note here that the product of a signal with time is equal to its integral.}\footnote{It should also further be recalled here that multiplication by $s$ is the equivalent of taking the derivative and division by $s$ its integral. Further note that under appropriate circumstances $t = 1/s$. From this, the result reported above should follow.}
	\subitem \textbf{However! This is not a stable system.} Simply because an input is bounded does not mean a system will be stable. Such boundedness merely gives us a specific means by which to evaluate stability in systems. In this case a bounded input produces an unbounded output.
	\item $y(t) = x(t) + 5$ | Here we see the effects of a possible DC offset (the ``+5'' term) on a system. Let's input another simple bounded signal, $x(t) = -3.3$. Then, $-3.3 \rightarrow \text{system} \rightarrow y(t) = -3.3 + 5 = 1.7$. 
	\subitem \textbf{Here we have a bounded input which produces a bounded output.} Hence, we have a stable system!
	\item $y(t) = x(t)\cdot\sin(t)$
\end{enumerate}

\section{Stability at poles and zeroes}
Recall our series RLC circuit. It produced a transfer function with two general points of interest, poles and zeroes. Pondering still further what the Laplace transform tells us, we are ultimately taking the integral sum of an exponential function whose exponential is determined by either a pole or a zero. 

\begin{itemize}
	\item In the case of \textbf{zeros}, summation is to zero. It is equivalent to taking the integral of a sinusoidal function centered at 0 over time. It sums to zero. Hence the name.
	\item The other case that we care about is that of the \textbf{poles} in which the summation is just barely infinite. In this case, you can imagine a sinusoid shifted such that its minimum intersects the x-axis. In this scenario, the area under the curve as time approached infinity would be just barely infinite.
	\subitem It would become unbounded if its amplitude slightly increased with time and it would be finite if its amplitude decreased with time. The sweet spot representing this boundary is our ``pole''.
\end{itemize}

It is within those just barely infinite realms that we will meet our destiny.


\section{Resonance}
Resonance can be both useful and annoying and it is a prime example of marginal stability.
Electrical resonance occurs in a electric system at a particular resonant frequency when the impedance of the circuit is 
\begin{itemize}
	\item at a minimum in a series circuit or
	\item at maximum in a parallel circuit.
\end{itemize}
That is, resonance usually happens when the transfer function peaks in absolute value. 

\begin{center}
	\includegraphics[width=\textwidth]{figures/14.01.png}
\end{center}


\section{Q factor}
\begin{itemize}
	\item The quality factor (``Q factor'') is a dimensionless parameter that describes how damped an oscillator or resonator is. 
	\item It characterizes a resonator's bandwidth relative to its center frequency.
	\item A higher Q indicates a lower rate of energy loss relative to the stored energy of the oscillator, i.e., the oscillations die out more slowly. 
	\item For example, a pendulum suspended from a high-quality bearing, oscillating in air, would have a high Q, while a pendulum immersed in oil would have a low Q.
\end{itemize}
\begin{center}
	\includegraphics{figures/14.02.png}
\end{center}
The bandwidth $\Delta f = f_2 - f_1$ of a damped oscillator is shown on a graph of energy versus frequency. The Q factor of the damped oscillator, or filter, is $f_c/\Delta f$. The higher the Q, the narrower and ‘sharper’ the peak is.

The resonant frequency $f_r$, has a resonance width or \textbf{full width at half maximum (FWHM)} of $\Delta f$ and can be used to defined the Q factor as 
\begin{equation}
	Q = \frac{f_r}{\Delta f}
\end{equation}




\newpage
\section{A few examples}
\subsection{What is $\tilde{Z}(s)$ (i.e., $\tilde{V}(s)/\tilde{I}(s)$) of a resistor and a capacitor in parallel?}
\begin{enumerate}
	\item Show the \textit{frequency response} (i.e., $\tilde{Z}(s)$ v. frequency) of such a system. If it helps to ascribe values, assume that the resistor is 1 ohm and the capacitor is 10 farad.
	\item Comment on the behavior. Is it like anything you have seen before?
	\item (\textit{The kind of question that might be on a glorified quiz.}) If the current through the circuit is 1 amp when $t \geq 0$, what will be the voltage response?
\end{enumerate}

\subsection{What is the voltage time response of a system with a given impedance?}
Given that 
\begin{equation}
	\tilde{Z}(s) = \frac{s^3+9}{(s+1)(s+3)}
\end{equation}
\begin{enumerate}
	\item What is the \textit{time response} of potential to a unit step of current?
	\item (\textit{The kind of question that might be on a glorified quiz.}) Which exponential term will most significantly affect the signal?
\end{enumerate}

\subsection{Is the system stable?}
\begin{enumerate}
	\item A system, $y(t) = \int_{-\infty}^{\infty}x(\tau)d\tau$, when $x(t) = \cos(t)$.
	\item A system, $y(t) = \int_{-\infty}^{\infty}x(\tau)d\tau$, when $x(t) = u(t)$.
	\item A system, $y(t) = x(t)/t$, when $x(t) = 2$.
	\item A system, $y(t) = dx(t)/dt$, when $x=1$.
	\item A system, $y(t) = dx(t)/dt$, when $u(t)$.
\end{enumerate}


\section{Lecture 14 Board Pictures (3/12/19)}

\includegraphics[width=\textwidth]{figures/3-12-19_Fig.1.jpg}
\\
\includegraphics[width=\textwidth]{figures/3-12-19_Fig.2.jpg}
\\
\includegraphics[width=\textwidth]{figures/3-12-19_Fig.3.jpg}
\\
\includegraphics[width=\textwidth]{figures/3-12-19_Fig.4.jpg}
\\
\includegraphics[width=\textwidth]{figures/3-12-19_Fig.5.jpg}
\\
\includegraphics[width=\textwidth]{figures/3-12-19_Fig.6.jpg}


\part{\& Signals}



\chapter{System response: III. Active filters}
03/14/2019 

\section{Recall, an inverting amplifier}
Given an inverting amplifier (as reported below), we know that the ratio of the output to the input is
\begin{equation}
	\frac{V_o}{V_i} = -\frac{R_f}{R_{in}}
\end{equation}

\begin{center}
	\includegraphics[width=0.5\textwidth]{figures/15.01.png}
\end{center}
This can be more generally thought of in terms of a ratio of the feedback impedance, $Z_f$, and the input impedance, $Z_{in}$

\begin{equation}
	\frac{V_o}{V_{in}} = -\frac{Z_f}{Z_{in}}
\end{equation}
With this in mind, we can start to analyze any number of impedance combinations to determine four very useful types of active filters.

\section{An ideal inverting integrator}
\textbf{The first combination of impedances we will consider is a resistor, $R_{in}$, at the input and a capacitor, $C_F$ in the feedback loop.}

This will produce a (voltage gain) transfer function, $V_o/V_{in}$:

\begin{equation}
	\frac{V_o}{V_{in}} = -\frac{Z_f}{Z_{in}} = -\frac{1/sC_F}{R_{in}} = -\frac{1}{sR_{in}C_F}
\end{equation}

We may remember that division by $s$ in the $s$-domain is equivalent to integration in the time domain. Thus, the time-varying output is 

\begin{equation}
	V_o(t) = -\frac{1}{R_{in}C_F}\int_{-\infty}^{t} V_{in}(\tau)d\tau
\end{equation}

Hence, this system will produce an output which integrates the input with respect to time and scales it to $1/R_{in}C_F$
An \textbf{integrator} produces an output proportional to the time integral of its input.
\\

\textit{Question to ponder: what frequency wouldn't produce an output voltage?}

\subsection{A modification yielding a low pass filter}
Because at 0 Hz, $C_F$ produces an open circuit, the feedback loop is no longer connected and hence no output is produced, we can include a resistor in combination with that feedback capacitor, so as to prevent this. This leads to our second combination: \textbf{a resistor, $R_{in}$, at the input and a capacitor, $C_F$, in parallel with a resistor, $R_F$, in the feedback loop.}

This will yield a transfer function of

\begin{equation}
	\frac{V_o}{V_{in}} = -\frac{Z_f}{Z_{in}} = -\frac{R_F \vert \vert (1/sC_F)}{R_{in}} = -\frac{1}{R_{in}C_F}\cdot \frac{sR_F C_F }{1 + sR_F C_F}
\end{equation}

And let us consider its response with respect to frequency. To help matters, let us consider resistors of value 1 $\Omega$ and a capacitor of 10 mF. (Let's consider the output for frequencies of 1 Hz, 10 Hz, 100 Hz, 1 kHz, and 10 kHz.)

\begin{center}\includegraphics{figures/15.02.png}\end{center}
\section{An ideal differentiator}

For further reference, here is some more info on low pass filters that encorporates the corner frequency and other possible modifications:

Passive Low Pass: https://www.electronics-tutorials.ws/filter/filter\_2.html
Active Low Pass: https://www.electronics-tutorials.ws/filter/filter\_5.html

\textbf{The third combination of impedances we will consider is a capacitor, $C_{in}$, at the input and a resistor, $R_F$ in the feedback loop.}

This produces a voltage gain response of

\begin{equation}
	\frac{V_o}{V_{in}} = -\frac{Z_f}{Z_{in}} = -\frac{R_F}{1/sC_{in}} = -sR_FC_{in}
\end{equation}
Since multiplication by $s$ in the $s$-domain is equivalent to differentiation in the time-domain, then the time-domain output potential may be written as

\begin{equation}
	V_o = -R_fC_{in}\frac{d V_{in}(t)}{dt}
\end{equation}

Hence, in this case, the circuit has a transfer function which takes the time-derivative of the input voltage, scales it by the product of the resistor and capacitor, and inverts.

\subsection{A modification yielding a high pass filter}
At very low frequencies, the input capacitor begins to introduce some effects that me undesirable (e.g., at low frequencies, no current flows in the input branch.) If we were to add a resistor in series with the capacitor at the input to prevent the non-flowing of current we would produce our forth combination: \textbf{a capacitor, $C_{in}$, in series with a resistor, $R_{in}$ at the input and a resistor, $R_F$ in the feedback loop}.

\begin{equation}
	\frac{V_o}{V_{in}} = -\frac{Z_f}{Z_{in}} = -\frac{R_F}{R_{in} + (1/sC_F)} = -R_FC_{in}\cdot\frac{1}{1 + sR_1C_1}
\end{equation}

And let us consider its response with respect to frequency. To help matters, let us consider resistors of value 1 $\Omega$ and a capacitor of 10 mF. (Let's consider the output for frequencies of 1 Hz, 10 Hz, 100 Hz, 1 kHz, and 10 kHz.)

\begin{center}\includegraphics{figures/15.03.png}\end{center}

For further reference, here is some more info on high pass filters that encorporates the corner frequency and other possible modifications:

Passive High Pass: https://www.electronics-tutorials.ws/filter/filter\_3.html
Active High Pass: https://www.electronics-tutorials.ws/filter/filter\_6.html

\section{Non-inverting variants}
For each of these types of active filters, there are non-inverting versions which produce similar responses.

\begin{center}
	\includegraphics[width=0.5\textwidth]{figures/15.06.png}
\end{center}

Such a circuit will produce a gain transfer function of 
\begin{equation}
	\frac{V_o}{V_{in}} = 1 + \frac{R_2}{R_1}
\end{equation}
A more generalized version of this relationship may be thought of the ratio of the ``feedback'' impedance, $Z_F$, ($R_2$ in the figure above) to the ``ground'' impedance, $Z_G$ ($R_1$ in the figure above) plus 1. Hence,

\begin{equation}
	\frac{V_o}{V_{in}} = 1 + \frac{Z_F}{Z_G}
\end{equation}


\subsection{A non-inverting low pass filter}

Let us approach this case by viewing the two inputs of our operational amplifier as having undergone voltage division.

\begin{equation}
	\frac{V_o}{V_{in}} = \frac{V_+}{V_{in}}\cdot \frac{V_o}{V_+} = \left(\frac{1/sC}{R+ 1/sC}\right)\cdot\left(1+ \frac{R_F}{R_G}\right) = \left(\frac{1}{1 + sRC}\right)\cdot\left(1+ \frac{R_F}{R_G}\right) 
\end{equation}

\begin{center}
	\includegraphics[width=\textwidth]{figures/15.04.png}
\end{center}

\subsection{A non-inverting high pass filter}

Let us again approach this  by viewing the two inputs of our operational amplifier as having undergone voltage division.

\begin{equation}
	\frac{V_o}{V_{in}} = \frac{V_+}{V_{in}}\cdot \frac{V_o}{V_+} = \left(\frac{R}{R+ 1/sC}\right)\cdot\left(1+ \frac{R_F}{R_G}\right) = \left(\frac{sRC}{1 + sRC}\right)\cdot\left(1+ \frac{R_F}{R_G}\right) 
\end{equation}

\begin{center}
	\includegraphics[width=\textwidth]{figures/15.05.png}
\end{center}


\section{Combining active filters}


\section{Lecture 15 Board Pictures (3/14/19)}

\includegraphics[width=\textwidth]{figures/3-14-19_Fig.1.jpg}
\\
\includegraphics[width=\textwidth]{figures/3-14-19_Fig.2.jpg}
\\
\includegraphics[width=\textwidth]{figures/3-14-19_Fig.3.jpg}
\\
\includegraphics[width=\textwidth]{figures/3-14-19_Fig.3.5.jpg}
\\
\includegraphics[width=\textwidth]{figures/3-14-19_Fig.4.jpg}
\\
\includegraphics[width=\textwidth]{figures/3-14-19_Fig.5.jpg}
\\
\includegraphics[width=\textwidth]{figures/3-14-19_Fig.6.jpg}


\chapter{System response: IV. Feedback and moving forward}
03/28/2019
\minitoc
\newpage

\section{An introduction}
\begin{itemize}
	\item \textbf{A system} is a thing or a group of things whose behavior can be described.
	\item \textbf{A control system} is a thing or group of things whose behavior can be \textit{controlled}; in this sense a control system is able to alter the future state of a system based on a desired state.
	\item \textbf{Control theory} is the strategy to control systems by designing the systems or the inputs with which they are convolved in time. 
\end{itemize}

\section{To this point, an open loop}
\begin{itemize}
	\item Until now we have dealt with ``open-loop'' systems -- systems whose input does not depend on a system's output
	\item This is good for systems with consistent and generally predictable states (e.g., lights being on or off, the surface potential of your heart's dipole, the temperature of water during a shower)
	\item In the case of a car, for example, the input that is delivered by a driver to accelerate or decelerate the vehicle is the angle they push the gas pedal to. We can imagine then, a system which describes the relationship between the $\theta$ of that pedal input and convert it to the velocity of the car moving forward. Such a system is sometimes called a ``plant''
	\item In such open loop designs, the relationship between the input and the output is pretty straightforward: the system, convolved with the input, produces an output
\end{itemize}

\section{Onward, a closed loop}

\begin{itemize}
	\item We will now start to consider systems where at least some portion of the output is used to influence the system input, thereby changing the output, etc. 
	\item In this sense, the output is ``fedback'' to the input. 
	\item Moreover, this ``closes the loop'' between input and output and helps to hone in on the desired output.
\end{itemize}

\includegraphics[width = \textwidth]{figures/16.01.png}
\begin{itemize}
	\item \textbf{Reference}, $r(t)$ is the value we wish the system to ultimate produce. This might be an intensity of light, a temperature of a room, a velocity of a car, etc.
	\item \textbf{System output}, $y(t)$,  is the output the system produces in response to its system input (itself a function of the measured error's control strategy when passed to the controller)
	\item Measured error, $e(t)$, the difference between the reference value and the measured output value (which is the system output value that the sensor measures)
	\item \textbf{Controller}, $C$, uses the error between the reference and the output to change the system inputs, $u(t)$\footnote{Please note that the $u(t)$ here is not the same as the unit step function.} to the system, $P$
	\item \textbf{System input}, $u(t)$ is the input delivered directly to the system/plant
	\item \textbf{System}, $P$ is the system's transfer function. That is, this is the block in the diagram which represents the behavior of the system alone, absent any control strategies.
	\item \textbf{Sensor}, $F$, is used to measure the system output, $y(t)$ to compare the actual output to the desired output.
\end{itemize}


\section{Block diagram algebra}
\includegraphics[width = \textwidth]{figures/16.04.png}
\newpage
\includegraphics[width = \textwidth]{figures/16.05.png}


\section{A closed-loop transfer function}
\begin{center}
	\includegraphics[width = 0.5\textwidth]{figures/16.02.png}
\end{center}

\begin{enumerate}
	\item The output, $Y(s)=P(s)U(s)$
	\item The system input, $U(s)=C(s)E(s)$
	\item The (reference) error, $E(s)=R(s)-F(s)Y(s)$
\end{enumerate}

\begin{equation}
	Y(s)=\left({\frac {P(s)C(s)}{1+P(s)C(s)F(s)}}\right)R(s)=H(s)R(s)
\end{equation}

The expression
\begin{equation}
	H(s)={\frac {P(s)C(s)}{1+F(s)P(s)C(s)}}
\end{equation}
is referred to as the closed-loop transfer function. 
\begin{itemize}
	\item The \textbf{numerator} is the forward (open-loop) gain from r to y.
	\item The \textbf{denominator} is one plus the gain in going around the feedback loop, the ``loop gain''.
\end{itemize}


\section{PID feedback control}
A \textbf{proportional-integral-derivative controller} (``PID controller'') is a feedback-control technique widely used to control systems.

A PID controller continuously calculates an error value $e(t)$ as the difference between a desired setpoint and a measured process variable and applies a correction based on proportional, integral, and derivative terms. 

\begin{center}
	\includegraphics[width = \textwidth]{figures/16.03.png}
\end{center}

If u(t) is the control signal sent to the system, $y(t)$ is the measured output and $r(t)$ is the desired output, and $e(t)=r(t)-y(t)$ is the tracking error, a PID controller has the general form

\begin{equation}
	 u(t)=K_{P}e(t)+K_{I}\int e(\tau ){\text{d}}\tau +K_{D}{\frac {{\text{d}}e(t)}{{\text{d}}t}}
\end{equation}
where $K_P$, $K_I$, and $K_D$ are constants which scale the effects of each form of control. Generally these are chosen to achieve some effect (minimize overshoot, increase response time, minimize prolonged error, etc.).

Without much erffort we can see the the PID controller transfer function would be of the general form

\begin{equation}
	C(s)=\left(K_{P}+K_{I}{\frac {1}{s}}+K_{D}s\right)
\end{equation}

The goal here would be to use these terms to try to compensate for terms present in the plant. For instance, a system in which the output completely and flawless tracked the input would have a transfer function of $H(s) = 1$.

\section{Board Pictures (3/26/19)}

\includegraphics[width=\textwidth]{figures/3_26_1.jpg}
\\
\includegraphics[width=\textwidth]{figures/3_26_2.jpg}
\\
\includegraphics[width=\textwidth]{figures/3_26_3.jpg}
\\
\includegraphics[width=\textwidth]{figures/3_26_4.jpg}
\\
\includegraphics[width=\textwidth]{figures/3_26_5.jpg}
\\
\includegraphics[width=\textwidth]{figures/3_26_6.jpg}
\\
\includegraphics[width=\textwidth]{figures/3_26_7.jpg}
\\
\includegraphics[width=\textwidth]{figures/3_26_8.jpg}
\\
\includegraphics[width=\textwidth]{figures/3_26_9.jpg}
\\





\part{in Biomedical Engineering}

\chapter{Bioelectricity: I. Passive properties}
04/02/2019 
\minitoc
\newpage
\section{A quick review of some of our basic impedance knowledge}
\subsection{Impedance as a vector}
\begin{equation}
	Z = R + \jmath X = |Z|\angle \theta = |Z|e^{\jmath\theta}  
\end{equation}
\subsection{Impedance as admittance}
\begin{equation}
	\frac{1}{Z} = Y = G + jB
\end{equation}
\subsection{Impedance equivalents}
\begin{eqnarray}
	Z_{eq,R}= R \\
	Z_{eq,C} = \frac{1}{\jmath \omega C} \\
	Z_{eq,L} = \jmath \omega L
\end{eqnarray}
\subsection{Impedance of networks}
\begin{equation}
	Z_{eq,series} = Z_1 + Z_2 + Z_3 + ...
\end{equation}
\begin{equation}
	Z_{eq,parallel} = \left(\frac{1}{Z_1}+ \frac{1}{Z_2}+ \frac{1}{Z_3} + ...\right)^{-1}
\end{equation}

\section{A long derivation of a simple model}

Draw for yourself a simple picture a cell in a bath and while doing so ask yourself the deep questions regarding the universe. If you're looking for one to munch on consider this: what separates life from non-life?

The cell membrane.

And it acts like a capacitor. Cell membranes are made of a phospholipid bilayer, which, if you squint at it, is kind of like two conductive plates sandwiching a dielectric. (Close enough for government work.) 

On either side of that cell membrane is what we might call an \textit{intra}cellular and \textit{extra}cellular component. These, at their most basic level, must put up some sort of fight to an induced voltage, to scale the resistance of current. So let's say they are both resistors.

With this, we may begin our first modeling foray into the wonderful world of bioimpedance (a model whose use first began to see prominence in the early 1900s: 
\begin{equation}
	R_1+R_2\vert\vert C.
\end{equation}
This yields an initial model of
\begin{align}
	Z_{eq} &= Z_1 + \frac{Z_2\cdot Z_3}{Z_2 + Z_3} \\
	&= R_1 + \frac{R_2\cdot (1/\jmath \omega C)}{R_2 + (1/\jmath \omega C)} \\
	&= R_1 + \frac{R_2}{1 + \jmath \omega R_2 C}.
\end{align}
To remove that bottomr denominator we apply the complex conjugate
\begin{align}
	Z_{eq} &= R_1 + \frac{R_2}{1 + \jmath \omega R_2 C}\cdot\frac{1-\jmath\omega C}{1 - \jmath\omega C} \\
	&= R_1 + \frac{R_2}{1 - \jmath\omega R_2 C + \jmath\omega R_2 C - \jmath^2 \omega^2 R_2^2C^2} \\
	&= R_1 + \frac{R_2 - \jmath\omega R_2^2 C}{1 + \omega^2 R_2^2 C^2}.
\end{align}
At this point it might be useful to define a time constant, since it appears that the cell related terms (the cell membrane, $C$, and the intracellular component, $R_2$). Let's say
\begin{equation}
	\tau = R_2 C.
\end{equation}
Rewriting the previous equation:
\begin{align}
	Z_{eq} &= R_1 + \frac{R_2 - \jmath\omega \tau R_2}{1 + \omega^2 \tau^2}.
\end{align}
Separating the real and imaginary terms so that we might consider resistance and reactance separately:
\begin{align}
	Z_{eq} &= \left(R_1 + \frac{R_2}{1 + \omega^2 \tau^2}\right) - \jmath\left(\frac{\omega \tau R_2}{1 + \omega^2 \tau^2}\right).
\end{align}
Hence, 
\begin{align}
	R &= \left(R_1 + \frac{R_2}{1 + \omega^2 \tau^2}\right) \\
	X &= - \left(\frac{\omega \tau R_2}{1 + \omega^2 \tau^2}\right).
\end{align}

It's at a point like this that I like to consider a few extremes in a problem. The reason for that is because it is a very easy measurement to take with the equipment you will have. It will always be easy to send in a 0 and send in way more than you want. So, it being easy to find in lab, in the world, and in life, let use familiarize ourselves with their nature.

At very very very low frequencies ($\omega \rightarrow 0$), resistance tends to become equal to the sum of $R_1$ and $R_2$. At very very high frequencies ($\omega \rightarrow \infty$), resistance tends to become equal to $R_1$. From this I think we can all agree to the following:
\begin{align}
	R_0 &= R_1 + R_2 \\
	R_{\infty} &= R_1 \\
	R_1 &= R_{\infty} \\
	R_2 &= R_0 - R_{\infty}.
\end{align}

We can perform a similar analysis for reactance. At very very very low frequencies ($\omega \rightarrow 0$), reactance tends to become 0. At very very high frequencies ($\omega \rightarrow \infty$), reactance tends to become equal to 0. The combination of these two points represent the extremes of our resistance reactance plane. And they are measurable!

But before we measure, we often want to have in mind some notion of the result to be gotten. Let's consider, analytically, what the signal will end up looking like on the resistance-reactance plane.

Converting our equations from the model-based parameters to the measured ones, yields

\begin{align}
	R &= \left(R_{\infty} + \frac{R_0 - R_{\infty}}{1 + \omega^2 \tau^2}\right) \\
	X &= - \left(\frac{\omega \tau (R_0 - R_{\infty})}{1 + \omega^2 \tau^2}\right).
\end{align}
From this, we can attempt to solve the equations by looking for shared terms. In this case, let us try to isolate the $\omega\tau$ found in both the resistance and reactance terms.
\begin{align}
	R &= R_{\infty} + \frac{R_0 - R_{\infty}}{1 + \omega^2 \tau^2} \\
	R - R_{\infty} &= \frac{R_0 - R_{\infty}}{1 + \omega^2 \tau^2} \\
	(R - R_{\infty})(1 + \omega^2 \tau^2) &= R_0 - R_{\infty} \\
	1 + \omega^2 \tau^2 &= \frac{R_0 - R_{\infty}}{R - R_{\infty}} \\
	\omega^2 \tau^2 &= \frac{R_0 - R_{\infty}}{R - R_{\infty}}  - 1 \\
	\omega \tau &= \sqrt{\frac{R_0 - R_{\infty}}{R - R_{\infty}}  - 1}.
\end{align}
We can substitute this into our reactance equation
\begin{align}
	X &= - \left(\frac{\omega \tau (R_0 - R_{\infty})}{1 + (\omega \tau)^2}\right) \\
	X &= - \left(\frac{\left(\sqrt{\frac{R_0 - R_{\infty}}{R - R_{\infty}}  - 1}\right) (R_0 - R_{\infty})}{1 + \left(\sqrt{\frac{R_0 - R_{\infty}}{R - R_{\infty}}  - 1}\right)^2}\right) \\
	X &= - \left(\frac{\left(\sqrt{\frac{R_0 - R_{\infty}}{R - R_{\infty}}  - 1}\right) (R_0 - R_{\infty})}{1 + \frac{R_0 - R_{\infty}}{R - R_{\infty}}  - 1}\right) \\
	X &= - \left(\frac{\left(\sqrt{\frac{R_0 - R_{\infty}}{R - R_{\infty}}  - 1}\right) (R_0 - R_{\infty})}{\frac{R_0 - R_{\infty}}{R - R_{\infty}}}\right) \\
	X &= - \left(\left(\sqrt{\frac{R_0 - R_{\infty}}{R - R_{\infty}}  - 1}\right) (R - R_{\infty}) \right) \\
	X^2 &= \left(\frac{R_0 - R_{\infty}}{R - R_{\infty}}  - 1\right) (R - R_{\infty})^2 \\
	X^2 &= \left(\frac{R_0 - R_{\infty} - R + R_{\infty}}{R - R_{\infty}}\right) (R - R_{\infty})^2 \\
	X^2 &= \left(\frac{R_0 - R }{R - R_{\infty}}\right) (R - R_{\infty})^2 \\
	X^2 &= \left(R_0 - R \right) (R - R_{\infty}) \\
	X^2 &= R_0R - R_0R_{\infty} - R^2 + R R_{\infty}.
\end{align}
It is with the appearance of two squared axes terms ($X^2$ and $R^2$) in that equation that put me in a mind go go looking for circles. With that focus, I will endeavor to put this into a format consistent with a semi-circle.

\begin{align}
	X^2 &= R_0R - R_0R_{\infty} - R^2 + R R_{\infty} \\
	X^2 + R^2- R_0R - R R_{\infty} &=  - R_0R_{\infty} \\
	X^2 + R^2- R(R_0 + R_{\infty}) &=  - R_0R_{\infty} \\ 
	X^2 + R^2- R(R_0 + R_{\infty}) + \left(\frac{R_0 + R_{\infty}}{2}\right)^2 &=  - R_0R_{\infty} + \left(\frac{R_0 + R_{\infty}}{2}\right)^2 \\
	X^2 + \left(R - \frac{R_0 + R_{\infty}}{2}\right)^2 &=  - R_0R_{\infty} + \frac{R_0^2}{4} + \frac{2R_0R_{\infty}}{4} + \frac{R_{\infty}^2}{4} \\
	X^2 + \left(R - \frac{R_0 + R_{\infty}}{2}\right)^2 &= \frac{R_0^2}{4} - \frac{R_0R_{\infty}}{2} + \frac{R_{\infty}^2}{4} \\
	X^2 + \left(R - \frac{R_0 + R_{\infty}}{2}\right)^2 &= \left(\frac{R_0 - R_{\infty}}{2}\right)^2.
\end{align}
This final result demonstrates that we would expect the impedance vector to carve out a semi-circle in the resistance-reactance plane, centered at $((R_0 + R_{\infty})/2, 0)$ and with a radius of $(R_0 - R_{\infty})/2$.

Such a tracing on the resistance-reactance plane is sometimes called an Argand diagram, a Wessel diagram, or a Cole-Cole plot. The latter is a woefully entrenched incorrect way of describing it and the former two describe a complex plane more generally. I prefer to simply call it the result on the resistance-reactance plane.

\section{Implications of the model}
\subsection{What happens if $R_1$ goes up?}
In the case in which there is less conductive material in our extracellular space, for instance, when there is less water within our blood vessels (i.e., when we become dehydrated), what happens?

Recall, $R_0 = R_1 + R_2$ and $R_{\infty} = R_1$, so it would shift the curve rightward, elongating our impedance vector and making our phase angle a little more shallow.

\subsection{What happens if $R_1$ goes down?}
In this case there is more conductive fluid in our extracellular space, as is what happens in many renal disorders, such as end-stage renal disease, one of the final consequences of diabetes. This shifts the curve leftward. This also makes our phase angle a more sensitive measure and thus if we were to try to characterize the disorder, we could use it. (As is the case of those working with ``bioreactance''.)

\subsection{What happens if $R_2$ goes up?}
$R_2$ may increase, for example, when cells are lysing/dying, being invaded by parasites, etc. Such an increase would expands the circle, but leaves the leftmost point the same, which makes sense since the extracellular has not been altered. Thus as cells begin to die, their impedance goes up. This is a phenomenon that is seen in electrosurgery and is one of the challenges to ensuring an optimal amount of current delivery to heat the tissue.


\section{Now, that's what I call a darn good question}
In class, I was thoroughly stumped. I was asked, ``why is the reactance negative?'' We had just derived a basic bioimpedance model and the reactance, as you may well know dear reader, is negative, such that the impedance vector carves out a shifted semi-circle along the \textit{negative} reactance portion of the resistance-reactance plane.\footnote{Yes, this would be a model without a constant phase element.} There I stood dumbfounded for eternity until a student pointed out that there was a negative sign in our math from the get go! 

It answered the question, but not thoroughly enough for my like. I report below a more thorough attempt at an answer.

The impedance of a capacitor is
\begin{equation}
	Z_C = \frac{1}{\jmath \omega C}.
\end{equation}

Let's play with some algebra. Recall that we define $\jmath = \sqrt{-1}$. I think that you would agree that it's fair to say that the following is true
\begin{equation}
	\frac{1}{\jmath} = \frac{1}{\jmath}.
\end{equation}
No tricks so far. Let's multiply the right-hand-side by $\frac{\jmath}{\jmath}$, such that

\begin{equation}
	\frac{1}{\jmath} = \frac{1}{\jmath}\cdot \frac{\jmath}{\jmath}.
\end{equation}
This I think you'll agree leads to the following logic:
\begin{align}
	\frac{1}{\jmath} &= \frac{1}{\jmath} \\
	&= \frac{1}{\jmath}\cdot \frac{\jmath}{\jmath} \\
	&= \frac{\jmath}{\jmath^2} \\
	&= \frac{\jmath}{-1} \\
	&= -\jmath \\
	\frac{1}{\jmath} &=-\jmath.
\end{align}
Thus,
\begin{equation}
	Z_C = \frac{1}{\jmath}\cdot \frac{1}{\omega C} = - \frac{1}{\omega C}.
\end{equation}

Therefore, the \textit{reactance that is produced by capacitance} will be negative. Conversely, reactance produced by inductance will be positive. I leave it to the interested reader to prove this to themselves.

Really, what the ``$-\jmath$'' result represents in our math is a 90 degree phase shift between the capacitor's voltage and its current. That is, if one were to try to force a current into a capacitor, a potential would develop in a manner that in which its intensity is 90 degrees out of alignment. Put simply, the current "leads" the voltage (by about 90 degrees)\footnote{Again, we are not considering other sorts of ``constant phase elements'', though the interested reader is encouraged to consider their consequences.}.

\section{A summary of ``The theory and fundamentals of bioimpedance analysis in clinical status monitoring and diagnosis of diseases''}

\subsection{Article details}
“The Theory and Fundamentals of Bioimpedance Analysis in Clinical Status Monitoring and Diagnosis of Diseases”. Source: https://www.ncbi.nlm.nih.gov/pmc/articles/PMC4118362/ 

\subsection{Advantages of bioimpedance measurements}
\begin{itemize}
	\item low cost
	\item portable systems
	\item wide range of utilizations, non-invasive
\end{itemize}


\subsection{Definition of Bioimpedance} 
The ability of biological tissue to impede electric current

\subsection{Active vs. Passive Response}
\begin{itemize}
	\item \textbf{Active:} Biological tissues provoke electricity on their own from ionic activities inside cells (for example, electrocardiogram signals from the heart)
	\item \textbf{Passive:} Biological tissues are stimulated through an external electrical current source; an elicited reaction to a stimulus.
\end{itemize}


\subsection{Methods for measuring whole-body impedance using different placements and types of electrodes}
\begin{enumerate}
	\item Hand to foot method: gel-filled electrodes made to minimize gap 
	\item Foot to foot method: pressure-contact foot-pad electrode
	\item Hand to hand method: handheld impedance meter
	\item Segmental bioimpedance analysis: detects the fluctuation in extracellular fluid due to differences in posture. This method measures the body in 5 segments treated as cylinders (both arm and leg limbs plus the torso). 
\end{enumerate}

\subsection{Using Bioimpedance to measure human health status} estimating the hydration status of individuals using height indexed resistance and reactance data (an R-Xc graph) from bioimpedance measurements. “The data allows creation of 50\%, 75\%, and 95\% tolerance ellipses that determine increasing and decreasing body mass if the minor vector falls in the left and right half of the 50\% ellipse, along with increasing and decreasing hydration ratio if the major vector falls in the lower and upper half of the 50\% ellipse.”

\subsection{Clinical Bio-impedance} Read about how our own professor developed a technique called ``DRIVE'', also known as dynamic respiratory impedance volume evaluation which ``uses the bioimpedance of the upper limb to predict shifts in blood volume in response to cardiac and respiratory phenomenon'' : https://belmont.bme.umich.edu/research/ 

Using this technique, a wearable sensor smaller than a credit card was designed to measure heart rate, respiratory rate, and body temperature and volume status. 

\subsection{Body Composition} Impedance used to measure Fat mass and Fat-free mass, as well as create body fluids estimations. 

Your body is made of fat mass and fat free mass. Fat free mass is composed of bone mass and body cell mass, of which in body cell mass is composed of 73.2\% water and the other 26.8\% proteins. Of the total body water weight, 29\% is made from extracellular fluids and 44\% is made of intracellular fluids.

\subsection{Potential future applications} Measuring the bio impedance of the blood and using a control feedback loop to to supply the correct amount of blood back during blood loss. Could potentially be used for dialysis to take the correct amount of waste out of the body.

\section{Board Pictures (4/2/19)}

\includegraphics[width=\textwidth]{figures/4-2-19_Fig.1.jpg}
\\
\includegraphics[width=\textwidth]{figures/4-2-19_Fig.2.jpg}
\\
\includegraphics[width=\textwidth]{figures/4-2-19_Fig.3.jpg}
\\
\includegraphics[width=\textwidth]{figures/4-2-19_Fig.4.jpg}
\\

\chapter{Bioelectricity: II. Active properties}
04/04/2019
\minitoc
\newpage

\section{How electrical potential arises in a cell}
As we know, cells are far from stagnant. There is a constant movement of molecules and ions in and out of the cell depending on that cell's needs. When an uneven distribution of charges occur inside and outside the cell, an ion gradient is formed, which in turn creates a potential across the membrane of the cell. 


\subsection{Sodium-potassium pump}
\begin{itemize}
	\item Active element
	\item Only pump we have that consumes energy in the form of ATP (from food)
	\item The pump utilizes an ATPase
	\item Pumps 3 sodium ions out of the cell and 2 potassium ions into the cell
	\begin{itemize}
		\item This aids in creating a net negative charge within the cell and a net postive charge outside the cell. This charge separation creates a negative transmembrane potential.
	\end{itemize}
\end{itemize}

The purpose of the sodium-potassium pump in many cells is to transport these ions across the membrane in order to reach some kind of threshhold required by the cell to move other larger particles in and out. This is an active pump, meaning it requires the input of energy to function (i.e. food). This pump accomplishes the movement of 3 Na+ ions out of the cell for every 2 K+ ions moved in, creating an overall negative charge inside the cell and a relatively positive one outside, creating a potential across the membrane. 

\subsection{Membrane capacitance}
\begin{itemize}
	\item Passive element
	\item approximately 1 $\mu$F/cm$^2$
	\item High capacitance is needed for energy storage across membranes
	\item Determines temporal voltage response (Activation)
\end{itemize}

The cell membrane itself has a capacitance of about 1 microFarad per centimeter squared. This is a relatively high capacitance, required in order to store energy across the membrane. The capacitance determines the time response of voltage across the cell membrane. 

\subsection{Potassium ion channel}
\begin{itemize}
	\item Only allows potassium through it
	\item Opens/Closes dynamically and stochastically (random with respect to time)
	\item Contains 4 gating elements (all N gates)
	\begin {itemize}
		\item Open when there is a positive membrane potential
	\end {itemize}
\end{itemize}

This channel, as you may guess, transports K+ ions. It has dynamically opens and closes stochastically, which in layman's terms means at random intervals of time. This pump contains four "N gates" that control the opening and closing of the channel, typically opening at times when the transmembrane potential is positive. 

\subsection{Sodium ion channel}
\begin{itemize}
	\item Only allows sodium through it
	\item Mostly closed at rest
	\item Opens/Closes based on time
	\item Contains 4 gating elements (3 M and 1 H gate)
\end{itemize}

Another surprise, the sodium ion channel transports sodium ions across the cell membrane. In their resting and inactive positions, they allow no ions through. This state is reached over time, after a trigger causes the gate to become activated. It is only in their activated state that Na+ is allowed through the channel. There are again four gates, 3 ``M gates'' and 1 ``H gate''. 

\subsection{Leakage channels}
\begin{itemize}
	\item Small channels that allow the passage of small particles
	\begin{itemize}
		\item Typically chloride
	\end{itemize}
	\item Stabilizes voltage changes
\end{itemize}

This subsection of channels includes all other channels that transport small charged particles across the membrane (such as chloride, for example). It is thanks to these "others" that quick changes in potential between the inside and outside of a cell can be readily stabilized. 
%\includegraphics[width=\textwidth]{figures/lecture18.1.png}





\section{Determining resting potential from a Kirchhoffian perspective}


\section{Electrocardiography}
Each of the previously discussed channels separate ions across the membrane, and have a gradient across them, which in turn creates an ionic current. The membrane current as a whole can be described as:
\begin {equation}
	I_m = I_k + I_Na + I_leakage + I_c + I_p
\end {equation}
Each of these individual terms are of the form:
\begin {equation}
	I_k = g_k * (V_m - E_k)
\end {equation}
where $g_k$ is the conductance of potassium, $V_m$ is the membrane voltage at rest, and $E_k$ is the Nernst potential. In order to find the membrane potential using all of the channels known to be present in the cell, use the following equation:
\begin {equation}
	V_m = \frac{g_k * E_k + g_Na * E_Na + g_leakage * E_leakage...} {\ g_k + g_Na + g_leakage ...}
\end {equation}
These equations can be applied to the human heart. Cardiomyocyte contractions are caused by the changes in membrane potential. Due to the uneven distribution of charges, a dipole arises, and we are able to measure this via leads and Einthoven's triangle.

\subsection{Leads and Einthoven's triangle}
Einthoven's triangle is an imaginary triangle made with three limb leads, the right arm, left arm, and the left leg. It forms an inverted triangle centered around the heart. 
\includegraphics[width=\textwidth]{figures/lecture17_triangle.png}
\begin{itemize}
	\item Lead I goes from right shoulder to left shoulder with negative electrode on right shoulder and positive on the left one.
	Lead I = left arm - right arm
	\item Lead II goes from the right arm and reaches to the leg on the left side with negative electrode on the shoulder and positive electrode on the leg.
	Lead II = left leg - right arm 
	\item Lead III goes from the left shoulder, having a negative electrode, to the left leg with a positive electrode.
	Lead III = left leg - left arm
\end{itemize}
Only two leads are needed to characterize a vector.
The heart at the center will produce a zero potential when the voltages are summed. According to Kirchoff's Voltage Law, I+III-II = 0.
\subsection{The P-wave}
The P-wave indicates atrial depolarization. This occurs when the sinoatrial node initiates an action potential over the atria. The Na+ channels in atrial cells open and become depolarized. The P-wave becomes visible on lead II.

\subsection{The Q-wave}
The Q-wave occurs next, as the impulse travels through the arterioventricular bundle and its branches. The impulse goes to the purkinje fibers. It indicates ventricular depolarization, which travels "up" the heart, making the Q-wave negative and create a pulse downwards.

\subsection{The R-wave}
The R-wave is the largest wave in the complex. It represents the electrical stimulus as it passes through the main portion of the ventricular walls. This occurs when impulse spreads to the contricle fibers of the ventricles, and potassium starts to flow into the ventricle cells. This is the first phase of action potential.

\subsection{The S-wave}
The S-wave is the downward pulse that follows the R-wave, but may not be present in all ECG leads on a given patient. The S-wave represents depolarization in the purkinje fibers, and the ventricles contract. This is the second and third phase of action potential. The S-wave is visible on lead II.

\subsection{The T-wave}
The T-wave represents the repolarization of the ventricles. It occurs when ventricles contract, like for the S-wave, and the impulse spreads through the heart. This causes a three-dimensional shift of the heart's dipole. The T-wave is also visible on lead II.

\includegraphics[width=\textwidth]{figures/lecture17_ecg.png}

\section{Basic hardware engineer's concerns}
\begin{enumerate}
	\item Sensing electrodes
	\begin{itemize}
		\item Interface between subject and the system (usually made with Ag/AgCl- Gel coupled materials)
	\end{itemize}
	\item Protection circuit
	\begin{itemize}
		\item Protects the device from high voltage
		Example: Voltage follower
	\end{itemize}
	\item Lead selector
	\begin{itemize}
		\item Compares different sets of leads
		\item Can be used to scrutinize a particular lead
	\end{itemize}
	\item Calibration Signal
	\begin{itemize}
		\item Send in a small impulse to figure out the transfer function of the system
		\item Transfer function may vary with time, so this is essential to calibrate the system
	\end{itemize}
	\item Preamplifier
	\begin{itemize}
		\item Amplifies ECG signal
		\item High input impedance
		\item High common-mode-rejection ratio
	\end{itemize}
	\item Driver amplifier
	\begin{itemize}
		\item Filters ECG signal using a bandpass filter (to get rid of low frequency drift and high frequency noise)
	\end{itemize}
	\item Driven-reference circuit
	\begin{itemize}
		\item Gives a reference point (ground) on the human body (usually the right leg is used)
	\end{itemize}
	\item Isolation circuit
	\begin{itemize}
		\item Protects the patient from dangerous currents
		\item Galvanic Isolation used in this class
	\end{itemize}
	\item Power supply
\end{enumerate}
\section{Basic software engineer's concerns}
\begin{enumerate}
	\item Analog digital conversion
	\item Memory
	\item Microcomputer/microcontroller
	\item Control program
	\item ECG analysis program
	\item Recorder
	\item Display
	\item User input
\end{enumerate}

\section{Electrophysiology}
Electrophysiology is a branch of physiology that deals with electical potentials of the body.
There are many specific branches of electrophysiology, one being electrocardiography. Four other common types are
\begin{enumerate}
	\item Electroencephalography(EEG), which deals with the brain
	\item Electrooculography(EOG), which deals with the eyes
	\item Electromyography(EMG), which deals with skeletal muscle
	\item Electrogastrography(EGG), which deals with the stomach and intestines
\end{enumerate}

\subsection{Electroencephalography(EEG)}
\begin{itemize}
	\item Electroencephalography is the monitoring of the activity of the brain by measuring the potential at many points across the scalp.
	\item An International 10-20 system is used to keep track of the location of each point that the potential is being measured at. These points are markers for where the electrodes are to be placed.
	\item A simplified represenatation of this system has electrodes on the cerebral cortex at each major lobe.
	\begin{enumerate}
	\item Frontal
	\item Temporal
	\item Parietal
	\item Occipital
	\end{enumerate}
	\item Each EEG contains a lot of information, as it is a sum total of all brain firings occuring .
	\item EEGs however do pickup electrical signals of other sources, so EEGs have to be corrected in order to get the separation of releveant and non-relevant signals. An example of this in a real-life situation is the cocktail party effect, where with enough focus and concentration, it is possible to filter out a bunch of stimuli, which would be different conversations, and focus in a specific stimulus, which would be the conversation a person would be engaging in. 
\end{itemize}
	
\subsection{Electrooculography(EOG)}
\begin{itemize}
	\item Electrooculography deals with tracking the dipole formed by assessing the cornco-retinal standing potential between front and back of the eye. At the back of the eye is the negative side, and the front of the eye is the positive side. By looking at the angle of the dipole and direction of potential created, we can determine whether a person is looking forward, up or down.
	\item This information can then be used for wheelchairs for paralyzed patients or text-to-speech programs.
\end{itemize}

\subsection{Electromyography(EMG)}
\begin{itemize}
	\item Electromyography is the measuring of electrical activity of skeletal muscles. EMGs are typically used for diagnosising abnormalities or for practical uses, like in athletics.
	\item The muscles in our body are connected to nerves within our brain via alpha-motor neurons, which are large neurons that are located in the brainstem and the spinal cord.
	\item The signals from these neurons cause contractions, which is a huge signal that is hard to measure and quantify. 
	\item However, each individual motor unit within the neuron has their own action potential trait, and the accumulation of these individual potentials becomes the EMG signal.
\end{itemize}

\subsection{Electrogastrography(EGG)}
\begin{itemize}
	\item Electrogastrography is used to determine the spread of potentials through out the GI tract's (mouth to rectum) muscles during rest and during contractions.
	\item Motility in the GI tract results from coordinated contractions of smooth muscle.
	\item The electrical signals comprise of action potentials and slow waves, which are a rhythmic movement of the smooth muscle in the GI tract.
	\item Common Cycles Per Minutes for Slow Waves for Various Smooth Muscles in the GI Tract
	\begin{enumerate}
		\item Stomach: 3 CPM
		\item Duodenum: 12 CPM
		\item Jejunum: 11 CPM
		\item Ileum: 8 CPM
		\item Rectum: 17 CPM
	\end{enumerate}
\end{itemize}

\section{Modeling a neuron}
\includepdf[pages=-]{figures/elijah-neuron.pdf}


\chapter{Digital circuits: I. Logic and digital components}
04/09/2019 
\minitoc
\newpage

\section{Digital components, ``logic gates''}
\subsection{NOT}
\begin{center}
	\includegraphics{figures/20.01.png}
\end{center}
\begin{equation}
	\overline{A} \text{ or } \neg A
\end{equation}
\begin{center}
	\begin{tabular}{c c}
		\textbf{Input} & \textbf{Output} \\
		A & $\neg$ A \\
		1 & 0 \\
		0 & 1\\
	\end{tabular}
\end{center}

\subsection{AND}
\begin{center}
	\includegraphics{figures/20.02.png}
\end{center}
\begin{equation}
	A\cdot B
\end{equation}
\begin{center}
	\begin{tabular}{c c c}
		\textbf{Input 1} &  \textbf{Input 2} & \textbf{Output} \\
		A & B & A AND B \\
		0 & 0 & 0 \\
		0 & 1 & 0 \\
		1 & 0 & 0 \\
		1 & 1 & 1
	\end{tabular}
\end{center}

\newpage

\subsection{OR}
\begin{center}
	\includegraphics{figures/20.03.png}
\end{center}
\begin{equation}
	A + B
\end{equation}
\begin{center}
	\begin{tabular}{c c c}
		\textbf{Input 1} &  \textbf{Input 2} & \textbf{Output} \\
		A & B & A OR B \\
		0 & 0 & 0 \\
		0 & 1 & 1 \\
		1 & 0 & 1 \\
		1 & 1 & 1
	\end{tabular}
\end{center}

\subsection{NAND}
\begin{center}
	\includegraphics{figures/20.04.png}
\end{center}
\begin{equation}
	\overline{A\cdot B} \text{ or } A \uparrow B
\end{equation}
\begin{center}
	\begin{tabular}{c c c}
		\textbf{Input 1} &  \textbf{Input 2} & \textbf{Output} \\
		A & B & A NAN B \\
		0 & 0 & 1 \\
		0 & 1 & 1 \\
		1 & 0 & 1 \\
		1 & 1 & 0
	\end{tabular}
\end{center}

\newpage

\subsection{NOR}
\begin{center}
	\includegraphics{figures/20.05.png}
\end{center}
\begin{equation}
	\overline{A+B} \text{ or } A \downarrow B
\end{equation}
\begin{center}
	\begin{tabular}{c c c}
		\textbf{Input 1} &  \textbf{Input 2} & \textbf{Output} \\
		A & B & A NOR B \\
		0 & 0 & 1 \\
		0 & 1 & 0 \\
		1 & 0 & 0 \\
		1 & 1 & 0
	\end{tabular}
\end{center}

\subsection{XOR}
\begin{center}
	\includegraphics{figures/20.06.png}
\end{center}
\begin{equation}
	A \oplus B
\end{equation}
\begin{center}
	\begin{tabular}{c c c}
		\textbf{Input 1} &  \textbf{Input 2} & \textbf{Output} \\
		A & B & A XOR B \\
		0 & 0 & 0 \\
		0 & 1 & 1 \\
		1 & 0 & 1 \\
		1 & 1 & 0
	\end{tabular}
\end{center}

\newpage

\subsection{XNOR}
\begin{center}
	\includegraphics{figures/20.07.png}
\end{center}
\begin{equation}
	\overline{A\oplus B} \text{ or} A \odot B
\end{equation}
\begin{center}
	\begin{tabular}{c c c}
		\textbf{Input 1} &  \textbf{Input 2} & \textbf{Output} \\
		A & B & A OR B \\
		0 & 0 & 1 \\
		0 & 1 & 0 \\
		1 & 0 & 0 \\
		1 & 1 & 1
	\end{tabular}
\end{center}

\section{Representing logic graphically}
\begin{center}
	\includegraphics[width = \textwidth]{figures/20.08.png}
\end{center}

\section{Functional completeness}
Or, how to make everything from NORs.
\subsection{NOR from NOR}

\includegraphics[width=\textwidth]{figures/NOR.jpg}

\subsection{NOT from NOR}

\includegraphics[width=\textwidth]{figures/NOT.jpg}

\subsection{AND from NOR}

\includegraphics[width=\textwidth]{figures/AND.jpg}

\subsection{OR from NOR}

\includegraphics[width=\textwidth]{figures/OR.jpg}

\subsection{NAND from NOR}


\subsection{XOR from NOR}

\includegraphics[width=\textwidth]{figures/XOR.jpg}

\subsection{XNOR from NOR}


\includegraphics[width=\textwidth]{figures/XNOR.jpg}

For examples and more explanation visit : http://www.ee.surrey.ac.uk/Projects/CAL/digital-logic/gatesfunc/

\newpage
\section{Digital components}
\subsection{Why silicon?}
\begin{itemize}
	\item Given the lattice structure of silicon atoms, a silicon atom can bond to its four (4) neighbors.
	\item Silicon has four (4) electrons in its valence shell.
	\item That enables the silicon atom to bind itself four (4) partners, covalently.  
	\item If bound covalently to those neighbors, each of the silicon atom's electrons will exist in its valence band.
	\item In this case, the only electrons that can move around are free electrons (i.e., electrons that have absorbed some energy) and they would have a hard time at that. It therefore follows that crystalline silicon (typically) has low conductivity. \textit{Pure silicon acts nearly as an insulator}.
	\item \textbf{Doping} is used to increase the performance of silicon, by mixing in an ``impurity''. These impurities interrupt the crystal and make conduction possible. Hence the term ``semiconductors''.
\end{itemize}

\begin{center}
	\includegraphics[width=0.5\textwidth]{figures/19.01.png}
\end{center}


\subsection{Doping}
There are two types of doping:

\subsubsection{N-type doping}
Let’s say you mix in a little bit of phosphorous with five (5) valence electrons. Now you’ve got one electron free to move around the system. Only a small amount ($<<1$\%) of impurity (e.g., phosphorus, arsenic) is needed to allow electric current to flow through the silicon. \textbf{N-type: electrons (a ``1'') can move around}

\subsubsection{P-type doping}
Let's now mix in a little boron or gallium (each having three (3) valence electrons. When mixed into the silicon lattice, they form ``holes'' where silicon has nothing to bond with. Holes have the effect of a positive charge and help conduct current by accepting electrons from a neighbor (thus moving the hole). \textbf{P-type: hole (a ``0'') can move around}



\subsection{Diodes}
\begin{itemize}
	\item What would happen if we combined n-type silicon with p-type silicon 
	\item We'd get a diode! 
	\item How does a diode work?
	\begin{enumerate}
		\item The abundant electrons present on the n-type semiconductor will tend to migrate over the boundary into the hole rich p-type semiconductor. 
		\subitem \textbf{(1s from n) $\rightarrow$ (0s in p)}
		\item This will tend to make the p-side slightly more negative and it will make the n-type slightly more positive. 
		\subitem \textbf{(p $\downarrow$, n$\uparrow$)}
		\item Given sufficient time, a potential gradient will form sufficiently large to prevent any further natural migration of electrons 
	\end{enumerate}
\end{itemize}





\subsection{Transistors}

\begin{itemize}
	\item The electronics inn your pockets/bags contain \textit{billions} of transistors
	\item Act as a switch with no moving parts
	\item Amplifies weak signals
	\item In its most basic sense, a transistor is just two of these diodes put back to back. (no matter how you flip the power source, looks the same.) This is known as a \textbf{bipolar junction transistor}.
	\item Different kinds of transistor (e.g., things like field-effect transistors) have different properties. 
\end{itemize}

\subsection{Terminology} 
A transistor consists of three parts: A Base, Collector, and Emittor

\begin{itemize}
	\item The \textbf{Base} is the lead responsible for activating the transistor. 
	\item The \textbf{Collector} is the positive lead. 
	\item The \textbf{Emittor} is the negative Lead. 
\end{itemize}

In an NPN transistor, the current will enter one side of the N-type region, which is the Collector. The Current will travel through the Base, which is the P-type region, and then exit through the other N-type region, the Emittor. You can remember these terms by imagining that the collector is "collecting" the current into it; then, due to the doped P-type next to it, it will flow through the NPN and out the other side, "emitting" the current out of it. The base is the middle part of the transistor, and is P-type in the NPN transistor or N-type in the PNP transistor.  


\subsubsection{NPN}
Let's begin by sandwiching a p-type semiconnductor between two n-types.
\begin{enumerate}
	\item Let's connect a power supply through the whole thing
	\item If we do that we’ll have a reverse biasing circuit in either direction.
	\item Let's also place another power supply between an N and a P.
	\item This gives us a forward biasing diode! (positive to positive)
	\item This will move electrons from the n region to the holes in the p region, this will cause some of the occupied holes’ electrons to be moved and circulate.
	\item Not all electrons will flow through the one circuit, now they will also allow for the migration from the other (perhaps much larger) circuit 
	\item This gives us three important components that I can never really truly remember unless I’m actually using the thing. But they are thus:
	\subitem You have a \textbf{small base current, B}, that can be amplified to a high \textbf{collector current, C}, by an \textbf{emitter, E}
	\subitem As a circuit element that looks like (NPN)
 	\begin{center}
		\includegraphics{figures/20.19.png}
	\end{center}
	\subitem You have a small base amplified to produce a large collector and emitter current. This will occur when there is a positive potential difference measured between the base to the emitter.
\end{enumerate}
\begin{center}
	\includegraphics[width = \textwidth]{figures/20.20.png}
\end{center}


\subsubsection{PNP}
\begin{enumerate}
	\item An n-type layer sandwiched between two layers of p-type material.
	\item A small current leaving the base is amplified in the collector output.
	\item A PNP transistor is on when the base is pulled low relative to the emitter
	\item Holes are injected into the base as the minority carriers
	\item Circuitly
	\begin{center}
		\includegraphics{figures/20.21.png}
	\end{center}
	\item The arrows in the symbols indicate the PN junction between the base and the emitter
	\item When the transistor is in the ``forward saturated'' / ``active'' / ``forward active'' mode, the arrow tells you which way the current is or very well ought to be going.
\end{enumerate}




\section{Board Pictures (4/9/19)}
I also included my notes as some of the board pictures are not as clear as I had hoped.

\includegraphics[width=\textwidth]{figures/4_9_19_board1.jpg}
\\
\includegraphics[width=\textwidth]{figures/4_9_19_board2.jpg}
\\
\includegraphics[width=\textwidth]{figures/4_9_19_board3.jpg}
\\
\includegraphics[width=\textwidth]{figures/4_9_19_board4.jpg}
\\
\includegraphics[width=\textwidth]{figures/4_9_19_board5.jpg}
\\
\includegraphics[width=\textwidth]{figures/4_9_19_board6.jpg}
\\
\includegraphics[width=\textwidth]{figures/notes1.jpg}
\\
\includegraphics[width=\textwidth]{figures/notes2.jpg}
\\
\includegraphics[width=\textwidth]{figures/notes3.jpg}
\\
\includegraphics[width=\textwidth]{figures/notes4.jpg}
\\
\includegraphics[width=\textwidth]{figures/notes5.jpg}
\\
\includegraphics[width=\textwidth]{figures/notes6.jpg}
\\



\newpage
\section{Worksheet}
In your extensive travels in this world you stumble unto a land inhabited only by knights and knaves. Knights always tell the truth and knaves always lie.
\subsection{Problem 1, Pedro and Apollonia}
You come across three inhabitants and ask the first, Pedro, “Are you a knight or a knave?” Pedro answers, but so quietly you can’t hear him. You ask Apollonia “What did he say?” to which she responds “Pedro said he was a knave.” Upon hearing this, Peter piped up and said “Don’t believe that; it’s a lie.” Is Peter a knight or a knave? [Further, is it possible to know what Pedro is?]




\subsection{Problem 2, Roger and Oedipa} 
Shortly after that you meet two inhabitants, Roger Mexico and Oedipa Maas. Roger claims, “Both of us are knaves.” What are Roger and Oedipa?


 
\subsection{Problem 3, Yes and No} 
Suppose you’ve heard a rumor that there’s gold buried nearby. You meet a local and want to know whether there really is gold in them thar hills, but you don’t know whether the person is a knight or a knave. If you are only allowed to ask only one question answerable by ``yes'' or ``no'', what do you ask?


\subsection{Problem 4, Lisa and Louise}
Lisa and Louise are twins indistinguishable in appearance. One always lies, the other always tells the truth. You don’t know which is which. You meet one of them and may ask one question to determine which twin is truth. What do you ask and what does it tell you?

\newpage

\subsection{Problem 5, NOT from NOR}
Make a NOT gate from one or more NOR gates.
\subsection{Problem 6, AND from NOR}
Make an AND gate from one or more NOR gates.
\subsection{Problem 7, OR from NOR}
Make an OR gate from one or more NOR gates.
\subsection{Problem 8, NAND from NOR}
Make a NAND gate from one or more NOR gates.


\chapter{Digital circuits: II. Discretization and acquisition}
04/11/2019
Here are some videos to help get started on building circuits:
(1) How to use a Breadboard: https://youtu.be/6WReFkfrUIk
This video describes how a breadboard is connected and what it means to connect circuit elements to the different holes in each row. It also builds a simple LED circuit.

(2) RC Filters: https://youtu.be/kc02HWprTo8
This video goes through the math behind RC filters and shows the resulting behavior of these filters on an oscilloscope, briefly showing the design of these circuits on a breadboard.

\section{Getting the Band(pass) Back Together}
A tutorial created by Isabel Holtan and Ryan Schildcrout.

\subsection{Materials}
Before you begin, you will need to gather the following materials:

\begin{itemize}
	\item 2 resistors
	\item 3 alligator clips
	\item an LM741 OpAmp
	\item 7 thick wires and 5 smaller wires
	\item a bread board
	\item 2 capacitors
	\item 2 output splitters
\end{itemize}
\includegraphics{figures/ex.01.png}
\includegraphics{figures/ex.02.png}
\includegraphics{figures/ex.03.png}
\includegraphics{figures/ex.04.png}
\includegraphics{figures/ex.05.png}
\includegraphics{figures/ex.06.png}
\includegraphics{figures/ex.07.png}

\subsection{Selecting Capacitor and Resistor Values}
In order to make an inverting bandpass filter, you should select corner frequencies that are higher than 1 Hz, ideally, in order to see an interesting response on the oscilloscope. This is because when you look at frequencies this low, they are very slow.  For this filter, both the low pass and high pass sections require a capacitor and a resistor. The equation for the corner frequency for this circuit is 1/RC. You have a lot of room to play with these corner frequencies, but the lower and upper values represent the bandwidth of your filter. 

\subsection{Circuits, Assemble!}
Place the op amp in the circuit across a gap. That way, the two sides of the op amp are not connected below the bread board.

\includegraphics{figures/ex.08.png}

Add the high pass resistor in series with the terminal 2 position of the op amp by placing one end in the same row as the terminal and the other in a row above. Make sure that these are not in the same row as any other elements. This is the inverting input.
 
\includegraphics{figures/ex.09.png}

Add the high pass capacitor in series with the resistor by placing one end in the same row as the other end of the resistor. 

\includegraphics{figures/ex.10.png}

Add a wire in series to the other end of the capacitor, and connect it to Va (voltage input).

\includegraphics{figures/ex.11.png}

Add one end of the low pass capacitor in series with terminal 2, and the other end in series with terminal 6. This puts the capacitor in parallel with op amp, as they both share the same nodes. Then, add the low pass resistor in series with the capacitor on both sides. This gives us the capacitor, the resistor, and the OpAmp all in parallel with each other. 

\includegraphics{figures/ex.12.png}
Next, add an output wire in series with terminal 6, and connect it to Vb.


\includegraphics{figures/ex.13.png}

Attach yellow wires to terminals 4 and 7 of the OpAmp; these will be the positive and negative source voltages. Don’t connect them; we will attach alligator clips to them later so we can power the OpAmp

\includegraphics{figures/ex.14.png}

Attach a wire in series with terminal 3, and connect it to ground. 

\includegraphics{figures/ex.15.png}

Now, we can start to add our power!

\subsection{With Great Power Comes Great Responsibility}

For clarity, we will be referring to the following locations as numbered below: 1 is the input voltage; 2 is the ground; 3 is the output voltage; 4 is the positive source voltage; 5 is the negative source voltage.


\includegraphics{figures/ex.16.png}

Use a voltage splitter to divide the output of the waveform generator, and attach the black wire to 2 (ground) in order to provide a reference to counteract the noise from the environment. Attach the red one to 1 (Va). 

\includegraphics{figures/ex.17.png}

The output will travel to the oscilloscope, where it will be read in and shown as a wave. So, attach one side to position 3 (the output). Or you can plug it into Vb, if you’re so inclined; there is no difference between these two positions.

Attach the other wire to position 2 (ground), because the oscilloscope will measure the difference between the output and the ground.


\includegraphics{figures/ex.18.png}

Plug the COM wire into 2 (ground), once again to act as a reference. Plug the 20+ into location 4 (positive source voltage) and the 20- into location 5 (negative source voltage).

\includegraphics{figures/ex.19.png}

This picture shows how all of the inputs and outputs actually attach to the power supply, oscilloscope, and waveform generator. 

\includegraphics{figures/ex.20.png}

\subsection{Testing, Testing...}

Now we can test how accurate our corner frequency is!

For a bandpass of 0.2Hz to 17.8Hz, here is an example of a signal within the bandwidth. This was driven at 4Hz, it can be seen that it has a fairly large amplitude.

\includegraphics{figures/ex.21.png}

This is the result after using 27Hz. The signal has significantly less amplitude and is therefore attenuated.

\includegraphics{figures/ex.22.png}

Now you have the fundamentals of building a bandpass! Of course, you can play around with this circuit to create different types of filters. Have fun, and be safe kids. \footnote{Barry's note: Thank you Isabel and Ryan for putting this all together.}

\chapter{Happenstance \& Circumstance: A few BME specific situations and standards}
04/16/2019
\minitoc
\newpage

\section{What is a ``medical device''?}
In the United States, medical devices are currently (ultimately) governed at the federal level where the term medical ``device'' is defined in 21 U.S.C. \S 321(h) as

\begin{quote}
	an instrument, apparatus, implement, machine, contrivance, implant, in vitro reagent, or other similar or related article, including any component, part, or accessory, which is--
	\\
	
(1) recognized in the official National Formulary, or the United States Pharmacopeia, or any supplement to them,\\

(2) intended for use in the diagnosis of disease or other conditions, or in the cure, mitigation, treatment, or prevention of disease, in man or other animals, or\\

(3) intended to affect the structure or any function of the body of man or other animals, and \\

which does not achieve its primary intended purposes through chemical action within or on the body of man or other animals and which is not dependent upon being metabolized for the achievement of its primary intended purposes.
\end{quote}

Any ``device'' to which that definition applies\footnote{With the exception of the term's use in 21 U.S.C. \S331(i), \S343(f), \S352(c), and \S362(c). Its currently debated relationship to software functions excluded in 21 U.S.C. \S 360(j)(o) is the subject of another discussion for another day.} is regulated in this country by a suite of standards, ranging from the entirely voluntary to the strictly mandatory.

\section{What is a ``standard'' and (how) is it enforced?}
As good a definition as I've yet come across, W.D. Rowe in ``Design and performance standards'' from \textit{Medical Devices: Measurements, Quality Assurance and Standards}, has it as

\begin{quote}
	A standard is a multi-party agreement for establishing an arbitrary criterion for reference.
\end{quote}
Each word in that sentence has a specific legal/functional meaning:
\begin{itemize}
	\item \textbf{Multi-party} means that more than one person, party, organization, agency, agent, individual, and/or government may be involved;
	\item \textbf{Agreement} indicates that the parties involved come to some mutually agreed upon understanding of the matters involved and of ways to resolve them; generally this understanding has been arrived at through unanimity, consensus, ballot, or some other method;
	\item \textbf{Establishing} defines the purpose of the agreement (to create a standard and carry forth its provisions);
	\item \textbf{Arbitrary} underscores that there is no absolute criterion undergirding the standard;
	\item \textbf{Criterion}, criteria, are those features, facets, aspects, and/or conditions which the parties have agreed will be the basis of the standard; and
	\item \textbf{Reference} represents the ideal towards which the standards sets one; it is what is desired and what reality will be measured against.
\end{itemize}


\section{A hierarchy of standards}
\subsection{Local and/or proprietary standards}
\subsection{Common interest standards}
\subsection{Consensus standards}
\subsection{Regulatory standards}

\newpage
\section{Legislation}
A chronological list of significant medical device legislation in the United States. Much of the content in this section comes from the U.S. Food and Drug Administration's own website onn the history of the organization found \href{https://www.fda.gov/MedicalDevices/DeviceRegulationandGuidance/Overview/ucm618375.htm}{\textbf{here}}.

\subsection{The ``Pure Food and Drugs Act of 1906''}
\begin{itemize}
	\item Established a precursor to today’s FDA (``Bureau of Chemistry'')l
	\item Prohibited interstate commerce of misbranded and adulterated food, and drugs
\end{itemize}

\subsection{The ``Federal Food, Drug, and Cosmetic Act of 1938''}
\begin{itemize}
	\item Primary statute that authorizes the FDA’s regulation and oversight of medical products
	\item Granted authority for factory inspections
	\item Extended prohibition of interstate commerce to misbranded and adulterated cosmetics and therapeutic medical devices
\end{itemize}

\subsection{The ``Public Health Service Act of 1944''}
\begin{itemize}
	\item Established certification of laboratories.
	\item Expanded oversight of biologics.
\end{itemize}

\subsection{The ``Radiation Control for Health and Safety Act of 1968''}
\begin{itemize}
	\item Intended to minimize exposure to electronic product radiation and intense magnetic fields.
	\item Created performance standards for radiation-emitting products, such as diagnostic x-ray machines, MRIs, microwave, ultrasound or diathermy devices, UV devices and laser devices.
\end{itemize}

\subsection{The ``Medical Device Amendments of 1976''}
\begin{itemize}
	\item The first attempt at a federal regulatory mechanism for medical devices designed, developed, tested, and sold in this country.
	\item Perhaps best known for introducing our current medical device classification based on risk to humans when used.
	\item Established the regulatory pathways for new medical devices (devices that were not on the market prior to May 28, 1976, or had been significantly modified) to get to market: \textbf{Premarket Approval (PMA)} and \textbf{premarket notification (510(k))}. Now, most devices are approved via 510(k). There is debate on whether or not the 510(k) pathway is more of a loophole or a necessary fast track for getting helpful devices to patients quickly. As biomedical engineers, our future employers will likely bring our work to market via this pathway. 
	\item Created the regulatory pathway for new investigational medical devices to be studied in patients (Investigational Device Exemption (IDE)).
	\item Established several key postmarket requirements: (1) registration of establishments and listing of devices with the FDA, (2) Good Manufacturing Practices (GMPs), and (3) reporting of adverse events involving medical devices.
	\item Authorized the FDA to ban devices.
\end{itemize}
\subsubsection{A classification of medical devices}
Regardless of classification, there exist general controls which apply to all medical devices bought, sold, manufactured, and used in the United States. Such controls include
\begin{enumerate}
	\item Prohibition against the adulteration or misbranding of devices;
	\item Requirement that domestic device manufacturers and (initial) distributors register their establishments and list their devices;
	\item The FDA has authority to audit/ban medical devices;
	\item Notification requirements of risks, repairs, replacements, and refunds for customers and end-users.
	\item Restriction of the sale, distribution, and/or use of devices; and
	\item Maintenance of ``good manufacturing practices'' (GMPs), records, reports, and inspections.
\end{enumerate}

\begin{itemize}
	\item \textbf{Class I (General Controls)} devices are considered as low risks for human use. For a Class I device, there already exists sufficient information/evidence to provide reasonable assurance of device safety and effectiveness. Such devices are \begin{enumerate}
		\item not to be for a use in supporting or sustaining human life or for a use which is of substantial importance in preventing impairment of human health, and
		\item does not present a potential unreasonable risk of illness or injury.
		\item Examples of Class I devices include elastic bandages, examination gloves, and hand-held surgical instruments.
	\end{enumerate}
	\item \textbf{Class II (Performance Standards)} devices are considered moderate risks for human use. Class II devices are those ``for which there is sufficient information to establish a performance standard to provide [] assurance'' that it would work as intended ``and for which it is therefore necessary to establish [...] a performance standard [...] to provide reasonable assurance of its safety and effectiveness.''
		\item Examples of Class II devices include powered wheelchairs, infusion pumps, and surgical drapes.
	\item \textbf{Class III (Premarket Approval) }devices are considered high risks for human use. To be a Class III device a device \begin{enumerate}
		\item ``cannot be classified as a class I device'',
		\item ``cannot be classified as a class II device'',
		\item ``is purported or represented to be for a use in supporting or sustaining human life or for a use which is of substantial importance in preventing impairment of human health or presents a potential unreasonable risk of illness or injury''
		\item Examples of Class III devices include replacement heart valves, silicone gel-filled breast implants, and implanted cerebellar stimulators.
	\end{enumerate}
\end{itemize}

\subsection{The ``Safe Medical Devices Act of 1990''}
\begin{itemize}
	\item Improved postmarket surveillance of devices by (1) requiring user facilities (such as hospitals and nursing homes) to report adverse events involving medical devices and (2) authorizing the FDA to require manufacturers to perform postmarket surveillance on permanently implanted devices if permanent harm or death could result from device failure. Postmarket surveillance is also required when a product is expected to "have significant use in pediatric populations." 
	\item Authorized the FDA to order device recalls and to impose civil penalties for violations of the The ``Federal Food, Drug, and Cosmetic Act of 1938''.
	\item Defined substantial equivalence (the standard for marketing a device through the 510(k) program). Reasons a device could not be considered substantially equivalent to another, as stated by the FDA, include:
	\subitem No predicate item exists.
	\subitem The device has a new intended use compared to the identified predicate product.
	\subitem the device has different technological characteristics that raise different questions of
safety and effectiveness than the identified predicate device; or
	\subitem the device has new indications for use or different technological characteristics than the identified predicate device, and required performance data was not provided to allow FDA to reach a substantial equivalence determination. This may include inadequate or inconclusive performance data (e.g., bench testing, clinical data, or animal data).
	\item Modified procedures for the establishment, amendment, or revocation of performance standards.
	\item Created the Humanitarian Use Device (HUD)/Humanitarian Device Ex4emption (HDE) programs to encourage development of devices targeting rare diseases.
\end{itemize}

\subsection{The ``Medical Device Amendments of 1992''}
\begin{itemize}
	\item  Amended certain provisions (section 519 of the ``Federal Food, Drug, and Cosmetic Act of 1938'') that relate to the reporting of adverse events. 
	\item The primary impact of the 1992 Amendments on medical device reporting was to define certain terms and to establish a single reporting standard for user facilities, manufacturers and distributors.
\end{itemize}

\subsection{The ``FDA Reform and Enhancement Act of 1996''}
\begin{itemize}
	\item Deals primarily with exports and establishing the basic requirements and procedures for exporting human drugs (also drug components) and biologics that may not be sold or distributed in the United States. 
	\item Also lists requirements for exporting drugs that are approved for marketing in the United States, but which are being exported for an unapproved use. 
\end{itemize}


\subsection{The ``Food and Drug Administration Modernization Act of 1997''}
\begin{itemize}
	\item Created the ``least burdensome'' provisions for premarket review.
	\item Created the option of accredited third parties to conduct initial premarket reviews for certain devices.
	\item Permitted the use of data from studies of earlier versions of a device in premarket submissions for new versions of the device.
	\item Provided for expanded access to investigational devices.
	\item Established the \textit{De Novo} program through which novel low-to-moderate risk devices could be classified into Class I or II instead of automatically classifying them into Class III. Many products go through the De Novo pathway after failing to get approved via 510(k). 
\end{itemize}

\subsection{The ``Medical Device User Fee and Modernization Act of 2002''}
\begin{itemize}
	\item Granted the FDA the authority to collect user fees for select medical device premarket submissions to help the FDA improve efficiency, quality, and predictability of medical device submission reviews.
	\item Enacted the Small Business Determination (SBD) program to permit reduced premarket approval fees for qualifying small businesses.
	\item Created FDA performance goals for decisions on certain premarket submissions.
	\item Established new regulatory requirements for ‘reprocessed’ devices.
	\item Authorized electronic registration of medical device firms.
	\item Established the Office of Combination Products.
\end{itemize}


\subsection{The ``Food and Drug Administration Amendments Act of 2007''}
\begin{itemize}
	\item Reauthorized the medical device user fee (``Medical Device User Fee and Modernization Act of 2002'', MDUFMA II), including improvements to premarket review times.
	\item Required that all registration and listing be performed electronically.
	\item Required the FDA to establish a unique device identification (UDI) system for medical devices to require device labels to bear a unique identifier.
\end{itemize}


\subsection{The ``Food and Drug Administration Safety and Innovation Act of 2012''}
\begin{itemize}
	\item Reauthorized the medical device user fee program (MDUFA III), including improvements to premarket review times and added shared outcome goals with industry.
	\item Created direct \textit{De Novo} pathway, permitting the classification of novel, low-to-moderate risk devices into Class I or II (rather than Class III) without first having to submit a 510(k).
	\item Changed the standards associated with disapproval of an IDE.
	\item Permitted the FDA to work with foreign governments to harmonize regulatory requirements
	\item Required FDA to provide a Substantive Summary when requested by the holder of the submission for significant decisions
	\item Expanded the application of the ``least burdensome'' principles in premarket reviews.
\end{itemize}

\subsection{The ``21st Century Cures Act of 2016''}
\begin{itemize}
	\item Codified into law the FDA’s expedited review program for breakthrough devices.
	\item Expanded the application of the ``least burdensome'' principles in premarket reviews.
	\item Streamlined processes for exempting devices from the premarket notification (510(k)) requirement.
	\item Increased the population estimate required to qualify for Humanitarian Use Device (HUD) designation from ``fewer than 4,000'' to ``not more than 8,000'' patients in the U.S. per year.
	\item Permitted the use of central Institutional Review Board (IRB) oversight rather than requiring only local IRBs for IDE and HDE activities.
	\item Required the FDA to revise the regulation of combination products.
	\item Codified into law a process for submitting requests for recognition/non-recognition of a standard.
	\item Clarified how certain digital health products can be regulated by defining the categories of medical software that can and cannot be regulated as devices.
\end{itemize}

\subsection{The ``Food and Drug Administration Reauthorization Act of 2017''}
\begin{itemize}
	\item Reauthorized the medical device user fee program (MDUFA IV), including improvements to premarket review times and investments in strategic initiatives like the National Evaluation System for health Technology (NEST) and patient input. 
	\item Authorized risk-based inspection scheduling for device establishments and prescribed other process improvements related to device establishment inspections.
	\item Decoupled accessory classification from classification of the parent device.
	\item Required the FDA to conduct at least one pilot project to explore how real-world evidence can improve postmarket surveillance.
\end{itemize}




\section{What does the FDA regulate?}
While not an exhaustive list, the following categories of products fall under the FDA's regulatory concerns:
\begin{itemize}
	\item Foods, including:
	\subitem dietary supplements
	\subitem bottled water
	\subitem food additives
	\subitem infant formulas
	\subitem other food products (although the U.S. Department of Agriculture plays a lead role in regulating aspects of some meat, poultry, and egg products)
	\item Drugs, including:
	\subitem prescription drugs (both brand-name and generic)
	\subitem non-prescription (over-the-counter) drugs
	\item Biologics, including:
	\subitem vaccines
	\subitem blood and blood products
	\subitem cellular and gene therapy products
	\subitem tissue and tissue products
	\subitem allergenics
	\item Medical Devices, including:
	\subitem simple items like tongue depressors and bedpans
	\subitem complex technologies such as heart pacemakers
	\subitem dental devices
	\subitem surgical implants and prosthetics
	\item Electronic Products that give off radiation, including:
	\subitem microwave ovens
	\subitem x-ray equipment
	\subitem laser products
	\subitem ultrasonic therapy equipment
	\subitem mercury vapor lamps
	\subitem sunlamps
	\item Cosmetics, including:
	\subitem color additives found in makeup and other personal care products
	\subitem skin moisturizers and cleansers
	\subitem nail polish and perfume
	\item Veterinary Products, including:
	\subitem livestock feeds
	\subitem pet foods
	\subitem veterinary drugs and devices
	\item Tobacco Products, including:
	\subitem cigarettes
	\subitem cigarette tobacco
	\subitem roll-your-own tobacco
	\subitem smokeless tobacco
\end{itemize}






\section{Some quick facts about the Food and Drug Administration}
From the U.S. FDA's website:
\begin{enumerate}
	\item FDA is responsible for the oversight of more than \$2.5 trillion in consumption of food, medical products, and tobacco.
	\item FDA-regulated products account for about 20 cents of every dollar spent by U.S. consumers.
	\item FDA regulates about 75 percent of the U.S. food supply. This includes everything we eat except for meat, poultry, and some egg products.
	\item There are over 19,000 prescription drug products approved for marketing.
	\item FDA oversees over 6,000 different medical device product categories.
	\item There are over 1,600 FDA-approved animal drug products.
	\item There are about 340 FDA-licensed biologics products.
	\item FDA regulations cover about 35,000 produce farms, 300,000 restaurant chain establishments, and 10,500 vending machine operators.
	\item FDA products are manufactured or handled at nearly 270,000 registered facilities, more than half of which are overseas.
	\item \textbf{About 80 percent of active pharmaceutical ingredients manufacturers are located outside of the U.S.}
	\item FDA-regulated products account for 12 percent of U.S. imports and 16 percent of U.S. exports.
	\item About 35 percent of medical devices used in this country are imports.
	\item The FDA budget for FY 2018 was \$5.4 billion.
	\item About 55 percent, or \$3 billion, of FDA’s budget is provided by federal budget authorization. The remaining 45 percent, or \$2.4 billion, is paid for by industry user fees.
	\item The FDA budget is equivalent to \$9.11 per American per year.
	\item The FDA budget includes 17,803 full time equivalents (FTEs).
	\item Human Drugs regulatory activities account for 30 percent of FDA’s budget; 69 percent of these activities are paid for by industry user fees.
	\item Devices and Radiological Health regulatory activities account for 10 percent of FDA’s budget; 35 percent of these activities are paid for by industry user fees.
	\item Foods regulatory activities account for 20 percent of FDA’s budget; 1 percent of these activities are paid for by industry user fees.
	\item Biologics regulatory activities account for 7 percent of FDA’s budget; 40 percent of these activities are paid for by industry user fees.
	\item Animal Drugs and Feeds regulatory activities account for 4 percent of FDA’s budget; 13 percent of these activities are paid for by industry user fees.
\end{enumerate}

\chapter{A philosophy of circuits, systems, and signals in biomedical engineering}
04/23/2019

\newpage







\part{Homework}
\setcounter{chapter}{0}
\chapter{Homework I}
Assigned January 22, 2019. Due January 31, 2019 as \textit{both} a hard copy in class and a pdf on Canvas. Each problem worth 10 points.

\setcounter{section}{0}
\section{On shocking the heart}
(15 points)	A typical automated external defibrillator (AED) delivers 200-1000 V in less than 10 ms. 
\begin{enumerate}
	\item If the AED in front us delivers a pulse of 600 V, how much current is needed to deliver 120, 240, and 360 Joules?
	\item Assuming two human hands have a mass of approximately 1 kg, how many chest compressions would be needed to deliver an equivalent amount of energy if each compression had a depth of 5 cm and was delivered at a constant speed of 0.5 m/s every 2 seconds?
	\item If you had the option of having your heart “jump-started” would you choose an AED or the bare hands of a stranger? Justify your answer.
\end{enumerate}

\newpage

\section{On energy within batteries}
(5 points) A  3.3 V battery you are considering for a wearable you are designing has a total charge of 300 mAh. How many joules is this battery capable of delivering?

\section{On predicting energy use}
(5 points) A fellow engineer bought a 12 V battery rated for 60 Ah. The experiment you both have in mind will draw 2 A over a 1 k$\Omega$ load. How long can your experiment last?

\section{On charges through equipment}
(5 points) A piece of medical equipment supplies 135 W at 220 V. How much electrical charge flows through the device in the 10 hours a nurse is on-call using it? And how many electrons does this charge correspond to?

\section{\textit{In vitro}}
(10 points) During an in vitro (petri dish) experiment, the peak electric power that a group of stem cells can tolerate without some serious functional consequence is known to have a threshold of about 1 mW. If the power delivered to this group of cells is defined as $p(t) = 2e^{-t}\sin 5t$ [mW], will the cell be harmed and if so how long can power be delivered before the cells are harmed?

\section{On equivalent resistance}
(10 points) Find the equivalent resistances. (Show your work.)
\begin{itemize}
	\item 1 square of resistors (R = 10 k$\Omega$) measured across corners.
	\item 2 resistors (R1 = 10 k$\Omega$, R2 = 4 k$\Omega$) in series.
	\item 3 resistors (R1 = R2 = R3 = 10 k$\Omega$) in parallel.
	\item 4 resistors (R1 = 1 k$\Omega$, R1 = 2 k$\Omega$, R1 = 3 k$\Omega$, R1 = 4 k$\Omega$) in parallel.
	\item 5 resistors (R = 10 k$\Omega$) in a pentagon, measured across each resistor.
\end{itemize}

\section{On uninterruptible power}
(10 points) Hospitals often employ what is known as a dynamic uninterruptible power system (D-UPS) comprising a diesel generator, a synchronous machine, and a kinetic energy unit. By way of example, if the energy to a particular hospital fails, the kinetic energy unit continues to feed 600 kW to the hospital for 20 seconds, allowing the generator and synchronous machine to take over and feed the load.
\begin{enumerate}
	\item What is the total energy capacity of this kinetic energy unit?
	\item If the hospital were operating at 240 kW (instead of the aforementioned 600 kW), how much longer can the unit feed the load?
\end{enumerate}


\section{On modeling current and voltage}
(10 points) Current passing through an electrical element in an AC situation can generally be defined as $i(t) = A \sin \omega t$. 
\begin{enumerate}
	\item Determine the energy on this element if the voltage across it is 
\\
$v(t) = B \cos \omega t$.
	\item Plot current, voltage, and energy as functions of time using MATLAB (or an equivalent software package). You may use any values of $A$, $B$, and $\omega$ that you’d like, but unit values (=1) might make your life/work easier. (Present the code you used to arrive at your plot.)
\end{enumerate}


\section{On the materiality of the human body}
(10 points) The materiality of the human body ensures that it puts up some resistance to the flow of current. Find some values of impedance for the following human body parts. Be sure to cite your source(s) and explain why you trust it. (Given that there is a frequency dependence to these values, you may present any value of impedance found between 1 | 100 kHz.)
\begin{enumerate}
	\item The whole body, from head to toe.
	\item A single limb (such as an arm or a leg).
	\item Blood.
	\item Muscle.
	\item Fat.
\end{enumerate}


\section{On circuit analysis}
(20 points) For the circuit shown below.
\begin{center}
	\includegraphics{figures/hw1.10.png}
\end{center}

\begin{enumerate}
	\item Use the delta-wye transformation rule to determine the power dissipated by R1.
	\item Find the voltage at node a (just after R1). 
	\item If R4 were changed to 100 $\Omega$, would the results to (10.1) and/or (10.2) change? How do you know?
	\item How many nodes, branches, and loops are there?
	\item What is the voltage at each node, the current through each branch?
\end{enumerate}




\chapter{Homework II}
Assigned February 8, 2019. Due February 14, 2019 as \textit{both} a hard copy in class and a pdf on Canvas. Each problem worth 10 points.
\setcounter{chapter}{2}
\setcounter{section}{0}
\section{On the directionality of resistors}
Though we rarely think about it, resistors have a directionality to them. To prove this to ourselves, let us consider a rectangular prism with a square cross-sectional area with a width of $a$ on its square face, a length of $L$, and a resistivity of $\rho$. An example of such a conductor may be seen in the figure below.
\begin{center}
	\includegraphics{figures/hw2.01.png}
\end{center}
\begin{enumerate}
	\item Determine the resistance between two parallel square faces, two parallel rectangular faces, and determine the ratio of these two resistances.
	\item How would you go about performing a similar analysis for a resistor with a circular cross-sectional area? 
\end{enumerate}



\section{On nodal analysis}
For the circuit shown below, find the current flowing though $R3$ ($i3$) and the current flowing through $R4$ ($i4$). 
\begin{center}
	\includegraphics{figures/hw2.02.png}
\end{center}


\section{On mesh analysis}
For the circuit shown below.
\begin{center}
	\includegraphics{figures/hw2.03.png}
\end{center}
\begin{enumerate}
	\item Perform a mesh analysis and determine the mesh currents in each loop. (You may solve it any way you’d like, just walk me through how you solve it.)
	\item Determine the current through R3. How you would actually measure the actual current through R3 in an actual circuit?
\end{enumerate}



\section{On Thevenin and Norton equivalent circuits}

For the circuit shown below, determine the Thevenin and Norton equivalent circuits between terminals a and b.
\begin{center}
	\includegraphics[width=0.5\textwidth]{figures/hw2.04.png}
\end{center}

\section{On superposition}
Using the superposition theorem, find the voltage \color{red} \textbf{Va} \color{black} in in the circuit shown below. Be sure to show all your steps.

\begin{center}
	\includegraphics[width=0.5\textwidth]{figures/hw2.05.png}
\end{center}

\section{On Thevenin and Norton equivalent circuits, again}
For the circuit shown below, determine the Thevenin equivalent circuit that would be found between the \color{red}\textbf{a} \color{black} and \color{red} \textbf{b} \color{black} terminals.
\begin{center}
	\includegraphics[width=0.5\textwidth]{figures/hw2.06.png}
\end{center}

\section{On input impedance}
Electrocardiographic (ECG) signals tend to be about 1 mV before being measured by a monitoring system (that is, it is about 1 mV just before it gets to our skin). If skin resistance is about 100 k$\Omega$, what is the voltage as measured by our system if our system has an input impedance of 500 k$\Omega$, 1 M$\Omega$, and 2M$\Omega$. You may model the entire situation as a series combination of elements. What effect do you think sweating at the area of measurement will have on the voltages you just calculated?

\section{On current through elements}
For the circuit below, find the current passing through each element.

\begin{center}\includegraphics{figures/hw2.08.png}\end{center}

\section{On blood vessel conductivity}
Our blood vessels are something of glorified hollow conductive cylinders. Let us assume that the resistivity of the vessel, $\rho$, is constant throughout our length of interest, $L$. If we were to apply a potential difference, $V$, between the inner surface of the blood vessel (with radius $a$) and the outer surface of the blood vessel (with radius $b$), what would be the resistance we measure? (Hint: the differential resistance in this case is equal to $dR = \rho(dL/A)$.)

\section{On personal experience}
To the extent you feel comfortable sharing this information, have you ever had to interact with a piece of biomedical equipment that utilized electronic circuitry? (This may include anything from a Fitbit to an ECG and beyond.) If so, what aspects of that experience relating to the equipment did you find fascinating, and which aspects of it would you personally liked to have seen improved? [If you do not wish to share a personal experience or perhaps do not have one to share, consider the question in the abstract: \textit{knowing what you know, what might you wish to see improved regarding current medical instruments}?]


\chapter{Homework III}
Assigned February 22, 2019. Due February 28, 2019 as \textit{both} a hard copy in class (at the beginning thereof) and a pdf on Canvas (by 11:59pm). Each problem worth 20 points.
\setcounter{chapter}{3}
\setcounter{section}{0}


\section{On mathematical proficiency}
Please find the Laplace transforms for the following functions. Be sure to show as much of your work as you can. And please also resist the urge to use computational means of finding these answers. Try it yourself! It's good exercise.

\begin{enumerate}
	\item $a(t) = \sin(\omega t)$
	\item $b(t) = \cos(\omega t)$
	\item $c(t) = 1 - \cos(\omega t)$
	\item $d(t) = t$
	\item $f(t) = t^n$
\end{enumerate}

\newpage

\section{On poles and zeros, exponentially}

Find the poles and zeroes of the following functions. Again, show as much of your work as you can and resist those computational urges.

\begin{enumerate}
	\item $f(t) = e^{-3t}$
	\item $f(t) = e^{-3t - 2t + 1t}\cdot e^{-t}$
	\item $f(t) = e^{-3t}\cdot \sin(3t)$
	\item $f(t) = e^{-3t}\cdot \cos(4t - 3t)$
	\item $f(t) = e^{3t}\cdot e^{-2t} \cdot \sin(t)\cdot cos (-t)$
\end{enumerate}

\section{On the stability of geopolitical systems}
``Stability'' is a property we seek in many systems. Representing something of a ``back of the envelope'' approximation of the time course of geopolitical entities, a model was first proposed by a fellow by the name of Lewis Fry Richardson to track the development of weapons used/accrued by the nations of the earth. Richardson's Arms race, expressed as the coupled differential equation below, relates the rate at which one country (Country 1) develops its arms, $dx/dt$ to the rate at which another country (Country 2) develops its arms $dy/dt$:

\begin{align}
	\frac{dx}{dt} &= Ay - mx \\
	\frac{dy}{dt} &= Bx - ny
\end{align}
One can see that these rates are proportional to the the amount of arms the other country has minus the amount of arms the country has. If we know that $s < 0$ for a system to be stable, determine the conditions of stability in this system. (Assume the world starts peacefully with no weapons in either country.) Comment on the results. 

\newpage

\section{On a circuit as a differential equation}
For circuit below
\begin{enumerate}
	\item Write its differential equation and take its Laplace transform (assume all initial conditions are equal to zero). 
	\item Find its transfer function (the ratio of voltage to current).
	\item  Choose \textbf{five} triplets of values (different from those printed on the figure). Plot the output of each transfer function as \textit{either} a function of time ($t$) or frequency ($s = \jmath \omega $).
\end{enumerate} 


\begin{center}
	\includegraphics{figures/hw3.01.png}
\end{center}

\newpage
\section{On poles and zeros, applicably}
Believe it or not, this Laplace and circuitry stuff can actually be useful to those more biologically inclined among us. Indeed, it can be used to describe how drugs perfuse into our bodies. Let us consider taking some drugs.\footnote{Don't do drugs, kids.}  We may model the drug as a charged capacitor, C1, with a potential of 100 V across it and a value of 1 F. Recalling that the tissues within our bodies may be modeled as the combination of a resistor, R1, in series with the parallel combination of a resistor, R2, and a capacitor, C2, we may model the perfusion of the drug within the body as the time rate of change of potential within the circuit shown at right. Once we’ve taken the drug (t = 0), the switch is closed. Using nodal analysis, determine the potential at point b in the s-domain. What are the poles of this model? 
\begin{center}
	\includegraphics{figures/hw3.02.png}
\end{center}

\chapter{Homework IV}
Assigned March 14, 2019. Due March 28, 2019 as \textit{both} a hard copy in class (at the beginning thereof) and a pdf on Canvas (by 11:59pm). Each problem worth 20 points.
\setcounter{chapter}{4}
\setcounter{section}{0}

\section{On equivalent differential equations for transfer functions}
For each of the transfer functions, $\tilde{H}(s)$, below, find a time-domain differential equation, $h(t)$, that would produce such a transfer function. (In this problem and those that follow from it, you may choose any initial conditions you would like, but I suggest setting them to 0 as a first attempt.)
\begin{enumerate}
	\item $\tilde{H}(s) = \frac{4}{s^2+5s+6}$
	\item $\tilde{H}(s) = \frac{4}{s^2+4s+4}$
	\item $\tilde{H}(s) = \frac{4}{s^2+5s+4}$
	\item $\tilde{H}(s) = \frac{4}{s^2-4}$
	\item $\tilde{H}(s) = \frac{4}{s^2-5s-6}$
\end{enumerate}

\newpage
\section{On designing circuitry}
Let us treat the transfer functions reported in Problem 1 as system impedance, $\tilde{Z}(s)$.
For each of the differential equations you found from Problem 1, draw a circuit which whose output, $v(t)$, represents the voltage time response of an input of a unit step of current, $u(t)$ applied at time, $t = 0$. (That is, design a circuit which when subjected to an input $u(t)$ would produces an output of $v(t) = z(t)*u(t)$.)

\section{On determining output}
Given the transfer functions reported in Problem 1, please answer the questions below.
\begin{enumerate}
	\item Is the system stable when subjected to a unit step function? (In other words, is it BIBO stable?)
	\item How damped is the system?
	\item Where are the poles and zeros of the transfer function in the $s$-domain?
	\item How long would it take for each system to achieve 95\% of its total response? (For systems to which this does not apply, please state why.)
\end{enumerate}


\section{On designing filters}
Design the following filters (using an input impedance, R1 = 1 k$\Omega$) and produce a Bode plot demonstrating your results. (You may plot your results via computational means, i.e., a computer program, or by hand, by using what we learned in this class.)
\begin{enumerate}
	\item A first order high-pass filter with a corner frequency at 0.5 Hz and a magnitude of gain of 10.
	\item A second low-pass filter with a corner frequency of 40 Hz and a magnitude gain of 50.
	\item A bandpass filter with filter with a lower corner frequency of 0.05 Hz and a higher corner frequency of 150 Hz.
\end{enumerate}


\section{On convolving}

Given an input signal (seen on the left) and a system impulse response, SIR (seen on the right), determine the output signal that would arise. That is, convolve the input with the SIR. You may solve by hand (so long as your work is clear) and/or MATLAB, Excel, or some equivalent software to arrive at your answers. Please include all code used.
\begin{enumerate}
	\item What is the output (the convolution) of ``Input (a)'' and ``SIR (a)''?
	\item What is the output of ``Input (b)'' and ``SIR (b)''?
	\item What is the output of ``Input (c)'' and ``SIR (c)''?
\end{enumerate}

\begin{center}
	\includegraphics[width=\textwidth]{figures/hw4.01.png}
\end{center}

\chapter{Homework V}
Assigned March 14, 2019. Due April 11, 2019 as \textit{both} a hard copy in class (at the beginning thereof) and a pdf on Canvas (by 11:59pm). Each problem worth 20 points.
\setcounter{chapter}{5}
\setcounter{section}{0}


\section{On something new from a few familiar things}
Two RC circuit sections (with capacitors grounded) are separated by a voltage buffer. An input signal $v_i(t)$ is applied to first $R_1C_1$ section, producing the signal $v_i(t)$ at node 1, it is buffered by the op-amp to the input of the second $R_2C_2$ section (node 2), the output of which is $v_o(t)$.
\begin{center}
	\includegraphics[width=0.75\textwidth]{figures/hw5.01.png}
\end{center}
\begin{enumerate}
	\item Derive the differential equation relating the circuit output, $v_o(t)$, to its input, $v_i(t)$.
	\item Determine its characteristic equation (i.e., of the form $s^2 + 2\zeta \omega_n s + omega_n^2$) and find its roots.
	\item Determine whether this circuit could produce an underdamped response.
\end{enumerate}

\newpage

\section{On reviewing the fundamentals}
Recall a series circuit of a resistor, a capacitor, and an inductor. Let us hold that C = 200 nF and L = 50 mH.
\begin{enumerate}
	\item Over what range of resistors values is the system underdamped, critically damped, and overdamped?
	\item From within the range found above, select one representative value for the underdamped, critically damped, and overdamped case. Plot each case’s voltage response over time.
\end{enumerate}

\section{On bioimpedance}
In class we derived the analytical solution of the equivalent impedance, $Z_{eq}$ of a simple bioimpedance model, $R_1 + R_2\vert\vert C$ as seen from the resistance reactance plane. Find the analytical solution to the equivalent \textit{admittance} of a model in which $R_1\vert\vert(R_2 + C)$. (Hint: it might be more helpful to think in terms of conductances!) Interpret your results.


\section{On the body electric}

Given Einthoven's triangle below of the QRS complex, draw (by hand) the shape of the resulting electrocardiographic signal for lead I, lead II, and lead III. [Please use the conventional arrangement of Einthoven’s triangle when referring to your leads, unless you explicitly state you do otherwise.]

\begin{center}	\includegraphics[width=0.5\textwidth]{figures/hw5.02.png}
\end{center}

\section{On identifying real-world instrumentation}
As it turns out, there are hundreds and thousands of medical devices out there that use electric circuits. We noted five separate electrophysiological modalities in class – electrocardiography, electroencephalography, electrogastrography, electromyography, and electrooculaography. Find at least three commercial examples of medical devices for each listed modality (i.e., a total of 15 devices). List (1) who makes it, (2) how the product generally works, and (3) if you can find it, how much it costs. (You may also comment on whether you believe that price to be fair/reasonable.) 


\chapter{Homework VI}
Assigned March 14, 2019. Due April 23, 2019 as \textit{both} a hard copy in class (at the beginning thereof) and a pdf on Canvas (by 11:59pm). Each problem worth X points.
\setcounter{chapter}{6}
\setcounter{section}{0}

\section{On functional completeness}
In class we discussed the fact that all possible computational relationships (NOT, AND, OR, NAND, NOR, XOR, and XNOR) can be represented by a combination of NOR gates. During class we were only able to prove it for NOT, AND, and OR. Prove it for all cases. That is, make NAND, NOR, and XOR gates (don’t do XNOR) from NOR gates.

\newpage
\section{On knights, knaves, truths, and lies}
Recall the land inhabited by knights and knaves. Knights always tell the truth and knaves always lie. For each of the following give an answer and explain your reasoning as best you can. It will often help to draw a truth table or a logic/decision tree. Please use your own methods or feel free to ask me what I might do.

Please note that I will make use of the singular form of ``they'' and ``them'' to describe the individuals below. If that leads to any consternation, you may replace with any singular pronoun of your choice.

\begin{enumerate}
	\item You come across three inhabitants and ask the first, A, ``Are you a knight or a knave?'' A answers, but so quietly you can’t hear them. You ask B ``What did A say?” to which B responds “A said they were a knave.'' Upon hearing this, C piped up and said ``Don’t believe that; it’s a lie.'' Is C a knight or a knave? (Further, is it possible to know what A is?)
	\item Instead of asking if A was a knight or a knave, you could have asked how many of the three of them were knaves. So, upon meeting the next three inhabitants (D, E, and F), D answers indistinctly, so you ask E what D had said. E said that D had said that exactly two of them were knaves. F says E is lying. Is it possible to know what D is? (Further, what are E and F?)
	\item Meeting with two inhabitants, G and H, G says, ``Both of us are knaves.'' What is G? What is H?
	\item Meeting two other inhabitants, I and J, I says, ``At least one of us is a knave.'' What are I and J?
	\item Meeting yet two more, K and L, K says, ``We are the same type – we are either both knights or both knaves.'' What are K and L?
	\item Meeting another two, M and N, you ask M if they are a knight and get an answer. You then ask N if M is a knight and get an answer. Are M and N’s answers the same or different? 
\end{enumerate}

\newpage

\section{On logic gates}
Did you know that each of the above knight and knave situations can be represented by a combination of logic gates? For example, a simple single knave can be represented by a NOT gate since they always lie. An easy way to work in this matter is to assign 0s and 1s to lies and truth (and thereby liars and truth-tellers and thereby knaves and knights) and construct a truth table. From that truth table and from the logic you used in deriving the solution, put together a sequence of logic gates for each of the situations reported in Problem 2.

\part{Glorified Quizzes}

\setcounter{chapter}{0}
\chapter{A Glorified Quiz I}
Given on February 12, 2019. All problems worth 10 points. Do not spend an inordinate amount of time on any given problem. Try your best.

\setcounter{section}{0}
\minitoc
\newpage

\section{Problem 1 on basic impedance}

Find the equivalent impedance of the following (``+'' means ``in series with'' and ``$\vert\vert$'' means ``in parallel with'') being sure that you end up with no imaginary terms in denominators. It may be helpful to draw these circuits as you understand them.
\begin{enumerate}
	\item 
	\item 
	\item 
\end{enumerate}


\section{Problem 2 on basic knowledge}
\begin{enumerate}
	\item 
	\item 
	\item 
	\item 
	\item 
\end{enumerate}

\section{Problem 3 on current and potential}
A current, $i(t)$, enters a device, causing a potential drop, $v(t)$. The current and the potential can be described mathematically as

\begin{equation*}
	i(t) = \begin{cases} 
      0 \text{ A}, & t< 0\\
      f(t) \text{ mA}, & t \geq0 \\
   \end{cases}
\end{equation*}
\begin{equation*}
	v(t) = \begin{cases} 
      0 \text{ V}, & t< 0\\
      g(t) \text{ V}, & t \geq0 \\
   \end{cases}
\end{equation*}

\begin{enumerate}
	\item 
	\item 
	\item 
\end{enumerate}

\newpage

\section{Problem 4 on transresistance amplification or bioamplification}
There are at least two kinds of current-to-voltage converters (also known as \textit{transresistance amplifiers)}.

\section{Problem 5 on equivalent impedance}
Choose one of the following (or do both for bonus points).
\begin{enumerate}
	\item Determine the equivalent impedance seen between terminals a and b.
	\item Determine the equivalent impedance of a three-dimensional network of resistors, each of value $R$, as measured from the two farthest corners.
\end{enumerate}


\section{Problem 6 on differential amplification}
\begin{enumerate}
	\item What is the potential difference between points a and b?
	\item What is the gain seen between points a and b?
	\item What is the current running through R2?
	\item What is the current running through R4?
	\item If the operational amplifier (LM741) were only supplied 10 V, what is the largest V1 we could input before our output was saturated?
\end{enumerate}

\section{Problem 7 on counting cells}
A Coulter counter (an electronic means by which to count the number of cells passing by a channel) is essentially a bridge circuit that determines the presence of a cell by tracking a change in due to a change in resistance from the cell. Referring to the circuit below, $R_1 = R_2 = R_3 = R_4 = R$ at rest (no cells present) and every time a cell passes through the channel R2 increases its resistance by $\Delta R$.

\begin{enumerate}
	\item 
	\item 
	\item 
\end{enumerate}


\section{Problem 8 on your own}

\begin{enumerate}
	\item Set up the series of simultaneous equations to \textbf{find the mesh currents}.
	\item Determine \textbf{the matrix form} of the simultaneous equations.
	\item Determine \textbf{the power dissipated} by three of your resistors.
	\item Find \textbf{the equivalent resistance} of the system as measured from the two most distant corners of your circuit drawing.
\end{enumerate}

\section{Problem 9 also on your own}

\begin{enumerate}
	\item Determine whether your circuit \textbf{satisfies the fundamental theorem of network topology}.
	\item \textbf{Determine the nodal voltages} (\textit{at the very least set them up as a solvable set of simultaneous equations} and try to solve them).
	\item \textbf{Determine the current} going through three of your resistors.
\end{enumerate}

\section{Problem 10 on miscellanea}
\begin{enumerate}
	\item 
	\item 
	\item 
\end{enumerate}


\newpage
\section{Useful equations in alphabetical order}
\begin{eqnarray*}
	b &=& l+n-1 \\
	C &=& \frac{Q}{V} \rightarrow i(t) \frac{dQ(t)}{dt} = C\frac{dV(t)}{dt}\\
	\text{Cramer's Rule: } x &=& \frac{D_1}{D}, y = \frac{D_1}{D}\text{ for }a_1x+b_1y=c_1, a_2x+b_2y=c_2\\
	\text{where D } &=& \begin{bmatrix}
		a_1 & b_1 \\ a_2 & b_2
	\end{bmatrix}, D_1 = \begin{bmatrix}
		c_1 & b_1 \\ c_2 & b_2
	\end{bmatrix}, D_2 = \begin{bmatrix}
		a_1 & c_1 \\ a_2 & c_2
	\end{bmatrix}  \\
	\mathbf{E} &=& \rho \mathbf{J} \\
	E &=& pt \\
	e &=& -1.602\cdot10^{-19}\text{ C} \\
	i &=& \frac{dq}{dt} \\
	i_{N} &=& i_{sc} = V_{Th}/R_{Th} \\
	\text{KCL: } \sum_{x=1}^{n}i_x &=& 0 \\
	\text{KVL: } \sum_{x=1}^{n}v_x &=& 0 \\
	L &=& \frac{\Phi}{I} \rightarrow v(t) = \frac{d\Phi}{dt} \\
	p &=& \frac{dw}{dt} \\
	Q &=& \int_{t_1}^{t_2}i(t)dt \\
	R &=& \rho \frac{l}{A} \\
	v &=& \frac{dw}{dq} \\
	v &=& iR \\
	V_{Th} = V_{oc} \\
	\mathbf{Z}&=& R + \jmath X = \vert \mathbf{Z} \vert e^{\jmath \theta} \\
	\text{Parallel: } \frac{1}{Z_{eq}} &=& \frac{1}{Z_{1}} + \frac{1}{Z_{2}} + \frac{1}{Z_{3}} + ...\\
	\text{Series: } Z_{eq} &=& Z_1 + Z_2 + Z_3 + ...
\end{eqnarray*}

\chapter{A Glorified Quiz II}
\section{Problem 1, a review}
\section{Problem 2, our biopotential}
\section{Problem 3, (con)/(in)volved}
\section{Problem 4, implanted}
\section{Problem 5, to zap your brain}

\newpage
\section*{Useful equations in alphabetical order}
\begin{eqnarray*}
	b &=& l+n-1 \\
	C &=& \frac{Q}{V} \rightarrow i(t) \frac{dQ(t)}{dt} = C\frac{dV(t)}{dt}\\
	\text{Convolution: } (f*g)(t) &=&  \int _{-\infty }^{\infty }f(\tau )g(t-\tau )\,d\tau \\
	\text{Cramer's Rule: } x &=& \frac{D_1}{D}, y = \frac{D_1}{D}\text{ for }a_1x+b_1y=c_1, a_2x+b_2y=c_2\\
	\text{where D } &=& \begin{bmatrix}
		a_1 & b_1 \\ a_2 & b_2
	\end{bmatrix}, D_1 = \begin{bmatrix}
		c_1 & b_1 \\ c_2 & b_2
	\end{bmatrix}, D_2 = \begin{bmatrix}
		a_1 & c_1 \\ a_2 & c_2
	\end{bmatrix}  \\
	\delta(t) &=& \begin{cases} 
      \infty, & t = 0\\
      0, & t \neq 0\\
   \end{cases} \\
   \delta_{\Delta}(t) &=& \begin{cases} 
      \frac{1}{\Delta}, & \text{if } 0< t < \Delta \\
      0, & \text{otherwise }\\
   \end{cases} \\
	\mathbf{E} &=& \rho \mathbf{J} \\
	E &=& pt \\
	e^- &=& -1.602\cdot10^{-19}\text{ C} \\
	e^{\jmath x} &=& \cos x + \jmath \sin x \\
	e^{\jmath \pi} &=& -1 \\
	i &=& \frac{dq}{dt} \\
	i_{N} &=& i_{sc} = V_{Th}/R_{Th} \\
	\text{KCL: } \sum_{x=1}^{n}i_x &=& 0 \\
	\text{KVL: } \sum_{x=1}^{n}v_x &=& 0 \\
	L &=& \frac{\Phi}{I} \rightarrow v(t) = \frac{d\Phi}{dt} \\
	\end{eqnarray*}

\newpage

\begin{eqnarray*}
	\mathcal{L}\{f(t)\} &=& \int_0^{\infty}e^{-st}f(t)dt \\
	\mathcal{L}\{1\} &=& \frac{1}{s} \\
	\mathcal{L}\{e^{at}\} &=& \frac{1}{s - a} \\
	\mathcal{L}\{\sin t\} &=& \frac{a}{s^2 + a^2} \\
	\mathcal{L}\{\frac{dx}{dt}\} &=& s\tilde{X} - x(o) \\
	p &=& \frac{dw}{dt} \\
	Q &=& \int_{t_1}^{t_2}i(t)dt \\
	R &=& \rho \frac{l}{A} \\
	r(t) &=& {\begin{cases}t,&t\geq 0;\\0,&t<0\end{cases}}\\
	u(t) &=& \begin{cases} 
      1, & t \geq 0\\
      0, & t \neq 0\\
   \end{cases} \\	v &=& \frac{dw}{dq} \\
	v &=& iR \\
	V_{Th} &=& V_{oc} \\
	\mathbf{Z}&=& R + \jmath X = \vert \mathbf{Z} \vert e^{\jmath \theta} \\
	\text{Parallel: } \frac{1}{Z_{eq}} &=& \frac{1}{Z_{1}} + \frac{1}{Z_{2}} + \frac{1}{Z_{3}} + ...\\
	\text{Series: } Z_{eq} &=& Z_1 + Z_2 + Z_3 + ... \\
	Z_{eq,C} &=& \frac{1}{\jmath \omega C}\\
	Z_{eq,L} &=& R \\
	Z_{eq,R} &=& \jmath \omega L 
\end{eqnarray*}

\chapter{A Glorified Quiz III}
\textbf{04/18/2019, in class}. Below is the outline of the third glorified quiz.

\minitoc

\section{Our words, our definitions}
\section{A series RLC circuit}
\section{A Laplace transform}
\section{Current through tissue}
\section{An instrumentation amplifier}
\section{A filter}
\section{Putting it all together}
\section{Truth and lies}
\section{Turning it off and on again}
\section{A parade}
\newpage
\section*{Useful equations in alphabetical order}
\begin{eqnarray*}
	b &=& l+n-1 \\
	C &=& \frac{Q}{V} \rightarrow i(t) \frac{dQ(t)}{dt} = C\frac{dV(t)}{dt}\\
	\text{Convolution: } (f*g)(t) &=&  \int _{-\infty }^{\infty }f(\tau )g(t-\tau )\,d\tau \\
								  &=&  \int _{-\infty }^{\infty }f(t - \tau )g(\tau )\,d\tau \\
	\text{Cramer's Rule: } x &=& \frac{D_1}{D}, y = \frac{D_1}{D}\text{ for }a_1x+b_1y=c_1, a_2x+b_2y=c_2\\
	\text{where D } &=& \begin{bmatrix}
		a_1 & b_1 \\ a_2 & b_2
	\end{bmatrix}, D_1 = \begin{bmatrix}
		c_1 & b_1 \\ c_2 & b_2
	\end{bmatrix}, D_2 = \begin{bmatrix}
		a_1 & c_1 \\ a_2 & c_2
	\end{bmatrix}  \\
	\delta(t) &=& \begin{cases} 
      \infty, & t = 0\\
      0, & t \neq 0\\
   \end{cases} \\
   \delta_{\Delta}(t) &=& \begin{cases} 
      \frac{1}{\Delta}, & \text{if } 0< t < \Delta \\
      0, & \text{otherwise }\\
   \end{cases} \\
	\mathbf{E} &=& \rho \mathbf{J} \\
	E &=& pt \\
	e^- &=& -1.602\cdot10^{-19}\text{ C} \\
	e^{\jmath x} &=& \cos x + \jmath \sin x \\
	e^{\jmath \pi} &=& -1 \\
	i &=& \frac{dq}{dt} \\
	i_{N} &=& i_{sc} = V_{Th}/R_{Th} \\
	\text{KCL: } \sum_{x=1}^{n}i_x &=& 0 \\
	\text{KVL: } \sum_{x=1}^{n}v_x &=& 0 \\
	L &=& \frac{\Phi}{I} \rightarrow v(t) = \frac{d\Phi}{dt} \\
	\end{eqnarray*}


\begin{eqnarray*}
	\mathcal{L}\{f(t)\} &=& \int_0^{\infty}e^{-st}f(t)dt \\
	\mathcal{L}\{1\} &=& \frac{1}{s} \\
	\mathcal{L}\{e^{at}\} &=& \frac{1}{s - a} \\
	\mathcal{L}\{\sin t\} &=& \frac{a}{s^2 + a^2} \\
	\mathcal{L}\{\frac{dx}{dt}\} &=& s\tilde{X} - x(o) \\
	p &=& \frac{dw}{dt} \\
	Q &=& \int_{t_1}^{t_2}i(t)dt \\
	R &=& \rho \frac{l}{A} \\
	r(t) &=& {\begin{cases}t,&t\geq 0;\\0,&t<0\end{cases}}\\
	\text{Systems in general: D.E.}  &=& \frac{d^x}{dt^2} + 2\zeta\omega_n \frac{dx}{dt} + \omega_n^2\\
	\text{Systems in general: T.F.}  &=& \frac{K}{s^2 + 2\zeta\omega_n s + \omega_n^2}  \\
	u(t) &=& \begin{cases} 
      1, & t \geq 0\\
      0, & t \neq 0\\
   \end{cases} \\	v &=& \frac{dw}{dq} \\
	v &=& iR \\
	V_{Th} &=& V_{oc} \\
	\mathbf{Z}&=& R + \jmath X = \vert \mathbf{Z} \vert e^{\jmath \theta} \\
	\text{Parallel: } \frac{1}{Z_{eq}} &=& \frac{1}{Z_{1}} + \frac{1}{Z_{2}} + \frac{1}{Z_{3}} + ...\\
	\text{Series: } Z_{eq} &=& Z_1 + Z_2 + Z_3 + ... \\
	Z_{eq,C} &=& \frac{1}{\jmath \omega C}\\
	Z_{eq,L} &=& R \\
	Z_{eq,R} &=& \jmath \omega L 
\end{eqnarray*}


\newpage


\part{Collected exercises}

\setcounter{chapter}{0}
\chapter{Collected exercises}

\setcounter{section}{0}
\minitoc

\section{A constant charge through a cross-section}
How much charge passes through a cross-section of a conductor in 60 seconds if a DC current value is measured at 0.1 mA?


\section{An arbitrary charge through a cross-section}
Determine the total charge entering a terminal between $t = 0$ seconds and $t = 10$ seconds if the current (in amps) passing through is
\begin{equation}
	i(t) = \frac{1}{\sqrt{5t+2}}. 
\end{equation} 


\section{A ``tera''ble puzzle}
Approximately how much current is necessary to transmit one terabyte of information in an hour?


\section{A pacemrker's power requirements}
A cardiac pacemaker will provide approximately 5,000 J of energy over 5 years. Determine the capacity of a 5 V lithium battery necessary to drive this pacing such that only 40\% of its energy is spent over that time.


\section{A neuron's excitation energy}
A colleague of yours has been in their lab ginning up new neurons. You, as their resident electrical expert, are tasked with determining the energy consumed by the cell. If the current and voltage variations are found to be functions of time ($t \geq 0$)
\begin{eqnarray}
	i(t) = 3t \\
	v(t) = 10 e^{6t}
\end{eqnarray}
determine the energy consumed between 0 and 2 ms.


\section{A thump to the chest}
(a) A typical defibrillator delivers 200-1000 V in less than 10 ms. How much current is needed to deliver 120, 240, and 360 Joules?
\\
(b) A human heart ways about 300 grams. From approximately how high of a cliff would one have to drop a heart such that the impact was equivalent to the energy delivered to someone's chest from a defibrillator?

\section{An ohmic power expression}
Utilizing Ohm's law, express units of power to include ohms.



\section{A toaster based problem}
A toaster draws 2 A at 120 V. What is its resistance?

\section{Another toaster based problem}
How much current is drawn by a toaster with a resistance of 10 $\Omega$ at 110 V?


\section{A current power}
In the circuit shown, calculate the current, $i$, the conductance, $G$, and the power, $p$.


\section{A sodium channel's conductance}
Conductance ($G$/$A$) of a sodium channel of a cell membrane at a specific time is 10 mS/cm$^{2}$. If the channel length as 100 nm, what is its conductivity?



\section{A simple tissue's resistance}
Determine the resistance of a homogenous and isotropic tissue with a cross-sectional area which can be described by the functions $y = 8 - x^2$ from $x = -2$ cm to $x = +2$ cm, a length of 10 cm (parallel to the z-axis), and a resistivity of 80 $\Omega$m.

\section{A few nodes of KCL}
Use KCL to write equations at each node.
\begin{center}
	\includegraphics[width=0.5\textwidth]{figures/04.problem1.png}
\end{center}


\section{Matrix notation}
Write the matrix form of the equations written above. 


\section{Cramer's rule}
Using Cramer’s rule on the matrix equations above, what are the results?



\section{Straight to the matrix}
Write the node-voltage equations to the circuit at right in the matrix form.
\begin{center}
	\includegraphics[width=0.5\textwidth]{figures/04.problem4.png}
\end{center}

\section{Laplace transformations}
Find the Laplace transform of the following:
\begin{itemize}
	\item $f(t) = 9$
	\item $f(t) = \delta(t)$ the Dirac-delta function
	\item $f(t) = e^{-3t/2}$
	\item $f(t) = \sin\omega t$
\end{itemize}


\section{Differential equation of a series RLC circuit}
What is the differential equation describing an inductor, a resistor, and a capacitor in series?



\section{Differential equation of a series RLC circuit with numbers}
What is the differential equation describing an inductor, a resistor, and a capacitor in series? What is a solution to that  differential equation if L = 1 H, R = 1 $\Omega$, and C = 100 mF? Take the Laplace transform of the differential equation. What are its poles and zeros?



\section{An $s$-plane with found zeros}
Draw an s-plane. Label the axes. Plot the poles and zeros you found. What can you say of the behavior of the system? If the poles had an imaginary component to them (that is, if they were pushed along the vertical axis), how would that affect our system?

\section{An $s$-plane with arbitrary zeros}
Draw an s-plane. Label the axes. Plot these poles -- (-2,0) and (-5,0) -- and this zero -- (0,0). Plot what the signal would look over time at each of these poles and zeros. 

\section{$\tilde{Z}(s)$ (i.e., $\tilde{V}(s)/\tilde{I}(s)$) of a resistor and a capacitor in parallel}
\begin{enumerate}
	\item Show the \textit{frequency response} (i.e., $\tilde{Z}(s)$ v. frequency) of such a system. If it helps to ascribe values, assume that the resistor is 1 ohm and the capacitor is 10 farad.
	\item Comment on the behavior. Is it like anything you have seen before?
	\item (\textit{The kind of question that might be on a glorified quiz.}) If the current through the circuit is 1 amp when $t \geq 0$, what will be the voltage response?
\end{enumerate}

\section{A voltage time response of a system with a given impedance}
Given that 
\begin{equation}
	\tilde{Z}(s) = \frac{s^3+9}{(s+1)(s+3)}
\end{equation}
\begin{enumerate}
	\item What is the \textit{time response} of potential to a unit step of current?
	\item (\textit{The kind of question that might be on a glorified quiz.}) Which exponential term will most significantly affect the signal?
\end{enumerate}

\section{A stable system?}
\begin{enumerate}
	\item A system, $y(t) = \int_{-\infty}^{\infty}x(\tau)d\tau$, when $x(t) = \cos(t)$.
	\item A system, $y(t) = \int_{-\infty}^{\infty}x(\tau)d\tau$, when $x(t) = u(t)$.
	\item A system, $y(t) = x(t)/t$, when $x(t) = 2$.
	\item A system, $y(t) = dx(t)/dt$, when $x=1$.
	\item A system, $y(t) = dx(t)/dt$, when $u(t)$.
\end{enumerate}

\section{Pedro, Apollonia, \& Peter}
You come across three inhabitants and ask the first, Pedro, ``Are you a knight or a knave?'' Pedro answers, but so quietly you can't hear him. You ask Apollonia ``What did he say?'' to which she responds ``Pedro said he was a knave.'' Upon hearing this, Peter piped up and said ``Don't believe that; it’s a lie.'' Is Peter a knight or a knave? (Further, is it possible to know what Pedro is?)


\section{Roger \& Oedipa} 
Shortly after that you meet two inhabitants, Roger Mexico and Oedipa Maas. Roger claims, ``Both of us are knaves.'' What are Roger and Oedipa?

\section{Yes \& No} 
Suppose you’ve heard a rumor that there’s gold buried nearby. You meet a local and want to know whether there really is gold in them thar hills, but you don’t know whether the person is a knight or a knave. If you are only allowed to ask only one question answerable by ``yes'' or ``no'', what do you ask?

\section{Lisa \& Louise}
Lisa and Louise are twins indistinguishable in appearance. One always lies, the other always tells the truth. You don’t know which is which. You meet one of them and may ask one question to determine which twin is truth. What do you ask and what does it tell you?


\section{NOT $\rightarrow$ NOR}
Make a NOT gate from one or more NOR gates.
\section{AND $\rightarrow$ NOR}
Make an AND gate from one or more NOR gates.
\section{OR $\rightarrow$ NOR}
Make an OR gate from one or more NOR gates.
\section{NAND $\rightarrow$ NOR}
Make a NAND gate from one or more NOR gates.


\chapter{A set of exercises in preparation of the first Glorifed Quiz}
02/07/2019 

\begin{enumerate}
	\item Find the equivalent resistance of the following [``+'' means ``in series with'', ``$\vert\vert$'' means ``in parallel with'']:
	\subitem a.	R1 + R2$\vert\vert$(R3 + R4)
	\subitem b.	(R1 + R2)$\vert\vert$(R3 + R4)
	\subitem c.	R1 + (R2 + R3$\vert\vert$(R4 + R5)||R6)
	\subitem d.	(R1$\vert\vert$R2$\vert\vert$R3)$\vert\vert$(R4$\vert\vert$R5$\vert\vert$R6)||(R7 + R8$\vert\vert$R9)
	\item To use superposition, one must turn off sources. A voltage source is replaced by what? A current source is replaced by what?
	\item A 5 µF capacitor has accumulated a total charge of 1250 µC. What is the voltage across the capacitor?
	\item Name each of the four types of dependent power sources and explain their functioning a bit beyond their name.
	\item The unit of current is the ampere, A. What are some equivalent units (for instance, what is its relationship to charge)?
	\item Does potential change instantaneously over the terminals of a capacitor?
	\item An RC circuit (in which ``R + C'', seen below) has a time constant, $\tau = RC$. Prove the following at a time $t_0$ when the switch is thrown to complete the circuit.
	
	\begin{center}\includegraphics{figures/q1.01.png}\end{center}
	
	\subitem a. $I(t) = \frac{V_0}{R}\cdot e^{-t/\tau}$
	\subitem b. $V(t) = V_0\left(1-e^{-t/\tau}\right)$
	\subitem c. $Q(t) = C\cdot V_0\left(1-e^{-t/\tau}\right)$
	\item What is the difference between active and passive circuit elements?
	\item What can an ideal current source do what a real one cannot?
	\item What can ideal voltage source do what a real one cannot?
	\item Draw the symbol for a polarized capacitor.
	\item Draw the symbol for a potentiometer.
	\item Draw an op-amp, label its terminals, and explain what each does.
	\item What’s another name for a conductor?
	\item Which way does current run in a diode?
	\item Describe what Ohm’s law tells us in your own words.
	\item What is conductance? State Ohm’s law using conductance. What does it tell us about eh current passing through conductive materials?
	\item What is the equation for resistance in terms of resistivity? If resistivity goes up, what happens to resistance? If the cross-sectional areas decreases, what results? If a resistor keeps the exact same cross-sectional area over a length L, yet we contorted it into a very weird shape (say a balloon-animal poodle), will its resistance remain the same according to the aforementioned equation. Does that agree with your intuition?
	\item A capacitor has a time variance of current according to the follow formulation $I(t) = \frac{dQ(t)}{dt} = C\frac{dV(t)}{dt}$. That might be useful to know. See if you can prove that relationship to yourself knowing that C = Q/V.
	\item What is an inductor and how does it work?
	\item What is impedance? What is impeded? What does the impeding?
	\item What is the real and imaginary component of impedance.	
	\item How would I represent impedance by a magnitude and phase?
	\item 	What is impedance’s inverse? What are its real and imaginary components?	
	\item What is the impedance of a resistor, capacitor, and inductor?
	\item Find the equivalent impedance of the following, being sure to remove imaginary numbers from the denominator [``+'' means ``in series with'', ``$\vert\vert$'' means ``in parallel with'']:
	\subitem a.	Z1$\vert\vert$(Z2 + Z3)
	\subitem b.	(R1 + C1)$\vert\vert$C2$\vert\vert$(R2 + R3 + C3)
	\subitem c.	(R1 + L1)$\vert\vert$(C1 + L1)$\vert\vert$(C1 + R1)
	\item Convert a Wye circuit, with all R = 10 ohms, to a Delta circuit.
	\item Find the equivalent impedance of a Delta circuit of capacitors.	
	\item Given the following map, can you plan a parade rate that crosses each bridge exactly once? Why or why not?
	
	\begin{center}\includegraphics{figures/q1.02.png}\end{center} 
	
	\item 32.	Prove the fundamental theorem of network topology given any circuit (for instance, those aforementioned).
	\item Kirchhoff’s current law is a mere recitation of what other law of physics?
	\item Kirchhoff’s voltage law is a mere recitation of what other law of physics?
	\item When do we most use KCL? When do we must use KVL? Which do you prefer in general and why?
	\item Draw a complicated circuit (at least four branches) and solve for the current going through each component.
	\item Analyze the aforementioned complicated circuit using nodal analysis. Analyze said circuit with mesh analysis. Prove the validity of the fundamental theorem of network topology using said circuit.
	\item What is the difference between earth ground and chassis ground?
	\item Draw a circuit using each of the following, R1 = 10 ohms, R2 = 20 ohms, R3 = 30 ohms, R4 = 40 ohms, V1 = 20 V, V2 = 10 V, and I1 = 2 A. Solve using nodal analysis if you can (do you need to make a supernode?). Solve using mesh analysis if you can (do you need to make a supermesh?).
	\item Can you imagine modeling some actions in terms of circuits? For instance, if you were asked to model an equivalent circuit based on the operations of a shop floor, could you do it? If you were asked to model some aspect of physiology could you? The resistance of blood to flow? The capacitance of the lungs? The duration of our blinks?
	\item What is Cramer’s rule and how do we use it?
	\item Draw a circuit with at least five resistors and two current sources. Put that circuit into matrix formation for nodal voltages from inspection alone. What values should be along the diagonal, what values should be on the off-diagonal?
	\item To solve the above circuit in MATLAB, what would you type in?
	\item For any circuit stated previously, add a voltage source to one of the branches and solve using nodal analysis. What new technique did you need to employ and why did you need to? What does that technique do to our matrix?
	\item To any previously mentioned circuit, add a dependent current source and let it be dependent on the current going through R1. How does this change your nodal voltages?
	\item Given a jolt of 240 J from an AED, provide an energy equivalent. For example, how many little soft kitten jumps on the chest (1 kg, 5cm) is it equivalent to? It would be like falling from a height of how much? [Hint: the ground comes up at you.]
	\item A hospital consumes 500 kW in 60 seconds. How many people running on treadmills would it take to power it for a minute?
	\item A nurse starts their shift turning on medical doohickey that consumes 50 W at 220 V. How many electrons’ worth of charge pass through the doohickey before the nurse turns it off at the end of their shift, 10 hours later?
	\item Explain how mesh analysis works.
	\item Draw a mesh circuit with at least 6 unique elements. Analyze it.
	\item Draw another mesh circuit with at least four loops. Analyze it.
	\item For the two above-mentioned mesh circuits, put them directly into matrix form. What values go along the diagonal? What values go on the off-diagonals?
	\item For one of the mesh circuits you drew in either, add a dependent current source and solve using mesh analysis.
	\item How do we measure current and potential using actual ammeters and voltmeters? Why must we “break” the circuit to measure current? How do we measure potential across an element?
	\item Say, in as much detail as you can muster, what you believe the Thevenin circuit theorem is about.
	\item Say, in as much detail as you can muster, what you believe the Norton circuit theorem is about.
	\item Changing a potential source to a current source is an example of what technique to analyze circuits?
	\item What are two attributes necessary for things to be ``linear''?
	\item What is superposition and how do we use it to analyze a circuit?
	\item How do you turn off a voltage source in a circuit diagram? How do you turn off a current source?
	\item To one of the above dozen or so circuits you have drawn, find the Thevenin equivalent.
	\item To one of the above dozen or so circuits you have drawn, find the Norton equivalent.
	\item To the Thevenin in 62, convert it to its Norton equivalent. To the Norton in 63, convert it to its Thevenin equivalent.\
	\item What is an open circuit? What is a short circuit? How are they relevant to Thevenin and Norton equivalents?
	\item Draw an LM741 and label all 8 pins.
	\item What is an operational amplifier? What does it behave like? What can it do for us? How are they different from resistors, capacitors, and their ilk?
	\item What are the three key features we need to know to analyze op-amp circuits?
	\item Draw an inverting amplifier with R1 = 20 ohms, and R2 = 100 ohms. What is the gain on the input signal?
	\item Draw an inverting amplifier with Z1 = R1 = (Ra$\vert\vert$Rb) and Z2 = R2 = (Rc + Rb). Solve for the output voltage in terms of the input voltage.
	\item Draw a non-inverting amplifier with R1 = 100 kohms and R2 = 100 kohms.
	\item Draw the two circuits you drew above in series with one another. What is the output?
	\item What is a voltage follower and why is it important?
	\item If it’s so important you should be able to draw one and analyze its operation! Let’s do that. Draw a voltage follower and demonstrate its behavior in as much detail as you would need to convince another engineer you know what you’re talking about.
	\item Why might we want to use a voltage follower with an electrocardiogram?
	\item What is a summing amplifier?
	\item A summing amplifier can actually act as a logic converter is set up right. Can you imagine using a set of summing amplifiers to determine a winner in a simple tic-tac-toe game in which button presses turned on new voltage sources? Briefly describe how you could determine a winner.
	\item If I have a summing amplifier in which V1 = 10 V, R1 = 10 ohms, V2 = –10 V, R2 = 10 ohms, and Rf = 20 ohms, what is my output voltage?
	\item What is a differential amplifier and why is it important?
	\item Draw a differential amplifier. Analyze said differential amplifier. What is the output voltage in terms of the input voltage?
	\item What is a transfer function?
	\item You know how to go from a triangle (Delta) to a y (Wye) circuit for equivalence. How would you find the equivalent of going from a square to a cross?
	\item 	Draw a blood vessel. Find the electrical equivalent across the vessel and through the vessel.
	\item Draw the meanest circuit you think I’d ever give you on an exam. Then add a voltage dependent voltage source to one branch of it (its dependence rests on you). Solve it using nodal analysis, mesh analysis, superposition, and just plain looking at it.
\end{enumerate}

\chapter{A set of exercises in preparation of the second Glorified Quiz}
03/19/2019, a review held
03/21/2019, a quiz given


\begin{enumerate}
	\item Find the transfer function (as V/I, I/V, Vo/Vin, etc.) of the following (``+'' means ``in series with'', ``$\vert\vert$'' means “in parallel with''):
	\subitem R1 + R2$\vert\vert$(R3 + C1)
	\subitem (R1 + R2)$\vert\vert$(R3 + C1 + L1)
	\subitem C1 + (C2 + R1$\vert\vert$(R2 + C3)$\vert\vert$L1)
	\subitem (R1$\vert\vert$R2$\vert\vert$R3)$\vert\vert$(R4$\vert\vert$R5$\vert\vert$R6)$\vert\vert$(R7 + R8$\vert\vert$R9)
	\item Label and describe the pins of an operational amplifier.
	\item Draw and derive an inverting amplifier.
	\item Draw and derive a non-inverting amplifier.
	\item Draw and derive a differential amplifier.
	\item Draw and derive an operational amplifier.
	\item What is the Laplace transform? What is its definition? What does it do for us? How does it do that? Why do we care? When can you use it? When can you not? When should you?
	\item What is the s-domain? What are its axes? What do those axes represent? When is looking at the s-domain useful for us?
	\item The s of the s-domain has two components. What are they and what do they individually and collectively represent?
	\item What is at least one condition (w/r/t s) that must exist for a system to be stable? 
	\item What is the Laplace transform of a constant (i.e., of 1)?
	\item What is the Laplace transform of an exponential function (i.e., of $e^{at}$)?
	\item What is the Laplace transform of a derivative (i.e., of dx/dt)? First, second, third, fourth, etc.?
	\item Is the Laplace transform linear? And if so, what does that mean?
	\item What is Euler’s formula? What is Euler’s identity? When might it be useful (especially in this class)?
	\item What is the differential equation of a series RLC circuit? [I will take “series RLC circuit” here to mean a resistor, an inductor, and a capacitor in series.]
	\item What is the differential equation of a parallel RLC circuit? [I will take “parallel RLC circuit” here to mean a resistor, an inductor, and a capacitor in parallel.]
	\item What is the differential equation of a series RLC circuit in parallel with another series RLC circuit?
	\item What the inverse Laplace transform?
	\item Make sure you can invert a simple Laplace transformation (say of an RLC circuit) to yield the time-domain equation. For example:
	\subitem $1/(s^2+5s+6)$
	\subitem $(s+2)/(s+3)(s+4)$
	\subitem $s^2/(s(s^2+4s+4))$
	\subitem $(s+1)/(s^1+1)$
	\item For the s-domain representations show above, what sort of frequency dependent behavior do you expect each to have? How do you know?
	\item We have learned the general form the denominator (the poles) that many transfer functions take is: $s^2 + 2\zeta \omega_n s + \omega_n^2$. What is $\zeta$? What is $\omega_n$?
	\item Describe the behavior of a system when $\zeta$ is (1) zero, (2) between zero and one, (3) one, and (4) greater than one. Include graphs. Include solutions to the differential equations thus described. Be sure you could identify such behavior if given a graphical representation.
	\item A charged capacitor (charge it to whatever value you’d like) is dropped in series with a resistor at time t = 0. Use KCL to find first-order differential equation. What is its solution?
	\item The solution to the above example has a time constant, $\tau$. What is it (in terms of the resistor and capacitor terms)? What is the amplitude of the response when t = 1$\tau$? When t = 2$\tau$? 3$\tau$? 4$\tau$? 5$\tau$? 10$\tau$?
	\item If one were to take the derivative of the solution to 24 at t = 0, the slope of the line would intersect time axis at what point? Why might this be useful?
	\item If there is no excitation signal (as in 24), but there are initial conditions, what kind of response is this, ZIR or ZSR?
	\item BTW, what’s ZIR \& ZSR?
	\item A voltage source, a resistor, and a switch are in series with (R2 + C1)$\vert\vert$R3. What is the differential equation of the circuit? What is its Laplace transformation? What is its solution?
	\item What is partial fraction decomposition and why should I know it?
	\item For the circuit at right, assume the initial condition of the inductor’s current is i(0) = I0. Apply KVL around the loop to find the first-order differential equation. Solve said equation. What is the time constant?
	\item If instead of charging the inductor in the above example, we removed the charge and put in a time varying voltage signal and measured the voltage drop across the resistor, would we record be recording the zero state response, the zero input response, or some combination of the two?
	\item For the circuit below, assuming i(0) = 10 A, calculate i(t) and ix(t).
	\begin{center}\includegraphics{figures/q2.02.png}\end{center}
	\item Below is what’s called a “gate function”, which is like a step function that switches on at one value of t and switches off at another. Express this voltage pulse (seen at right) in terms of the unit step. Calculate its derivative.
	\begin{center}\includegraphics{figures/q2.03.png}\end{center}
	\item What is the definition of a unit step function?
	\item What is the definition of a unit impulse function? How is it related to the unit step function?
	\item What is the definition of a unit ramp function? How is it related to the unit step function?
	\item For 35-37, draw a graphical representation of the functions.
	\item If you wanted to shift a unit step/impulse/ramp function by an amount t0 in time, how does this alter your definition of the function as reported above?
	\item If you multiplied a function f(t) by an impulse function that has been time shifted (by an amount t0), what is the result? 
	\item If the switch of the circuit shown below is closed at t = 0, find the time-dependent voltage function for the circuit for both t $<$ 0 and t $\geq$ 0.
	\begin{center}\includegraphics{figures/q2.05.png}\end{center}
	\item The above example has two responses, a natural response (using stored energy) and a forced response (using an independent source). Describe these in your own words.
	\item One could also decompose the responses by a transient response and a steady-state response. Describe these in your own words.
	\item Describe convolution in your own words.
	\item What is the formal definition of convolution?
	\item Draw a (simple) system function and a (simple) input function. What is the convolution of your input with your system? 
	\item Convolution with the impulse response produces what?
	\item What are the commutative, associative, and distributive properties of convolution?
	\item What is the relationship between convolution and the Laplace transform? (Similarly, what is the relationship between convolution the Fourier transform?)
	\item Draw a passive low-pass circuit. What is the transfer function between Vo/Vin? What is the cutoff frequency?
	\item Draw a passive high-pass circuit. What is the transfer function between Vo/Vin? What is the cutoff frequency?
	\item Draw a bode plot for a first-order passive low-pass filter. What is the pass band? What is the stop band? What is the slope of the tracing within the stop band? At the corner frequency, what is the amplitude in dB? Draw a corresponding phase diagram. What is the phase shift one order of magnitude below the cutoff frequency? What is the phase shift one order of magnitude above the cutoff frequency?
	\item Draw a bode plot for a first-order passive high-pass filter. List all the ways it differs from the first-order passive low-pass filter of 50.
	\item Draw a bode plot and phase diagram for the series combination of a first-order passive low-pass filter with a cutoff frequency of 50 Hz and a first-order passive low-pass filter with a cutoff frequency of 500 Hz.
	\item Redo the above questions but using ``second-order'' instead of ``first-order'' filters. How does this change your results?
	\item Why do we care about the phase of a signal?
	\item Draw the bode plot and phase diagram of the following transfer functions:
	\subitem $100(s + 1) / (s + 10)(s + 100)$
	\subitem $(s^2 + 27s + 100) / (s^2 + 4s + 4)$
	\subitem $(s(s + 100)) / (s(s + 1)(s + 10))$
	\subitem $s^2 / (s^3 + 2s^2 + 3s + 4)$
	\item What are the general forms of the transfer functions for low-pass, high-pass, and band-pass filters?
	\item What is an integrator? How would you make one with an operational amplifier? What is the Laplace transform of an integrator / integration? What is one advantage and disadvantage to using an ideal inverting integrator?
	\item We can modify the ideal inverting integrator by placing a resistor in the feedback loop of the op-amp. What does this produce? Sketch a bode plot relating Vo/Vin for such a modified inverting integrator.
	\item Given real values for the modified inverting integrator, could you determine the frequency at which you would expect the amplitude of a signal in the stop-band to be half the value one would expect to see in the pass-band?
	\item What is a differentiator? How could you make one with an operational amplifier? What is the Laplace transform of a differentiator / differentiation? What is one advantage and disadvantage to using an ideal inverting differentiator?
	\item We can modify the ideal inverting differentiator by inserting an input resistor (generally before the capacitor). What does this produce? Sketch a bode plot relating Vo/Vin for such a modified inverting differentiator.
	\item What is the input impedance of the following system? (Note, this is from an actual bit of implantable electronics.)
	\begin{center}\includegraphics{figures/q2.01.png}\end{center}
	\item What is a plant? What is a controller? What is a system?
	\item What is open-loop control? What is closed-loop control?
	\item Given a simple block diagram with F = 1 / (s + a), G = s / (s + b), and H = 1 / s; what is the overall system response (output / input)?
	\item What is proportional, integral, and derivative control?
	\item What is the s-domain representation of P, I, and D control?
	\item How can we make for a stable arms race?
\end{enumerate}

\newpage

\section{Exam 2 Review Board Pictures (3/19/19)}

The glares made parts of the board difficult to see. 
All the notes below are hand renderings of board pictures taken today.

\includegraphics[width=\textwidth]{figures/3-19_Board1.png}
\\
\includegraphics[width=\textwidth]{figures/3-19_Board2.png}
\\
\includegraphics[width=\textwidth]{figures/3-19_Board3.png}
\\
\includegraphics[width=\textwidth]{figures/3-19_Board4.png}
\\
\includegraphics[width=\textwidth]{figures/3-19_Board5.png}
\\
\includegraphics[width=\textwidth]{figures/3-19_Board6.png}
\\
\includegraphics[width=\textwidth]{figures/3-19_Board7.png}

\chapter{A set of exercise in preparation for the third Glorious Quiz}

\section{Basic analysis}
\begin{center}
	\includegraphics[width=\textwidth]{figures/q3.09.png}
\end{center}

\newpage

\section{A wound wire or two}
\begin{center}
	\includegraphics[width=\textwidth]{figures/q3.10.png}
\end{center}

\newpage

\section{Potential across an inductor}
\begin{center}
	\includegraphics[width=\textwidth]{figures/q3.11.png}
\end{center}

\newpage

\section{An operational amplifier's bandwidth}
\begin{center}
	\includegraphics[width=\textwidth]{figures/q3.06.png}
\end{center}

\newpage

\section{The phase shift of an amplifier}
\begin{center}
	\includegraphics[width=\textwidth]{figures/q3.07.png}
\end{center}

\newpage

\section{Designing an arbitrary op amp circuit}
\begin{center}
	\includegraphics[width=\textwidth]{figures/q3.38.png} 
\end{center}

\newpage

\section{A gyrator}
\begin{center}
	\includegraphics[width=\textwidth]{figures/q3.39.png} 
\end{center}

\newpage

\section{Transfer function from a circuit, voltage}
\begin{center}
	\includegraphics[width=\textwidth]{figures/q3.29.png} 
\end{center}

\newpage

\section{Transfer function from a circuit, voltage, again}
\begin{center}
	\includegraphics[width=\textwidth]{figures/q3.31.png} 
\end{center}

\newpage

\section{Transfer function from a circuit, current}
\begin{center}
	\includegraphics[width=\textwidth]{figures/q3.30.png} 
\end{center}

\newpage

\section{Transfer function from a circuit, impedance}
\begin{center}
	\includegraphics[width=\textwidth]{figures/q3.33.png} 
\end{center}

\newpage

\section{Even more transfer functions, emphasis on the \textit{fun}}
\begin{center}
	\includegraphics[width=\textwidth]{figures/q3.15.png}
\end{center}

\newpage

\section{Sketching Bode plots}
\begin{center}
	\includegraphics[width=\textwidth]{figures/q3.32.png} 
\end{center}

\newpage

\section{Ordinary differential equations}
\begin{center}
	\includegraphics{figures/q3.13.png} \\
	\includegraphics{figures/q3.14.png}
\end{center}

\newpage

\section{A high pass}
\begin{center}
	\includegraphics[width=\textwidth]{figures/q3.34.png} 
\end{center}

\newpage

\section{A low pass}
\begin{center}
	\includegraphics[width=\textwidth]{figures/q3.35.png} 
\end{center}

\newpage

\section{An inverting integrator}
\begin{center}
	\includegraphics[width=\textwidth]{figures/q3.26.png}
\end{center}

\newpage

\section{Playing with a given transfer function}
\begin{center}
	\includegraphics[width=\textwidth]{figures/q3.01.png}
\end{center}


\section{Playing with another given transfer function}
\begin{center}
	\includegraphics[width=\textwidth]{figures/q3.02.png}
\end{center}

\newpage

\section{Playing with yet another given transfer function}
\begin{center}
	\includegraphics[width=\textwidth]{figures/q3.03.png} \\
	\includegraphics[width=\textwidth]{figures/q3.04.png}
\end{center}

\newpage

\section{Poles and zeros}
\begin{center}
	\includegraphics[width=\textwidth]{figures/q3.05.png}
\end{center}

\newpage

\section{System response from poles alone}
\begin{center}
	\includegraphics[width=\textwidth]{figures/q3.16.png}
\end{center}

\newpage

\section{Design, design, design, and design}
\begin{center}
	\includegraphics[width=\textwidth]{figures/q3.08.png}
\end{center}

\newpage

\section{Bridges and amplifiers}
\subsection{Example 1}
\begin{center}
	\includegraphics[width=\textwidth]{figures/q3.20.png} \\
	\includegraphics[width=\textwidth]{figures/q3.21.png}
\end{center}
\newpage
\subsection{Example 2}
\begin{center}
	\includegraphics[width=\textwidth]{figures/q3.22.png} \\
	\includegraphics[width=\textwidth]{figures/q3.23.png}
\end{center}
\newpage
\subsection{Example 3}
\begin{center}
	\includegraphics[width=\textwidth]{figures/q3.24.png} 
\end{center}

\newpage

\section{Bumping the jams}
\begin{center}
	\includegraphics[width=\textwidth]{figures/q3.36.png} 
\end{center}

\newpage

\section{Convolution}
\begin{center}
	\includegraphics[width=\textwidth]{figures/q3.12.png}
\end{center}

\newpage

\section{A pulse into a circuit}
\begin{center}
	\includegraphics[width=\textwidth]{figures/q3.37.png} 
\end{center}

\newpage

\section{``Silent'' knights and knaves}
\begin{center}
	\includegraphics[width=\textwidth]{figures/q3.19.png}
\end{center}

\newpage

\section{A block diagram}
\begin{center}
	\includegraphics[width=\textwidth]{figures/q3.25.png}
\end{center}

\newpage

\section{The heart of an electrocardiogram}
\begin{center}
	\includegraphics[width=\textwidth]{figures/q3.27.png} \\
	\includegraphics[width=\textwidth]{figures/q3.28.png}
\end{center}

\newpage

\section{Current through a cell}
\begin{center}
	\includegraphics[width=\textwidth]{figures/q3.17.png} \\
	\includegraphics[width=\textwidth]{figures/q3.18.png}
\end{center}

\newpage
Overview of Medical regulations in the US and the rest of the world \\

Links on this page lead to more information and an overview of the FDA's guidelines and changes over the 20th century, as Professor Belmont lectured on the lecture before our 3rd Glorified Quiz. As said in lecture, the US leads the world in the regulation of medical devices and these sites give overviews of that and how other major contributers to the medical world regulate in their own respective countries. \\

The FDA's own overview of their changes medical device regulation in the United States

https://www.fda.gov/MedicalDevices/DeviceRegulationandGuidance/Overview/ucm618375.htm


The next biggest influence in the world on medical device regulation is Europe whom has the Europe Medicines Agency(EMA) to regulate medical devices across the continent. It is useful to compare the legislation between these different powers to see how compromise can be made across the world to improve clarity across different powers of the world and their medical legislation.

https://www.ema.europa.eu/en/human-regulatory/overview/medical-devices


The United kingdom has their own regulation agency within Europe known as the Medicines and Healthcare products Regulatory Agency(MHRA) that regulates medical devices and other forms of medicine in the UK. This provides a could crossroads between a very large influence such as the FDA and the EMA to a smaller view of just the United Kingdom, which is not under the jurisdiction of the EMA

https://www.gov.uk/government/publications/report-a-non-compliant-medical-device-enforcement-process/how-mhra-ensures-the-safety-and-quality-of-medical-devices


\end{center}

\newpage
Extra Practice for Logic Problems

Website with problems for Knights and Knaves questions:
https://philosophy.hku.hk/think/logic/knights.php

Website with problems for Logic gates:
https://www.allaboutcircuits.com/worksheets/basic-logic-gates/

\newpage

\section{A few practice problems submitted by a colleague}

\begin{enumerate}
	\item True or False: A forward biased diode has current flowing through it and the potential difference increases. 
	\subitem \textbf{Answer:} False; the potential difference in the diode is smaller when it is forward bias because the positive terminal of the circuit is connected to the P-type material while the negative terminal is connected to the N-type material, which decreases the PN junction width and therefore the electric potential.
	\item When does resonance occur with an inductor and capacitor?
	\begin{enumerate}
		\item When impedance is at its minimum and the elements are in series
		\item When impedance is at its maximum and the elements are in series
		\item When impedance is at its minimum and the elements are in parallel
		\item When impedance is at its maximum and the elements are in parallel
		\item Both a and c
		\item Both b and d
	\end{enumerate}
	\subitem \textbf{Answer:} e) Resonance occurs when the inductance values of the two elements are equal, which happens at the minimum impedance in series or at the maximum impedance in parallel
	\item What does is the purpose of the -Vs and +Vs terminals on an operational amplifier?
	\subitem \textbf{Answer:} The potential at the -Vs terminal sets the lower pound of the final voltage potential while the +Vs terminal sets the upper bound. The  resulting voltage potential from the op amp cannot exceed either of these extremities.  
	\item What are the voltage transfer functions for inverting and non-inverting amplifiers?
	\subitem \textbf{Answer:} For inverting op amps: $(Vo/Vi) = -(Rf/Ri)$, and for non-inverting op amps: $(Vo/Vi) = 1 + (Rf/Ri)$
	\item What is an assumption we must make before using Norton and Thevenin equivalents to solve a circuit?
	\subitem \textbf{Answer:} We must assume that the circuit is linear, meaning that it abides by two rules: homogeneity (if $V=IR$, then $cV = cIR$) and additivity (if $V1 = I1*R1$ and $V2 = I2*R2$, then $Vt = V1 + V2 = I1*R1 + I2*R2$). 
	\item How many poles does the following transfer function contain and what are they: $(s-5)/(((s-6)^2)*(s-3))$
	\begin{enumerate}
		\item 2 poles at s = 3,6 
		\item 3 poles at s = 3,5,6 
		\item 1 regular pole at s = 3 and 1 repeating pole at s = 6 
	\end{enumerate}
	\subitem \textbf{Answer:} c) The term in the numerator (s-5) will give you a zero at s = 5, while the terms in the denominator would give us a pole at s = 3 and a repeating pole at s = 6.
	\item Find the time dependent voltage equation $v(t)$ given that its Laplace transform  $V(s) = 5/((s-6)*(s-3)^2)$
	\subitem \textbf{Answer:} First step is to perform partial fraction decomposition
	\begin{itemize}
		\item $5/((s-6)*(s-3)^2) = (A)/(s-6) + (Bs+C)/(s-3)^2$
		\item $5 = A(x-3)^2 + B(s-6)*(s-3) + C(s-6)$  by plugging in $s = 6$, we get that $A = 5/9$; when $s= 3$, $C = -5/3$
		\item By plugging in for $A$ and $C$, we get $5 =  (5/9)(x-3)^2 + B(s-6)*(s-3) + (-5/3)(s-6)$, from which we can get $B = -5/9$
		\item New $V(s) = -5/(9(s-3)^2) - 5/(3(s-3)) + 5(9(s-6))$
		\item by taking the inverse laplace transform, we get: $v(t) = (e^(3t)/9)(-15 + 5e^(3t) -5)$
	\end{itemize}
	\item True or False: We dope silicon with impurities in order to increase or decrease its conductivity
	\subitem \textbf{Answer:} True; Silicon molecules are in crystal structures in which they are all bound to 4 other silicon molecules. When adding an impurity that has more than 4 electrons in its valance shell, 4 of the electrons will bind to the silicon but the other will remain free to conduct an electrical charge. 
	\item Biomedical devices are classified into 3 ``classes.'' What are they?
	\subitem \textbf{Answer:} Class I devices require a general control that  must extend to all devices. These types of devices already exist and have been proved to not generate risks of any kind to the user. Class II devices are considered to only expose the user to a moderate risk and they must be proven to function as intended (can be done by similarity to current product). Finally, class III devices must be critical to a person's suvival or have high potential risks of ending a person's life. 
	\item Given the voltage transfer function $s(s-3)^2/(s-5)$, what will be the end-behavior slope of the Bode plot representing the transfer function?
	\subitem Answer: Since every zero adds 20dB/dec to the slope of the graph and every pole subtracts 20dB/dec, the slope = 20 + 2*(20) -20 = +40 dB/dec
\end{enumerate}

\newpage


\end{document}
