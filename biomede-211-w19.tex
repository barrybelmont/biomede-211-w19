\documentclass[11pt]{book}
\usepackage{amssymb,amsmath}
\usepackage{minitoc}



\begin{document}
\frontmatter
\dominitoc
\tableofcontents



\section{How can I print off and use this document?}
Frankly, in just about any way that’s useful to you. I am going to try something here, where I will try to make more or less the entirety of the notes associated with the Winter 2019 semester of BIOMEDE 211, ”Circuits, Systems, and Signals in Biomedical Engineering”, to you, dear reader.
\\
Please don’t plagiarize this. If you were raised right, you ought to know what that is. If you’d like my judgment on any sort of action, my opinions can be laid bare.
\\
The first assignment I am giving you (worth 4\% of your grade and which must be completed by the end of the semester) is to figure out where this document is located online, download it, print it off, sign your name to it, and get it to me. If you know who I am, I would expect a competent engi- neer to find that without much to-do about it. Start with Google, go from there. Further, for those in the class, BIOMEDE 211, Winter 2019, you must join Github and make at least four substantive contributions to this repository. The term all you engineers (and lawyers) can’t wait to parse is “substantive” to which I will always enter a judgment which I deem final in this class, and I am ever in favor of beneficence over stricture. So, just help out the class in a way you think is helpful and watch those around you do the same. Failure to contribute to this living document by the end of the semester for those in this class will result in a loss of up to 4\% of one’s total grade outright.


\section{On the first day, we will introduce ourselves.}



\newpage



\mainmatter
\setcounter{page}{1}



\part{Circuits}



\chapter{I. Potential, current, energy, conservation}
01/10/2019
\minitoc



\section{What is electricity?}

\begin{enumerate}
	\item A form of energy arising from the existence of charged particles. 
	\item A physical phenomenon; motion or presence of matter that has charge.
	\item Ill-defined.
\end{enumerate}



\section{Charge}
 
\begin{enumerate}
	\item Charge is the property of matter that causes it to experience a force when placed in an electromagnetic field; measured in coulombs (C)
	\item 	
	\item \textbf{How many electrons are needed to form one coulomb?} (What is the weight of all those electrons?) 
	\item One byte is eight bits. Bits are essentially a single electron stored in a transistor. \textbf{If we were to take all the electrons from one terabyte of well distributed information (equal number of ones and zeros), how many coulombs would we have?}
\end{enumerate}



\section{Current}

\begin{enumerate}
	\item The time rate of charge (charged particles) in motion; measured in amperes (A); defined mathematically as
	\begin{equation}
	\label{i=dqdt}
		i := dq/dt
	\end{equation}
	where $i$ is current, $q$ is charge, and $t$ is time
	
	\item Conversely, the total charge transferred over time can be expressed as 
	\begin{equation}
	\label{q=it}
		Q := \int_{t_0}^{t}i dt
	\end{equation}

	\item 1 ampere is equal to 1 coulomb/second
	\item Direct current, ``DC''
	\item Alternative current, ``AC''
\end{enumerate}


\subsection{The directionality of current}
Ultimately, the direction in which we say "current" flows is largely arbitrary. As arbitrary as choosing one type of charge and calling it ``positive'' and another ``negative''. The reason it doesn't matter is that the only consequence of having chosen a ``wrong direction'' for the current in a given analysis is that we have to switch the sign of the value. Thus, 3 amps in one direction is \textit{the exact same thing }as -3 in the opposite direction.
\begin{enumerate}
	\item Thanks to Benjamin Franklin we say that current is 
	\subitem i.	\textbf{Positive in the direction in which positively charged particles flow} and 
	\subitem ii.	\textbf{Negative in the direction in which negatively charged particles}
	\subitem iii.	We also now know that current results primarily from the movement of negatively charged particles (electrons) and therefore our convention is “wrong” in one sense, though convenient and entrenched enough that we’re not liable to change it in our life time (besides, the math comes out the same, and the actual flow of electrons will only matter to us in a few special circumstances, diodes)
\end{enumerate}


\subsection{The at times deadly serious nature of current}
Much of the point of learning this material here is its eventual application by our hands or by the hands of those we work with. Before we put any of this stuff in our hands, we should probably know what is and is not safe.
\begin{enumerate}
	\item 1 mA
	\item 10 mA
	\item 100 mA
	\item 1000 mA
\end{enumerate}


\subsection{The ``speed'' of current}
\pagebreak



\section{Potential (difference)}

\begin{enumerate}
	\item The amount of work needed to move a unit of (positive) charge from a reference point to another point [without producing an acceleration]).
	\item Potential is measured in ``volts'' and is often called ``voltage''. In this class we will endeavor to avoid such a term as it can be very confusing to talk about potential as if there were such a \textit{thing} as voltage.
	\item Defined as 
	\begin{equation}
		v:= \frac{dw}{dq}
	\end{equation}
	\item Potential describes the \textit{potential} to do something. Increasing the potential is akin to increasing the height of a cliff. The height does not \textit{do} anything other than increase what can be done on the drop. If potential is the cliff's height, charge would be pebbles you'd drop off the side, and current describes how fast those pebble fall.
	\item In this class, and for the vast vast majority of electrical engineering work, we care about the \textit{difference} in potential. One element held at 100 billion volts and another held at 100 billion + 1 volts has a \textit{potential difference} of 1 V, which is less than a single AA battery.
	\item Some typical voltages to be aware of
	\subitem \textbf{Consumer level batteries} (AA, AAA): 
	\subitem \textbf{Car batteries}: 
	\subitem \textbf{The ``mains''} (levels provided by power companies to consumers):
	\subitem \textbf{Power transmission lines}: 
\end{enumerate}



\section{Power}

\begin{enumerate}
	\item The time rate of expending or absorbing energy.
	\item Quantifies the rate of energy transfer.
	\item Mathematically:

	\item Measured in watts: 1 W = 
	\item \textbf{Passive sign convention}: 
\end{enumerate}



\section{Energy}

\begin{enumerate}
	\item 
	\item 
	\item 
	\item 
\end{enumerate}



\section{Conservation}
Here, as elsewhere, things will be conserved. In electrical circuits there are two laws of conservation that will matter most for us:
\begin{enumerate}
	\item \textbf{The Conservation of Mass.} 
	\item \textbf{The Conservation of Energy.} 
\end{enumerate}

In evaluating circuits, the main focus of the first third of this class, it will be the application of these two conservative laws that will enable us to ``solve'' them. That is, by understanding (1) how energy is generated and used and (2) how charges move around in closed loops (``circuits'') we will be able to predict the behavior of the myriad electrical systems which may cross our paths.


 
\newpage



\section{Worksheet}

\subsection{Problem 1, constant charge through a cross-section}
How much charge passes through a cross-section of a conductor in 60 seconds if a DC current value is measured at 0.1 mA?
\textbf{Solution}


\subsection{Problem 2, arbitrary charge through a cross-section}
Determine the total charge entering a terminal between $t = 0$ seconds and $t = 10$ seconds if the current (in amps) passing through is
\begin{equation}
	i(t) = \frac{1}{\sqrt{5t+2}}. 
\end{equation} 
\textbf{Solution}


\subsection{Problem 3, a "tera"ble puzzle}
Approximately how much current is necessary to transmit one terabyte of information in an hour?
\textbf{Solution}


\subsection{Problem 4, power necessary to run a pacemaker}
A cardiac pacemaker will provide approximately 5,000 J of energy over 5 years. Determine the capacity of a 5 V lithium battery necessary to drive this pacing such that only 40\% of its energy is spent over that time.
\textbf{Solution}


\subsection{Problem 5, energy needed to excite a neuron}
A colleague of yours has been in their lab ginning up new neurons. You, as their resident electrical expert, are tasked with determining the energy consumed by the cell. If the current and voltage variations are found to be functions of time ($t \geq 0$)
\begin{eqnarray}
	i(t) = 3t \\
	v(t) = 10 e^{6t}
\end{eqnarray}
determine the energy consumed between 0 and 2 ms.
\textbf{Solution}


\subsection{Problem 6, a thump to the chest}
(a) A typical defibrillator delivers 200-1000 V in less than 10 ms. How much current is needed to deliver 120, 240, and 360 Joules?
\\
(b) A human heart ways about 300 grams. From approximately how high of a cliff would one have to drop a heart such that the impact was equivalent to the energy delivered to someone's chest from a defibrillator?
\textbf{Solution}



\chapter{II. Circuit elements}
01/15/2019
\section{Active v. passive}
\section{Ohm's Law and what it means}
\section{Sources}
\section{Resistors}
\subsection{Resistance, $R$}
\subsection{Resistivity, $\rho$}
\subsection{Conductance}
\section{Capacitors}
\subsection{Its time varying behavior}
\subsection{Charge accumulation}
\subsection{A simple example}
\section{Inductors}
\section{Impedance}
\subsection{A quick note on ``imaginary'' numbers}
\section{Equivalent impedance}
\subsection{Impedances in general}
\subsection{Resistors}
\subsection{Capacitors}
\subsection{Delta-Wye ($\Delta$-$Y$) transformations}
\subsection{A few examples}
\section{Grounds}
\section{Conductors}
\section{Operational amplifiers}
\section{Diodes}
\section{Switches}
\section{Transistors}
\section{Transformers}
\newpage
\section{Worksheet}
\subsection{Problem 1, expressing power in ohms}
\subsection{Problem 2, a couple toaster based problems}
\subsection{Problem 3, currently conducting power}
\subsection{Problem 4, conductance of a sodium channel}
\subsection{Problem 5, resistance of a simple tissue}



\chapter{III. Operational amplifiers}
01/17/2019 
\section{Some details}
\section{Some rules}
\section{Some conveniences}
\section{Some examples}
\subsection{Inverting amplifier}
\subsection{Non-inverting amplifier}
\subsection{Voltage follower}
\subsection{Summing amplifier}
\subsection{Differential amplifier (as homework)}



\chapter{Circuit analysis: I. Nodal analysis}
01/22/2019
\section{Nodes and branches}
\section{Kirchhoff's Laws}
\subsection{Kirchhoff's Current Law}
\subsection{Kirchhoff's Voltage Law}
\section{Nodal analysis}
\section{Solving simultaneous equations}
\subsection{Cramer's Rule}



\chapter{Circuit analysis: II. Mesh analysis; Homework I}
01/24/2019
\section{Mesh analysis}
\section{Steps of mesh analysis}
\section{Writing mesh equations directly in matrix form}



\chapter{Circuit analysis: III. Supernodes and supermeshes}
01/29/2019 – Lecture 6. 
\section{Nodal analysis with an independent current source}
\section{Nodal analysis with voltage sources, \textbf{Supernodes}}
\section{Nodal analysis with controlled sources}
\section{Mesh analysis with current sources}
\section{Mesh analysis with controlled sources, \textbf{Supermeshes}}



\chapter{Circuit analysis: IV. Circuit theorems}
01/31/2019 – Lecture 7. 
\section{Circuit theorems}
\section{Linearity}
\section{Superposition}
\section{Source transformation}
\section{Thevenin equivalents}
\section{Norton equivalents}
\section{Equivalents with dependents}



\chapter{Circuit analysis: V. When to choose between analyses}
02/05/2019 – Lecture 8. 

\chapter{A review of the material thus far; Homework II}
02/07/2019 – Lecture 9. 

\section{How to measure voltage and current}

\chapter*{Exam I}
02/12/2019



\part{Systems}



\chapter{The Laplace Transform: I. What it is and why it is important}
02/14/2019 – Lecture 10. 
\section{How do we know our world looks like this?}
\section{Euler's identity / Euler's formula}
\section{The Laplace transform}
\section{The Laplace transform of 1}
\section{The $s$-plane}
\section{The linearity of the Laplace transform}
\section{The Laplace transform of $e^{at}$}
\section{The Laplace transform of $dx/dt$}
\section{The Laplace transform in RLC circuits}
\subsection{Resistors}
\subsection{Inductors}
\subsection{Capacitors}
\subsection{RLC}
\section{Two important places, zeros and poles}



\chapter{The Laplace Transform: II. How to use it}
02/19/2019 – Lecture 11. 
\section{The inverse Laplace transform}
\section{The Laplace transform of $\sin$}
\section{The Laplace transform of $t^n$}
\section{Some applicability}



\chapter{Circuits as ODEs: I. First-order}
02/21/2019 – Lecture 12. 
\section{Source-free RC circuits}
\subsection{One resistor, one capacitor}
\subsection{Two or more resistors and/or capacitors}
\section{Source-free ``active'' circuits}
\section{First-order systems with sources}
\section{Several singular functions}
\subsection{Unit step function, $u(t-t_0) = 1, t>t_0$}
\subsubsection{The Laplace transform of the unit step function}
\subsection{Unit impulse function, $\delta(t) = du(t)/dt$}
\subsubsection{Its ``sifting'' abilities}
\subsubsection{The Laplace transform of the unit impulse function}
\subsection{Unit ramp function, $r(t) = \int u(t)dt $}
\subsubsection{The Laplace transform of the unit impulse function}



\chapter{Circuits as ODEs: II. Second-order}
02/26/2019 – Lecture 13. 
\section{A series RLC circuit}



\chapter{System response: I. Convolution; Homework III}
02/28/2019 – Lecture 14. 
\section{An introduction to thinking in systems}
Viewing everything as a ``system''. 
\subsection{Domains of interest, of command}
\subsection{The time-domain, or: our typical realm}
\subsection{The frequency-domain, or: our new realm}
\subsection{The $s$-domain, or: our magical realm}
\section{Inputs and outputs}
\section{Somewhere in the between}
\section{Convolution in the time-domain}
\section{Multiplication in the frequency- and $s$-domain}



\chapter{System response: II. Stability}
03/12/2019 – Lecture 15. 
\section{An introduction}
\subsection{What do we mean by stability?}
\section{Undamped, $\zeta = 0$}
\section{Underdamped, $0 < \zeta < 1$}
\section{Overdamped, $\zeta > 1$}



\part{\& Signals}



\chapter{System response: III. The frequency domain}
03/14/2019 – Lecture 16. 



\chapter{System response: IV. Filters}
03/19/2019 – Lecture 17. 



\chapter{System response: V. Feedback; Homework IV}
03/21/2019 – Lecture 18. 



\chapter*{Exam II}
03/26/2019



\part{in Biomedical Engineering}



\chapter{Bioelectricity: I. Passive properties}
03/28/2019 – Lecture 19. 
\section{Modeling biological material with a simple circuit, $R_1 + (R_2||C)$}
\section{Resistance-Reactance Plane}
\section{What can we do with this information?}



\chapter{Bioelectricity: II. Active properties}
04/02/2019 – Lecture 20. 



\chapter{Bioelectricity: III. Measurement}
04/04/2019 – Lecture 21. 



\chapter{Digital circuits: I. Discretization}
04/09/2019 – Lecture 22. 



\chapter{Digital circuits: II. Logic; Homework V}
04/11/2019 – Lecture 23. 



\chapter{Happenstance: A few BME specific situations}
04/16/2019 – Lecture 24. 



\chapter{Circumstance: A few BME specific standards}
04/18/2019 – Lecture 25. 



\chapter{A philosophy of circuits, systems, and signals; Homework VI}
04/23/2019 – Lecture 26. 



\chapter*{Exam III}
04/26/2019 



\end{document}
